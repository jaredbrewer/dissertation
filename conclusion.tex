% The \vspace{} command in this chapter is just for aesthetic reasons - I don't like something new to start at the last line 
%of the page

% ONE OF THE BEST ONLINE LATEX REFERENCES IS AT :
% http://www.eng.cam.ac.uk/help/tpl/textprocessing/latex_advanced/latex_advanced.html

%% ALL figures are in EPS format: It is the best possible format 

Some ideas:

\begin{itemize}

\item The Zebrafish Mincle
\item Sufficiency
\item Cotranscriptional interplay
\item Integration of HIF-1\textalpha{} signaling
\item Lymphangiogenesis
\item Aspects that differentiate the isoforms in macrophages
\item Effect of NFAT mutation on macrophage biology
\item Mycobacterial interactions with other vasculature-relevant features -- plasmin, TIE2, fibronectin, etc.
\item Role of NFAT in neutrophils
\item New tools to study this pathway \textit{in vivo}
\item Generalizability
\item Novel Methods to Automate Measurement of Angiogenesis
\item Novel Methods to Automate Cell Feature Quantitation
\item Other Contributions of NFAT to Host-Mycobacterial Interactions
\item Promise as a Host-Directed Therapy

\end{itemize}

At the conclusion of the present work, many new questions have been generated while others remain unanswered. This work has accomplished two primary goals: addressing the intracellular signaling pathway within macrophages that is responsible for inducing angiogenesis during mycobacterial infection (the NFAT pathway) and setting the stage for future work to simplify and automate common procedures commonly used in the analysis of imaging data relevant to both zebrafish and tissue culture research. Some of these lingering questions will be addressed in the coming weeks and months while others stretch over the course of many years or decades as we delve into deeper and deeper understandings of the fundamental processes governing the nature of the angiogenic response to tuberculosis infection and how this can be a fruitful target for therapeutic intervention. Some of the questions remain largely technical: how are, specifically, zebrafish able to detect TDM when they lack an obviously annotated homolog to the human MINCLE/MCL? Others are important questions on the nature of the cell biology of these processes: what makes NFATC2 special in macrophages, is NFATC2 sufficient to induce VEGF, how does NFATC2 alter other aspects of macrophage behavior and biology, and how does NFAT intersect with HIF-1\textalpha{} signaling? These questions, among many others, are the subject of this concluding chapter; it is hoped that a comprehensive presentation of these questions will stimulate future generations to pursue answers and that these will further inform our understanding of the pathogenesis of tuberculosis.

\section{The Zebrafish Mincle}

Data from human cell culture and mice has implicated MCL and MINCLE as the primary C-type lectin receptors for TDM, which induces a variety of downstream responses, seemingly including the upregulation of VEGF and downstream angiogenesis (see \autoref{chap3}). However, the precise identity of the homolog of MCL or MINCLE in the zebrafish remains unknown. These two proteins arose from a tandem duplication and inversion at an unknown point in evolutionary history, although the two are ubiquitous across reptiles, birds, and mammals and present in at least some non-teleost fish, including the spotted gar. Given the strong, bidirectional selective pressure on both host and pathogen to modulate host PRR activity, these divergences are expected even between closely related species. This diversification is especially notable among CLRs: mice have no fewer than eight putative DC-SIGN homologs and a great deal of work was done to narrow down the functional ones in order to model human disease (Garcia-Vallejo and van Kooyk, 2013); on the other hand, the bovine homolog of Mincle was readily identifiable but had diverged in non-critical domains from the human Mincle (Feinberg et al., 2013; Furukawa et al., 2013).

% In human cell culture and in mice, TDM is thought to be detected first by a low affinity interaction with the constitutively expressed Mcl (macrophage C-type lectin; CLEC4D), which induces an inflammatory response that upregulates its own transcription as well as that of the related protein Mincle, which binds TDM with much greater affinity (macrophage inducible-C-type lectin; CLEC4E) (Ishikawa et al., 2009, 2017; Miyake et al., 2013; Richardson and Williams, 2014; Wilson et al., 2015; Yamasaki et al., 2009; Zhao et al., 2014). These arose from a tandem duplication and inversion at an unknown point in evolutionary history, although the two are ubiquitous across reptiles, birds, and mammals and present in at least some non-teleost fish, including the spotted gar (Amores et al., 2011). Given the strong, bidirectional selective pressure on both host and pathogen to modulate host PRR activity, these divergences are expected even between closely related species. This diversification is especially notable among CLRs: mice have no fewer than eight putative DC-SIGN homologs and a great deal of work was done to narrow down the functional ones in order to model human disease (Garcia-Vallejo and van Kooyk, 2013); on the other hand, the bovine homolog of Mincle was readily identifiable but had diverged in non-critical domains from the human Mincle (Feinberg et al., 2013; Furukawa et al., 2013).

Coinciding with their divergence from the other clades of fish, the teleost fish experienced a whole genome duplication that has left them with redundant copies of many proteins, with substantive consequences for their evolution (Amores et al., 2011; Glasauer and Neuhauss, 2014; Howe et al., 2013). Such a duplication enables diversification of function, increased mutational tolerance, and transcriptional inactivation of one or the other copies with little ability to a priori predict which is expressed (Opazo et al., 2013; Voldoire et al., 2017). As seen with the cow homolog of Mincle (bMincle), this can be compounded by genes that are under strong selective pressure or are intrinsically tolerant to mutations (van den Berg et al., 2012; Zou and Secombes, 2016). Despite these challenges, there is an abundance of evidence that zebrafish possess an as-yet unidentified MINCLE (and/or MCL) homolog including, but not limited to: a long evolutionary history alongside pathogenic mycobacteria, the clear, myd88-independent inflammatory response to purified TDM, and the in vivo attenuation of mutants lacking fully mature TDM. MINCLE has not been previously linked to angiogenesis, but the identification of this receptor in zebrafish would allow us to better understand the relevant pathway in humans; indeed, the role of Mincle during infection is unclear given that some groups have demonstrated that it contributes to bacterial control while others have seen no effect (Behler et al., 2012, 2015; Heitmann et al., 2013; Lee et al., 2012; Matsunaga and Moody, 2009). This has translational implications for modulating the activity of the human MINCLE to enrich for host-beneficial responses and also basic science implications in revealing the diversification of a receptor that maintains the ability to detect a common ligand. 

Although large amino acid segments of CLRs are able to undergo radical changes in primary sequence with few deleterious effects, there are several domains that have been identified as absolutely essential for binding to TDM. A specific set of criteria for this selection are listed in \autoref{minctab}(Alenton et al., 2017; Bird et al., 2018; Feinberg et al., 2013; Furukawa et al., 2013; Zelensky and Gready, 2005). Based on these criteria, we have identified four putative homologs with >50\% amino acid similarity to the human CLEC4E (Figure 3) and have identified transmitting nonsense mutations in each of them (Table 2). As further evidence, two of these homologs (77975 and 79903) are organized in tandem, mirroring the genomic organization of Mincle and Mcl in mammals. These have been crossed into the flk1:eGFP background and larvae will be assayed using both our TDM-injection model and with live mycobacteria. The results of these assays will determine the role of each in detecting and responding to TDM, with the potential to identify functional analogs of both Mincle and Mcl. 

\singlespacing
\begin{center}
\begin{longtable}{|>{\raggedright\arraybackslash}m{1.5in}|>{\raggedright\arraybackslash}m{4in}|}
\caption{Criteria used to select putative zebrafish homologs of the human MINCLE.}\label{minctab} \\

\hline
\thead{Criteria} & \thead{Rationale} \\
\hline
Possesses a gEPNn motif & Of the two major carbohydrate recognition domain motifs, the EPN motif is known to bind glucose-derived sugars while QPD motifs are known to bind galactose-derived sugars. As trehalose is a di-glucose and MINCLE and MCL both possess this EPN motif, this is an important first-pass selection criterion.
\hline
Lacks an intracellular ITAM motif & In humans, both MINCLE and MCL use Fc\textgamma R to signal as they lack their own ITAM motif. While not an essential quality to detect and respond to TDM, this would strengthen the similarities between the two; we have also published data implicating Fc\textgamma R in the zebrafish, which argues in favor of this shared layer of similarity as well.
\hline
Induced by infection & Using existing RNA-seq datasets, expression of these genes under inflammatory stimulus is an important indicator that they may be acting similarly to MINCLE, which is an inducible gene responsive to various inflammatory stimuli. 
\hline
Transmembrane helix & These surface receptors use a single-pass transmembrane helix to remain bound to the plasma membrane and transduce signals.
\hline
Hydrophobic amino acids in the CRD & One of the defining biochemical features of MINCLE is a small hydrophobic pocket that appears to be useful for binding to the mycolate tails of TDM; the presence of such a pocket would be evocative of further similarity to MINCLE.
\hline

\end{longtable}
\end{center}

\doublespacing

Even with functional, biological data, it is critical to validate the binding of the receptor to TDM to ensure that the interaction is direct, especially given the complexity of these pathways. To do this, we will clone the CRD of the putative protein, fuse it to a 3xFLAG tag, and express it in yeast to ensure proper folding. We can then purify the protein as previously described on a sepharose-trehalose column and wash it over PVDF membrane spotted with purified TDM (Feinberg et al., 2013, 2016). We can then detect this interaction using an α-FLAG primary antibody and an HRP-conjugated secondary antibody. With proper controls, this assay will unambiguously establish the direct interaction between the CLR of interest and its proposed ligand. This assay also has utility in exploring the relationship between human and zebrafish Mincle homologs, which is uniquely applicable given the known differences in binding between human and mouse homologs (Hattori et al., 2014).

A humanized model is best able to represent the human response to an immunological stimuli, so I am developing humanized zebrafish that express the cDNA from human Mincle specifically in macrophages (irg1:hMincle) or in neutrophils (lyz:hMincle). These express the full-length human Mincle alongside a tdTomato reporter. These will allow us to investigate the biology and conservation of human Mincle signaling pathways while also taking advantage of the live imaging with chemical and genetic manipulations possible in the zebrafish model. This platform will provide an overexpression and complementation strategy due to the high-degree of conservation in the adaptor proteins. 

\section{Integration of HIF-1\textalpha{} Signaling}

Under homeostatic conditions, the intravital oxygen concentration is maintained at a consistently high level, but under pathological conditions, the concentration of oxygen can fall precipitously. This decline in oxygen concentration requires a cellular response in order to adapt their metabolism and resolve the underlying causes of the hypoxia; these signals are mediated through hypoxia inducible factors, including three isoforms: HIF-1\textalpha, HIF-2\textalpha, and HIF3\textalpha{}. However, these can be induced both by hypoxia and by immunological stimuli. The regulatory network surrounding HIFs requires both transcriptional and posttranslational regulation. The best studied of these, HIF-1\textalpha is constitutively expressed in almost all cells but is rapidly degraded by a stoichiometric amount of the negative regulators PHD and FIH (Figure 10a). These proteins are hydroxylation enzymes that add -OH groups in an O2-dependent manner to defined proline and asparagine residues in order to target HIFs for proteasomal degradation. In the absence of oxygen, these enzymes no longer function and HIF-1\textalpha is stabilized and able to induce transcription (Figure 10b). This is the mechanism for hypoxic activation. Alternatively, an excess of HIF-1\textalpha transcription can enable functional upregulation by introducing a surplus of HIF protein and enabling partial escape prior to degradation (Figure 10c). This is the underlying principle of normoxic activation, which is the most common mechanism for HIF-1\textalpha activation downstream of immunological signals; HIF has been found to be essential for myeloid immune responses (Nishi et al., 2008; Schatz et al., 2016; Schoenen et al., 2014; Thompson et al., 2017). These separate mechanisms allow us to interrogate the underlying basis of HIF activation during mycobacterial infection and the separate contributions of them to the pathology of disease. 

HIF1\textalpha{} is a primary transcription factor for VEGF because angiogenesis is a powerful means of resolving local hypoxia. In the context of the granuloma, both the host and the pathogen must undergo adaptations to this new challenge (Harper et al., 2012; Tsai et al., 2006). The hypoxic nature of the granuloma has been previously documented (Prosser et al., 2017; Rustad et al., 2009), but assessment of the activating signals has previously been challenging and manipulating them even less tractable; as a result, the assumption has been that the effect is driven by low oxygen concentrations. Using the zebrafish, we can overcome some of these hurdles to probe the role of hypoxic signaling in the host response to mycobacterial infection. Additionally, we will be able to differentiate between the distinctive modes of HIF activation in order to better understand these multiple roles within the granuloma and their functional consequences for both the host and the bacteria. 

TRANSCRIPTIONAL AND POSTTRANSLATIONAL REPORTERS

As a preliminary step, a gross reporter of HIF-1\textalpha activation is needed in order to understand the degree to which this response is relevant in the proximity of the bacteria. Previous work in our lab using in situ hybridization for phd3 mRNA revealed the upregulation of this gene surrounding the mycobacteria, indicating hypoxia (Oehlers et al., 2015). The Elks lab generated a transgenic line using a BAC containing the promoter for phd3 that expresses GFP (Santhakumar et al., 2012; Vettori et al., 2017). This line, BAC(phd3:eGFPsh144), will serve as a useful spatial and temporal readout for HIF-1\textalpha activation across different tissues. However, this tool is unable to distinguish between normoxic and hypoxic activation or different cell types, so new tools are required to better address these questions. 

The primary means of oxygen-dependent regulation of HIFs depends on an oxygen degradation domain (ODD) that contains two (hydroxy)proline residues that are hydroxylated by PHD proteins, leading to proteasomal degradation (Figure 13a). This ODD has been shown to be both necessary and sufficient to direct oxygen-dependent degradation, so we will generate a macrophage-specific transgenic line linking an ODD to a fluorescent protein (irg1:ODD-moxCerulean3) (Costantini et al., 2015; Huang et al., 1998). This will provide a cell-autonomous readout of hypoxia because the cells will fluoresce brightly only under hypoxic conditions due to the use of a fluorophore bound to an ODD under the control of a hypoxia-independent promoter.

In parallel, a reporter is needed to provide a readout of normoxic activation of HIF-1\textalpha, which is predominantly thought to be at the transcriptional level. Therefore, I have cloned, de novo, a promoter element from hif1ab based on previously developed cell culture reporters (Walczak-Drzewiecka et al., 2008). This 1.8 kb fragment was cloned into a p5e Gateway vector and recombined to generate two fluorescent reporters: hif1ab:mScarlet and hif1ab:mClover3 (Bajar et al., 2016). As a powerful extension that may give unique insight into the interplay between transcription and proteasomal degradation, I will also generate a hif1ab:ODD-moxCerulean3 line that may lend new insight into the dynamics of these processes in vivo. 

Taken together, these transgenic lines will report (a) the activity of HIF-1\textalpha as a transcription factor, (b) the degree of hypoxia present in macrophages using an oxygen-sensitive fluorescent protein, (c) a HIF-1\textalpha promoter reporter to indicate the transcriptional activity of HIF-1\textalpha, and (d) a HIF-1\textalpha stability reporter combining both the HIF-1\textalpha promoter and an ODD to provide a real-time readout of HIF-1\textalpha transcription and degradation. These distinctions have been previously difficult to assess but these tools will give an unprecedented look at these complex interrelationships. These tools can also be expanded to generate reporters that are more applicable for different contexts (larvae vs. adult granulomas) and used in various combinations together to simultaneously report hypoxia and HIF-1\textalpha transcriptional upregulation (double transgenic irg1:ODD-moxCerulean3; hif1ab:mScarlet).

DOMINANT ISOFORMS

On account of the unique regulation of HIFs, other groups have developed both dominant-active isoforms that are insensitive to oxygen concentration and dominant-negative forms that prevent transcription of HIF target genes. Expressed ubiquitously, both alleles are lethal, but they have been used in cell culture with great success and, more recently, in zebrafish to uncover the roles of HIF in neutrophils and, more recently, macrophages (Elks et al., 2011, 2013; Gerri et al., 2017). I have independently cloned these alleles – the dominant-negative version is simply the first 330 amino acids that allows it to dimerize but not initiate transcription while the dominant-active was produced by performing multiple simultaneous site-directed mutagenesis to replace the two hydroxylated prolines and a hydroxylated asparagine with alanine (Seyfang and Jin, 2004).
Expressed in macrophages, these isoforms will reveal the macrophage-specific roles for HIF signaling and how that impacts infection outcome. To do so, we will generate two new transgenic lines, irg1:DA-hif1ab-p2a-tdTomato and irg1:DN-hif1ab-p2a-tdTomato, similar to those described previously (Gerri et al., 2017). There are many reasons to assume that these experiments will be informative: constitutive activation of HIF-1\textalpha is known to induce angiogenesis, potentially exacerbating the phenotype in the proximity of either TDM or infection while blockade of HIF-1\textalpha is likely to dramatically reduce angiogenesis, revealing a dependence on HIF for the TDM response. Additionally, HIF is known to regulate the transcription of two major antimicrobial proteins: tumor necrosis factor alpha (TNFα) and inducible nitric oxide synthase (iNOS). The effects of HIF in macrophages in inducing these proteins during infection is largely unknown, but these tools will provide new insight into these processes and strongly complement the previously described transgenic reporters.  

ADULT ZEBRAFISH INFECTION MODEL

Low oxygen concentrations are canonical features of both solid tumors and the mycobacterial granuloma. This local hypoxia is an obvious driver of the HIF-dependent response, but the interplay between hypoxic and normoxic mechanisms of HIF activation is challenging to explore and differentiate despite being an important distinction in targeting host processes and associated bacterial responses. However, the tools which we have developed here are uniquely positioned to do just that. As detailed previously, the larval and adult zebrafish models offer distinct advantages: while the larvae are more optically accessible, the adults possess adaptive immunity and form fully epithelialized granulomas. By using them together, we can get a much better idea as to the role of HIF in both early and established granulomas. 

Granuloma establishment takes approximately 14 days in our zebrafish model, approximately coinciding with the presumptive induction of non-sterilizing adaptive immunity, stretching our timeline beyond what is possible in the larval model. Although the adult zebrafish has the advantage of an adaptive immune system the ability to modulate the severity of infection based on initial dose, it is no longer optically transparent and long-term imaging is impractical. Using the previously described MycoGEM, we will be able to differentiate the role of hypoxia and normoxic activation in the granuloma in a model that more closely mirrors human disease (Cronan et al., 2018). Using adult double-transgenic irg1:ODD-moxCerulean3; hif1ab:mScarlet fish infected with either wild-type or ΔpcaA Mm, we can get a simultaneous readout of the degree of hypoxia in different parts of the granuloma alongside the magnitude of hif1ab transcription and how that relates to TDM responses. Additionally, use of the dominant-active and dominant-negative isoforms will provide unprecedented insight into the requirement for HIF in granuloma formation, stabilization, and bacterial control. 

The insights able to be gleaned from the adult are highly complementary to those gathered from the larvae. Indeed, by using similar tools in both contexts, we are able to follow the process of mycobacterial infection longitudinally from the earliest phases of infection response through the mechanisms of host control in the context of the granuloma. Upon completion of this aim, we will have gained previously impossible knowledge relating to the role of HIFs in mediating the response to TDM, controlling bacterial infection, and inducing angiogenesis in both contexts.

\section{Cotranscriptional Interplay}


\section{Lymphangiogenesis}
