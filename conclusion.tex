% The \vspace{} command in this chapter is just for aesthetic reasons - I don't like something new to start at the last line 
%of the page

% ONE OF THE BEST ONLINE LATEX REFERENCES IS AT :
% http://www.eng.cam.ac.uk/help/tpl/textprocessing/latex_advanced/latex_advanced.html

%% ALL figures are in EPS format: It is the best possible format 

%Some ideas:

%\begin{itemize}
%
%\item The Zebrafish Mincle
%\item Sufficiency
%\item Cotranscriptional interplay
%\item Integration of HIF-1$\upalpha$ signaling
%\item Lymphangiogenesis
%\item Aspects that differentiate the isoforms in macrophages
%\item Effect of NFAT mutation on macrophage biology
%\item Mycobacterial interactions with other vasculature-relevant features -- plasmin, TIE2, fibronectin, etc.
%\item Role of NFAT in neutrophils
%\item New tools to study this pathway \textit{in vivo}
%\item Generalizability
%\item Novel Methods to Automate Measurement of Angiogenesis
%\item Novel Methods to Automate Cell Feature Quantitation
%\item Other Contributions of NFAT to Host-Mycobacterial Interactions
%\item Promise as a Host-Directed Therapy
%
%\end{itemize}

At the conclusion of the present work, many new questions have been generated while others remain unanswered. This work has accomplished two primary goals: addressing the intracellular signaling pathway within macrophages that is responsible for inducing angiogenesis during mycobacterial infection (the NFAT pathway) and setting the stage for future work to simplify and automate common procedures commonly used in the analysis of imaging data relevant to both zebrafish and tissue culture research. Some of these lingering questions will be addressed in the coming weeks and months while others will stretch over the course of many years or decades as we delve into deeper and deeper understandings of the fundamental processes governing the nature of the angiogenic response to tuberculosis infection and how and when this can be a fruitful target for therapeutic intervention. 

This leaves a set of important questions, pertinent to model development, deeper understanding of the biology of NFAT within (granuloma) macrophages, the intersections between this pathway and other, known pathways involved in angiogenic responses, and the future of imaging analysis in the context of ever-growing computational power. 

\begin{itemize}
\item What is the TDM receptor in zebrafish and do they have an as-yet unannotated MINCLE homolog? 
\item How or why is NFATC2 special and is it sufficient to induce VEGF? 
\item How does the NFAT pathway alter other aspects of macrophage behavior potentially relevant to tuberculosis biology and does this pathway intersect with HIF-1$\upalpha$ signaling? 
\item Aside from TDM, do mycobacteria have other mechanisms for manipulating the host angiogenic response and, if so, what are they and how does that enhance our overall understanding of this process? 
\item Are our findings on the nature of NFATC2 in inducing tuberculous angiogenesis relevant to other disease contexts where VEGF signaling plays an important role? 
\item Could NFAT offer a meaningful mechanism for inhibiting angiogenesis in the context of disease as a host-directed therapy?
\end{itemize}

These questions, among many others, are the subject of this concluding chapter; it is hoped that a comprehensive presentation of these questions will stimulate future generations to pursue answers and that these will further inform our understanding of the pathogenesis of tuberculosis toward the goal of eradicating this disease.

\section{The Zebrafish MINCLE}

As discussed in \autoref{tdmreceptor}, data from human cell culture and mice has implicated MCL and MINCLE as the primary C-type lectin receptors for TDM, which induces a variety of downstream responses, seemingly including the upregulation of VEGF and downstream angiogenesis (see \autoref{chap3}). However, the precise identity of the homolog of MCL or MINCLE in the zebrafish remains unknown. These two proteins arose from a tandem duplication and inversion at an unknown point in evolutionary history, although the two are ubiquitous across reptiles, birds, and mammals and present in at least some non-teleost fish, including the spotted gar \citep{Miyake2013, Richardson2014}. Given the strong, bidirectional selective pressure on both host and pathogen to modulate host PRR activity, these divergences are expected even between closely related species \citep{Rambaruth2015}. This diversification is especially notable among CLRs: mice have no fewer than eight putative DC-SIGN homologs and a great deal of work was done to narrow down the functional ones in order to model human disease \citep{GarciaVallejo2013}; on the other hand, the bovine homolog of MINCLE was readily identifiable but had diverged in non-critical domains from the human MINCLE \citep{Feinberg2013, Furukawa2013}. These aspects of structural diversity add unique complexities to the identification of any putative functional homolog in the fish, which may have substantially diverged from the ancestral protein as well as the mammalian versions. Despite these challenges, such identification would both substantially advance the zebrafish\textit{M. marinum} model and deepen our understanding of shared mechanisms of detection and response to C-type lectin receptor ligands.

Despite these challenges, there is an abundance of evidence that zebrafish possess an as-yet unidentified MINCLE (and/or MCL) homolog including, but not limited to: a long evolutionary history alongside pathogenic mycobacteria, the clear, \textit{myd88}-independent inflammatory response to purified TDM, and the \textit{in vivo} attenuation of mutants lacking fully mature TDM. Zebrafish have long been speculated to have functional homologs of other C-type lectin receptors despite difficulty in their identification \citep{Petit2019}. MINCLE has not been previously linked to angiogenesis, but the identification of this receptor in zebrafish would allow us to better understand the relevant pathway in humans; indeed, the role of MINCLE during infection is unclear given that some groups have demonstrated that it contributes to bacterial control while others have seen no effect \citep{Behler2012, Behler2015, Heitmann2013, Lee2012}. This has translational implications for modulating the activity of the human MINCLE to enrich for host-beneficial responses and also basic science implications in revealing the diversification of a receptor that maintains the ability to detect a common ligand. 

Unlike MINCLE, the basal receptor MCL has clear roles in mediating protection against mycobacteria \citep{Wilson2015}.

Although large amino acid segments of CLRs are able to undergo radical changes in primary sequence with few deleterious effects, there are several domains that have been identified as absolutely essential for binding to TDM. A specific set of criteria for this selection are listed in \autoref{minctab}(Alenton et al., 2017; Bird et al., 2018; Feinberg et al., 2013; Furukawa et al., 2013; Zelensky and Gready, 2005). Based on these criteria, we have identified three putative homologs with $>$50\% amino acid similarity to the human CLEC4E in the carbohydrate binding domain (\autoref{minctab}) and have identified transmitting nonsense mutations in each of them (\autoref{zfmincs}). As further evidence, two of these homologs (77975 and 79903) are organized in tandem, mirroring the genomic organization of MINCLE and MCL in mammals. 

\singlespacing
\begin{center}
\begin{longtable}{|>{\raggedright\arraybackslash}m{1.5in}|>{\raggedright\arraybackslash}m{4in}|}
\caption{Criteria used to select putative zebrafish homologs of the human MINCLE.}\label{minctab} \tabularnewline

\hline
\thead{Criteria} & \thead{Rationale} \tabularnewline
\hline
Possesses a gEPNn motif & Of the two major carbohydrate recognition domain motifs, the EPN motif is known to bind glucose-derived sugars while QPD motifs are known to bind galactose-derived sugars. As trehalose is a di-glucose and MINCLE and MCL both possess this EPN motif, this is an important first-pass selection criterion. \tabularnewline
\hline
Lacks an intracellular ITAM motif & In humans, both MINCLE and MCL use Fc$\upgamma$R to signal as they lack their own ITAM motif. While not an essential quality to detect and respond to TDM, this would strengthen the similarities between the two; we have also published data implicating Fc$\upgamma$R in the zebrafish, which argues in favor of this shared layer of similarity as well. \tabularnewline
\hline
Induced by infection & Using existing RNA-seq datasets, expression of these genes under inflammatory stimulus is an important indicator that they may be acting similarly to MINCLE, which is an inducible gene responsive to various inflammatory stimuli. \tabularnewline
\hline
Transmembrane helix & These surface receptors use a single-pass transmembrane helix to remain bound to the plasma membrane and transduce signals. \tabularnewline
\hline
Hydrophobic amino acids in the CRD & One of the defining biochemical features of MINCLE is a small hydrophobic pocket that appears to be useful for binding to the mycolate tails of TDM; the presence of such a pocket would be evocative of further similarity to MINCLE. \tabularnewline
\hline

\end{longtable}
\end{center}

\doublespacing

Multiple approaches can be taken to identify the capacity of these proteins to bind TDM and generate meaningful biological responses. Going forward, I propose to take a biochemistry-first approach to this question as it enables greater flexibility in responding to new data and starts with a foundation of known interactions. Thus, I will utilize established methods of TDM blotting \citep{Jegouzo2014} and wash recombinant carbohydrate recognition domain-streptavidin fusion protein lysate across them and then detect these interactions using standard biotin-horseradish peroxidase detection. This will provide a quantifiable readout for both presence/absence of an interaction but also the strength of the interaction. Should none of these proteins efficiently bind TDM, it is relatively trivial to generate new chimeric proteins and test a range of others present in the zebrafish genome. This data can then be used to go back into the zebrafish to assess the \textit{in vivo} consequences of this interaction and also allow for more flexibility in approach -- rather that seeking the receptor, we can explore phenotypes that may be altered in this context across both angiogenic responses and more general immune responses.

Additionally, it may be of some use to study the specific human MINCLE-mycobacteria interactions in the context of a whole immune system. Thus, going forward, it would be logical for future researchers to develop transgenic zebrafish that express human versions of MINCLE and MCL in macrophages and, perhaps, neutrophils, to assess the contributions of the human protein to conserved responses. This may also help to clarify some of the conflicting data in the literature around the role of MINCLE by using an overexpression model to capture the effect of excess MINCLE signaling. In the long term, gene replacement of the native MINCLE-like homolog with the human MINCLE would allow for a more authentically humanized model of macrophage biology within the zebrafish.

\singlespacing

\begin{center}
\begin{table}	
\caption{Putative zebrafish MINCLE homologs with key details about their native structure and mutants that have been generated thus far.}
\label{zfmincs} \tabularnewline
\vspace{0.5cm}
\begin{tabular}{|l|l|l|l|l|l|}
\hline
\thead{Gene ID} & \thead{Length (a.a.)} & \thead{CRD} & \thead{Similarity} & \thead{Mutation} & \thead{Site (a.a.)} \tabularnewline
\hline
56379 & 263 & kEPNn & 50.7\% & ::13 & 43 \tabularnewline
\hline
79903 & 263 & gEPNn & 53.3\% & ::13 & 207 \tabularnewline
\hline
77975 & 170 & gEPNn & 60.3\% & $\upDelta$8 & 108 \tabularnewline
\hline
\end{tabular}
\end{table}
\end{center}

\doublespacing

\section{Integration of Hypoxia Signaling}

The literature is replete with descriptions of HIF-1$\upalpha$ regulation of VEGFA production and signaling; the logical means by which to alleviate local hypoxia is through the recruitment of vasculature carrying oxygenated blood. This allows angiogenesis to occur when necessary and for the vasculature to remain quiescent under homeostatic conditions, where intravital oxygen concentrations are maintained at a high level. However, in areas of pathogen invasion, tumor growth, or tissue damage, the local oxygen concentration can fall, triggering the activation of the HIF-1$\upalpha$ signaling pathway. HIF-1$\upalpha$ is an oxygen-sensing protein that is expressed and rapidly degraded under normoxic conditions but stabilized under hypoxia. Under standard oxygen concentrations, two classes of regulatory proteins mediate prolylhydroxylation and proteosomal degradation of HIF-1$\upalpha$ in a process dependent on molecular oxygen.

Alternatively, HIF-1$\upalpha$ can be induced through alterations in the homeostatic stoichiometry of HIF-1$\upalpha$ itself and the two families of regulatory proteins, PHD and FIH. This allow HIF-1$\upalpha$ to be induced under normoxic conditions, the presumed whole-body state of the zebrafish larva. This normoxia activation has been found to be important for myeloid immune responses, including those downstream of MINCLE activation, and may play an important role especially in the early signaling events of mycobacterial infection \citep{Nishi2008, Schatz2016, Schoenen2014, Thompson2017}. 

In the environment of the granuloma, both the host and pathogen must adapt to reduced oxygen tension. Mycobacteria can temporarily revert into a non-replicating state known as persistence, but this is not a viable strategy for long-term evolutionary success \citep{Ehrt2018, Stewart2003, Manabe2000, Pandey2008, zuBentrup2001}. However, during persistence, the bacteria are extremely difficult to kill as most antitubercular drugs are only effective on replicating bacteria \citep{Veatch2018}. Even during active growth, the bacteria and associated host cells must alter their metabolism to accommodate for reduced oxygen availability, with important consequences for host immunity \citep{Harper2012, Tsai2006, Prosser2017, Rustad2009, Galagan2013}. Despite our superficial knowledge about the importance and contributions of hypoxia in the lifestyle of \textit{M. tuberculosis}, we do not have a clear mechanism to genetically manipulate these responses or to differentiate the roles of hypoxia \textit{per se} from the activity of HIF-1$\upalpha$ signaling. Thus, going forward, new tools are going to be required to study not only the contributions of HIF-1$\upalpha$ in mycobacterial infections, but even more importantly, how those contributions intersect with the role of NFAT in inducing VEGFA production and angiogenesis in this environment.

\subsection{HIF for HIF's Sake}

Several groups, the most notable of which being Philip Elks's lab, have studied the contributions of HIF-1$\upalpha$ signaling to the immune response to mycobacterial infections, but there remain several needs as yet unaddressed. The Elks lab has utilized both dominant-negative and dominant-active versions of the zebrafish \textit{hif1ab} to modulate the activity of this pathway, particularly in neutrophils, and have found that activation of HIF-1$\upalpha$ prolongs inflammation and improve mycobacterial clearance, suggesting that at early time points, HIF-$\upalpha$ plays a protective role in infection \citep{Elks2011, Elks2013}. It was also found that HIF-1$\upalpha$ is important for the induction of TNF-$\upalpha$, which is a critical protective factor during infection \citep{Lewis2019, Flynn1995}. This work nicely complements work from Didier Stainier's lab, which used a combination of \textit{hif1aa} and \textit{hif1ab} mutant zebrafish and new macrophage-specific transgenic tools to manipulate the HIF-1$\upalpha$ signaling pathway and found that this pathway, specifically in macrophages, was critical for mediating developmental angiogenesis. Unlike the behaviors seen in neutrophils, specific expression of even a wild-type \textit{hif1ab} in macrophages was toxic to the cells and rendered them impotent \citep{Gerri2017}. This evokes an important function of this pathway in these cells that current tools remain unable to address. 

To further the study of the HIF pathway in the context of mycobacterial infection, a set of new tools should be made: one is a set of reporter constructs to better identify both hypoxia and transcriptional induction of \textit{hif1ab} in macrophages and the other is a conditional approach to the expression of dominant negative and dominant active versions of HIF-1$\upalpha$ in macrophages, potentially enabling new cell-autonomous understanding of the role of this pathway in mycobacterial pathogenesis and angiogenesis.

Previous work in our lab using in situ hybridization for phd3 mRNA revealed the upregulation of this gene surrounding the mycobacteria, indicating hypoxia \citep{Oehlers2015}. The Elks lab generated a transgenic line using a BAC containing the promoter for phd3 that expresses GFP \citep{Santhakuma2012}, will serve as a useful spatial and temporal readout for HIF-1$\upalpha$ transcription factor activity across different tissues. However, this tool is unable to distinguish between normoxic and hypoxic activation or different cell types, so new tools are required to better address these questions.

During conditions of hypoxia, HIFs are degraded through hydroxylation in the oxygen dependent degradation domain (ODD) that contains two proline residues that are hydroxylated by PHD proteins, leading to proteasomal degradation. This ODD has been shown to be both necessary and sufficient to direct oxygen-dependent degradation, so it seems reasonable to use a macrophage-specific promoter to drive expression of ODD linked to a fluorescent protein as a reporter for granuloma hypoxia. This would be stabilized at low oxygen concentration while being constitutively degraded under normoxia. This would allow for a clearer report of the degree of present hypoxia and complement existing tools, including hypoxyprobe \citep{Cousins2016, Huang1998}.

In parallel, a reporter is needed to provide a readout of normoxic activation of HIF-1$\upalpha$, which is predominantly thought to be regulated at the transcriptional level. Therefore, either ectopic expression or direct protein fusion strategies would be appropriate to the study of this pathway. If some promoter could be identified that responded comparably to the native \textit{hif1ab} promoter or a CRISPR-mediated knockin could be generated, this would be a useful reporter of transcription-level induction of HIF-1$\upalpha$, a process likely relevant to inflammatory responses during infection.  This, in tandem with the previously mentioned ODD transgenes would allow for a more thorough dissection of the relative contributions of hypoxia \textit{per se} and the activity of HIF-1$\upalpha$.

As previous transgenic attempts have failed, the expression of dn-hif1ab and da-hif1ab in macrophages will require new approaches. It seems that misregulation of HIF-1$\upalpha$ results in some sort of developmental toxicity in the macrophage, so it is critical that HIF is only modulated in the time and place where it is most relevant. Thus, using established estradiol-responsive constructs, a set if fish should be made expressing dn-hif1ab and da-hif1ab covalently linked to ER50 degradation domains, which sequester proteins in the cytosol and target them for degradation except when in the presence of tamoxifen. This would allow for the creation of HIF-modulating transgenes within macrophages that would be expected to have reduced toxicity and allow for time-targeted modulation of this pathway. Like the previous tools, this seems especially relevant in the context of mycobacterial granulomas from adult zebrafish, which are known to be hypoxic and more closely resemble human granulomas. These are likely to reveal not only new aspects of HIF modulation of angiogenesis, but broader impacts on the overall response to infection, especially given the central role of HIF-1$\upalpha$ in altering macrophage metabolism.

\subsection{Macrophage Metabolism in Immunity}

Upon activation, macrophages undergo a metabolic switch that corresponds to their longevity and function. While immediate-responding macrophages begin to utilize aerobic glycolysis for metabolism as a way to rapidly (albeit inefficiently) generate energy, longer-term macrophages utilize oxidative phosphorylation to optimize energy consumption over a long course of response. 

\subsection{HIF-NFAT Interactions}

\citep{WalczakDrzewiecka2008}

\section{Cotranscriptional Interplay}

To further dissect the contributions of other transcription factors to the NFAT-dependent angiogenesis response, we can make use of our THP-1 macrophage platform to study the role of NFAT interacting partners in the overall response to \textit{M. tuberculosis} exposure. While wild-type NFATC2 is able to interact with a panoply of other transcription factors through both C- and N-terminal domains, the domains important for particular binary interactions have been teased out over time, largely by Anjana Rao's lab. Thus, to simultaneously test the sufficiency of NFATC2 in inducing transcriptional responses, the importance of AP-1 transcription factor binding, and the ability for NFATC2 itself to bind DNA, I have developed a set of expression plasmids that drive expression of the following:

\singlespacing

\begin{center}
\begin{table}	
\caption{Lentiviral expression constructs to assess the role of NFATC2 domains for induction of VEGFA signaling.}
\label{table:canfat} \tabularnewline
\vspace{0.5cm}
\begin{tabular}{|l|l|}
\hline
\thead{Plasmid} & \thead{Utility} \tabularnewline
\hline
pLEX:mPapaya & Empty expression vector driving expression of only the conjugated fluorescent protein, for background comparison. \tabularnewline
\hline
pLEX:CA-NFAT1 & Expression of a constitutively active NFATC2 that drives transcriptional responses independent of calcineurin. \tabularnewline
\hline
pLEX:CA-NFAT1-$\upDelta$DBD & Expression of a constitutively nuclear NFATC2 that is unable to bind DNA, to assess the contributions of NFAT binding on the induction of VEGFA. \tabularnewline
\hline
pLEX:CA-NFAT1-$\upDelta$RIT & Expression of a constitutively active NFATC2 unable to interact with AP-1 transcription factors, to determine the contribution of this family to the VEGFA response. \tabularnewline
\hline
pLEX:CA-NFAT1-$\upDelta$DBD-$\upDelta$RIT & Expression of a constitutively nuclear DNA binding domain mutant also unable to interact with AP-1 transcription factors. \tabularnewline
\hline
\end{tabular}
\end{table}
\end{center}

\doublespacing



\subsection{NFAT:AP-1 Interactions}

\citep{Macian2001}

\subsection{NFAT and Other Transcription Factors}

\section{New Genetics Tools for the Study of NFAT Signaling}

One of the dominant tools in the field for the study of intracellular calcium flux is the use of GCaMP, a modified green fluorescent protein that fluoresces in response to calcium binding \citep{Nakai2001}. Since its initial development, many interactions have developed allowing for ever-finer detection of various aspects of cellular calcium. These tools have been used in the fish to both detect and manipulate cellular behavior \citep{Beerman2015}. These tools have clear promise in better understanding the biology of NFAT activation, but likely need tethering to either the channels or proteins themselves or a specific cellular compartment to increase spatial resolution; a whole-cell approach is no longer sufficient for the proper understanding of NFAT activity in this context and finer resolution would greatly aid in the identification of future mechanisms.

\section{Dissection of the NFAT Isoforms in Macrophages}

\citep{Shiau2015} % Talk about how to dissect macrophage-dependent vs. macrophage-independent responses.

\section{Mycobacterial Interactions with Other Aspects of Vascular Biology}

\citep{Correa2014}
\citep{ClaessonWelsh2015}
\citep{Eklund2017}
\citep{Oehlers2017}
\citep{Hato2008}
\citep{Sakamoto2010}
\citep{Keskin2015}
\citep{Shin2016}

\section{Lymphangiogenesis}

This project has focused strictly on the proliferation and growth of vascular endothelial cells during mycobacterial infections, a process known as angiogenesis. However, a parallel process exists concerning the less popularly known lymphatic vascular system and this is known as lymphangiogenesis. While this process is less extensively studied, it is a critical component of the vascular development of vertebrate organisms and is essential for proper fluid homeostasis and immune system function, as lymphatic vessels are the route along which antigen presenting cells "drain" into the lymph nodes to prime the adaptive immune system. Failures of lymphangiogenesis result in lymph\oe dema and general problems with fluid balance. In the context of disease, the lymphatic system is essential for proper immune behavior, but during chronic conditions like cancer or tuberculosis, lymphangiogenesis can be subverted by the insult for their own benefit. In cancer, lymphangiogenesis is utilized to provide additional routes of metastasis away from the primary tumor; the role in tuberculosis is comparatively less well studied.

\citep{Alitalo2005}
\citep{Bower2017a}
\citep{Bower2017b}
\citep{Bussmann2010}
\citep{Campuzano2017}
\citep{Dietrich2007}
\citep{Duong2012}
\citep{Hogan2009}
\citep{Wong2017b}
\citep{Makinen2001}
\citep{LeGuen2014}
\citep{Kuchler2006}
\citep{Haiko2008}
\citep{Stacker2014}
\citep{Nicenboim2015}
\citep{Onder2017}
\citep{Okuda2012}
\citep{Han2017}
\citep{Jung2017}
\citep{Harding2015}
\citep{vanLessen2017}
\citep{Shin2017}

\section{Generalizability}

Cryptococcus

\citep{Bojarczuk2016, Lin2006b}

\section{A Pressing Need for More Objective Approaches to Image Analysis}

\citep{Heath2017}

%\item Novel Methods to Automate Measurement of Angiogenesis
%\item Novel Methods to Automate Cell Feature Quantitation

\section{Closing Remarks}
