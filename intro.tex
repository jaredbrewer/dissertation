\section{Tuberculosis}

Of all the infectious agents to have ever afflicted humankind, \textit{Mycobacterium tuberculosis} is perhaps the most imminently successful. The primary cause of potentially greater than one billion human deaths since 1800 alone (approximately 9\% of all deaths in that time period) (citation), this disease has had profound impact on the cultural and political development of the modern world and continues to impact the lives of most people around the world today\footnote{For additional reading on this subject of how tuberculosis has impacted the development of human society, see (citation).}. Fallaciously considered a disease of antiquity, this disease manifests in active disease in greater than 10 million people each year and has killed greater than one million people per year each year since records or estimates have been available with the case and death burden rising due to health system neglect exposed by the COVID-19 pandemic ongoing at the time of this writing (citation). 

\textit{Mycobacterium tuberculosis} has long been of basic scientific interest on account of the myriad ways in which it undermines host immune responses to establish a replicative niche within the human lung. \textit{M. tuberculosis} infection results in the formation of caseating granulomas encased in a complex network of immune cells within which the bacteria replicate. Over evolutionary time, these bacteria have innovated novel ways of subverting host-protective immune responses while exacerbating maladaptive ones. This makes the study of tuberculosis not only the study of microbiology and immunology, but a fascinating study in the basic principles of cell biology. 

\subsection{History of Tuberculosis}

The overwhelming prevalence of tuberculosis in the 18th and 19th centuries led to an extreme degree of cultural salience for this disease in the daily lives of the people of those times. Responsible for the deaths of many preeminent public figures of these eras\footnote{The number of such public figures is far too great to list. From the 1840s and 1850s alone, tuberculosis was responsible for the deaths of Andrew Jackson (seventh president of the United States), Henry Clay (Secretary of State, Speaker of the House, three-time presidential candidate for the Whig Party), John C. Calhoun (Vice President, Secretary of State), Alexis de Tocqueville (famed French observer of American culture and author of the classic of political theory, Democracy in America), Henry David Thoreau (naturalist author of Walden), and Emily Bront\"{e} (author of Wuthering Heights).}, it is also a ubiquitous feature of the literature of those times as well. Perhaps most famously, tuberculosis is depicted as the disease that afflicts the Lowood School in Charlotte Bront��\"{e}'s Jane Eyre, among other novels depicting the disease then known as \textit{��consumption��} for the way in which it leads to cachexia, increasing pallor, hemoptysis, and ultimately death (citation).

This cachexia is a defining feature of tuberculosis across phylogenies; such progressive wasting unable to be ameliorated by improved nutrition is an unusual presentation strongly reminiscent of many cancers and rather dissimilar from most infectious diseases (citation). Indeed, as medical understanding of diseases progressed beyond concepts of humoral imbalance, a prevailing theory was that tuberculosis was a hereditary form of cancer due to the way it spread within families (citation). The functional and consequential similarities between tuberculosis and cancer are replete and will be a subject returned to throughout this document. However, as Louis Pasteur's germ theory came to be more widely accepted in the same period in the Industrial Revolution as Joseph Lister's antiseptic practices, it became a subject of scientific inquiry to identify the potentially infectious bases for seemingly transmissible diseases\footnote{This having the caveat that tuberculosis does not always appear to be transparently transmissible. The disease can take months to years to manifest in infected persons, leading to long time delays between putative exposure and active disease.} 

Robert Koch was the first to document the tubercle bacillus as an infectious agent in 1888. \textit{Mycobacterium tuberculosis} proved to be a foundational instrument in the broader development of the field of microbiology and provided Koch with the rationale for the development of what we know as the Koch's postulates\footnote{The postulates are as follows: the organism should be present in afflicted individuals, but not in healthy individuals; the organism should be able to be grown in pure culture; inoculation of a healthy host should recapitulate the disease; the organism should be able to be harvested back from the new host. Of course, each and every one of these postulates have been broken in pursuit of identifying disease-causing agents, but this remains the basis of identifying new infectious etiologies.}, a procedure for determining the infectious etiology of a disease. This, along with the identification of the anthrax bacillus (\textit{Bacillus anthracis}) sparked the beginning of the modern era of microbiology. It was only once the causative agent of tuberculosis was identified that it became possible to earnestly pursue curative therapies and vaccines, which came swiftly thereafter to varying degrees of efficacy. By 1921, Albert Calmette and Camille Gu\'{e}rin had attenuated \textit{Mycobacterium bovis} for use as a vaccine, creating the BCG vaccine in use today. 

Once thought to have been banished to the annals of history, tuberculosis, after a steady decline in cases throughout the middle of the 20th century\footnote{This was coincident with, but likely unrelated to, the development of effective antibiotic therapies. Indeed, the modern disparity between tuberculosis rates between the United States and Western Europe and much of the rest of the world is thought to have more to do with improved living conditions, growing herd immunity, and improved nutrition rather than the use of antibiotics as downward trends actually began 100 years prior to the discovery of streptomycin in 1944 (citation).}, came roaring back in the late 20th century with the introduction of HIV into the human population in the 1980s and the corresponding increase in susceptibility to infection, disease, and death from tuberculosis due to the immunocompromising nature of HIV/AIDS (citation). 

\subsection{Pathogenic Features of \textit{Mycobacterium tuberculosis}}

In addition to the clear relevance of the study of tuberculosis to human health, the unique biological features of this acid-fast, non-motile, slow-growing mycobacterial species make is a fertile ground for basic scientific studies into the way that both saprophytic and pathogenic species of bacteria adapt to adverse environments and ultimately establish a productive niche (citation). The physiological features of the bacillus -- a thick, hydrophobic cell wall, unique export and import systems, and novel mechanisms for cell division and stress tolerance -- make this a fascinating case study in the evolutionary processes that drive niche adaptation and, indeed, niche creation (citation). That related members of the same genus of bacteria occupy such diverse infectious niches across a wide spectrum of organisms (from fish and amphibians to birds and mammals and in every major organ system), with many also possessing stages of growth in the environment is a testament to the extent to which these species have evolved structures and responses that can accommodate a wide range of physical and chemical stressors. By contrast, some species, notably \textit{M. tuberculosis} and \textit{M. leprae} are tightly adapted to a more limited range of hosts and have lost the capacity for long-term survival outside of a mammalian host. This diversity within the genus offers abundant opportunity for gene-structure-function discovery to uncover factors both required for maintaining an environmental niche as well as those specifically required for either commensal or pathogenic association with hosts, an approach that has long been fruitful in the discovery of novel virulence factors (citations) but comparatively neglected in the basic bacteriological study of environmental mycobacteria.

When a pathogenic \textit{Mycobacterium} infects a na\"{i}ve host, there is a unified set of cellular and signaling events that occur at the interface of the host and the bacterium that facilitate either successful clearance or establishment of a productive infection. Taking human tuberculosis infection as the model, an infected person will cough up aerosolized droplets that contain often an individual bacillus. These individual bacilli can then be inhaled by a na\"{i}ve person nearby, establishing a new cycle of infection. Once that person has inhaled this bacillus, lung-resident macrophages\footnote{Also known as alveolar macrophages, these are one of many, many different types of tissue-resident macrophages. While it is well beyond the scope of this section to explore the distinctions, suffice it to say that macrophages in each tissue niche are functionally distinct from one another and exhibit distinct responses to stimuli. Tissue-resident macrophages are also established from a non-hematopoietic origin from the fetal yolk and are replication-competent, allowing them to self-renew \textit{in situ}. The study of macrophage biology is one of immense challenge and opportunity to uncover novel roles for these multifunctional cells, which are now known to play important roles in processes as diverse as germ cell maturation, metabolism, sleep, cardiovascular disease, and many more.} uptake the bacteria and, in an estimated 90\% of instances, are able to eradicate the infection at the source. However, in the 10\% or so of cases where initial clearance fails, an intricate cascade of events proceeds. After phagocytosis, the macrophage sets in motion signaling processes that should result in fusion between the phagosome and extant lysosomes within the cytosol. However, the bacteria, through a combination of structural features in the cell wall (more on this to come) and secreted effector proteins, blocks phagosomal-lysosomal fusion to establish a productive niche within the macrophage. Subsequently, the outpouring of secreted effectors from the bacteria into the host cytosol results in profound reprogramming that blocks apoptosis, downregulates production of select cytokines and chemokines (while enhancing the expression of others), directs the macrophage to recruit additional macrophages to further the replicative cycle, and directs the macrophage out of the lung proper and into the subpleural space surrounding the lungs, the actual site of primary infection. Due to the slow growth of \textit{M. tuberculosis}, it can take several days before the macrophage has become filled with such a large number of bacteria that it necroses, allowing the bystander macrophages and neutrophils to be infected anew. This cycle continues as more immune cells are recruited by the escalating infection and more extracellular bacteria accumulate. These stages of infection, cell lysis, aggregation, and further recruitment eventually result in the formation of what we know as a granuloma. 

Tuberculous granulomas are complex aggregates of (primarily) macrophages that have differentiated into a less inflammatory, epithelioid state that encapsulate a central focus of extracellular bacteria within a necrotic core. These epitheloid macrophages are augmented by the full spectrum of other immune cell types -- inflammatory macrophages, neutrophils, basophils, eosinophils, natural killer cells, T cells, B cells, and a range of stromal cells. As a discrete structure, these provide a full immunological nexus integrating essentially every identifiable immune cell population. These granulomas, although extractable intact from their environment, must exist in a tissue environment not of their own design. Although immune cells can be readily recruited, the extrapulmonary space is an existing physical location that can be remodeled to some degree but is inherent in the course of the infection. Mycobacteria are tasked with manipulating these tissues, which they do not directly infect, to further their own lifestyle. One of the ways in which they do this, and which is the focus of much of this work, is by inducing the pathological growth of blood vessels toward the site of infection in a process known as angiogenesis.

The classical description of hemoptysis is a clear clinical manifestation of this vascular involvement with the disease. As the granulomas cavitate and release their contents into the airway, the nearby vasculature is damaged. While such damage could have occurred incidentally by damaging the vessels that service the alveoli, it is now clear that much of this hemorrhage is the product of the encapsulating vascular web that is developed around the granuloma over the course of the infection. These clinical observations have, however, left unaddressed the role for this vasculature in the development, progression, transmission, and treatment of the disease. 

\subsection{Treatments for Tuberculosis and their Mechanisms of Action}

Mycobacterial infections are uniquely integrative biological phenomena that require a careful balance between both the host and the bacteria. The host, seeking to eradicate the bacteria, needs a potent but highly specific immune response capable of sterilizing the invading bacilli while the bacteria, seeking to establish a replicative niche, must evade these host defenses. Historically, treatment for bacterial infections has been through the application of bacteria-targeting antibiotics, despite their mechanism of action rarely being understood at the time of clinical introduction. Modern tuberculosis infections are treated with a four-drug cocktail of antibiotics over the course of six to nine months: isoniazid, ethambutol, pyrazinamide, and rifampin (citation). Should the bacteria exhibit resistance to one or more of these, a state known as multi- or extensive-drug resistance (MDR/XDR), additional drugs with further host toxicity are used: kanamycin, ciprofloxacin, and cycloserine are common choices, although new drugs are slowly coming onto the market (citation). Of these, bedaquiline appears to have the most promise in improving the overall treatment of tuberculosis, but long-term impact remains to be seen (citation). 

The first modern, clinically effective treatment for tuberculosis was pioneered by the discovery of streptomycin from \textit{Streptomyces griseus} in 1944 (citation). Unlike its antibiotic predecessor, penicillin, streptomycin was effective in killing \textit{Mycobacterium tuberculosis} bacilli \textit{in vitro}. However, due to its lack of oral bioavailability, the use of this drug was limited to hospitals and clinics able to deliver the drug intravenously. Additionally, like many of the attempts at drug development for tuberculosis that had preceded it\footnote{One of these, para-aminosalicylic acid (PAS) is an interesting, if distracting tale in the history of microbiology. For more information, see (citation)}, it was not particularly effective at eliminating disease when used alone. Streptomycin is an aminoglycoside antibiotic that acts by interfering with protein biosynthesis by poisoning the 30S subunit of the ribosome as well as by inhibiting peptidoglycan biosynthesis through nucleophilic attack of the glycosidic bonds in peptidoglycan (citation). These mechanisms are common to all of the diverse bacteria against which streptomycin is effective, making it a good general purpose antibiotic, if somewhat limited in the face of the unique features of mycobacterial anatomy.

Thus, the introduction of a mycobacteria-specific antibiotic in the form of isoniazid in 1952 was a major breakthrough in the treatment of this disease. Orally bioavailable and highly effective at killing mycobacteria, it comes with the dose limiting side effects of peripheral neuropathy and occasionally fatal hepatitis that make it a less than perfect therapeutic option (citation). It is still in use today on account of its synergy with other antimycobacterials and independent efficacy. Isoniazid works by targeting mycobacterial cell wall synthesis and targets InhA to block earlier stages of fatty acid biosynthesis. This prevents the synthesis of the mycolic acids that comprise the outermost layer of the cell wall and which are essential for mycobacterial survival and growth (citation). 

Recognizing the inherent limitations to isoniazid, additional drugs came into use over the next twenty years. Next on the list of drugs was ethambutol, which entered into use in 1961. Ethambutol, like isoniazid, targets the synthesis of the cell wall, this time by inhibiting the enzymatic ligation of arabinogalactan to the lower peptidoglycan layer and the outer mycolic acid layer, which destabilizes the cell wall and increases bacterial susceptibility to killing. Interestingly, the precise mechanism of action of ethambutol remains unknown despite over 60 years of extensive study (citation).

Rifampin (1965) was the next addition and has an entirely novel mechanism of action compared to the previous entrants. Targeting multiple simultaneous essential biological pathways is an excellent and repeatedly proven way of killing pathogens and preventing the emergence of resistance to all of them simultaneously (citation). Rifampin targets RpoB, the major subunit of the bacterial DNA-dependent RNA polymerase, which is essential for gene transcription. Although mutations have arisen that confer resistance to rifampin, these have particular fitness costs on the bacteria under conditions lacking antibiotic pressure. Rifampin has proven to be an excellent antimycobacterial drug with a comparatively favorable side effect profile compared to the other commonly used drugs.

To round out the four drug cocktail generally recommended for the first-line treatment of tuberculosis today, pyrazinamide (1972) is the most mechanistically interesting of the drugs commonly used to treat tuberculosis. It appears to work by diffusion into the acidic necrotic center of the granuloma where protons activate the prodrug and allow it to be enzymatically converted into pyrazinoic acid, the active antimicrobial. The low pH maintains the stoichiometry in favor of the protonated pyrazinoic acid form over the conjugate base pyrazinoate, facilitating diffusion into the cytosol of the bacteria. Despite the knowledge that has been ascertained about the precise conditions under which pyrazinamide is active, the mechanism of action remains under hot contention with a variety of different mechanisms proposed and the most recent -- that it inhibits the synthesis of the essential fatty acid and metabolic carrier coenzyme A --�� still under dispute (citation). Pyrazinamide, entirely by accident, takes advantage of the particular biological environment of the infecting bacteria to specifically target the pathogen. As a relatively innocuous prodrug that is activated at the precise site of infection, it is able to reduce some of the toxic effects that would be associated with direct use of pyrazinoic acid while concentrating active drug where bacteria are actively growing with passive diffusion moving additional prodrug into the granuloma to be activated and trapped inside the necrotic caseum (citation).

Modern antibiotic development generally has been stymied by a lack of incentive for the development of high research and development cost, low profit drugs. As new antibiotics are likely to be reserved for cases with extensive antibiotic resistance and are likely to be cost-prohibitive, few have been developed despite pressing need. One of the success stories is that of bedaquiline, which was first approved in 2012. The development of bedaquiline required ~\$500 million in public investment compared to ~\$100 million in investment by the profiteering corporation, Janssen Biotech (citation). Bedaquiline is a potent and highly effective drug reserved for use in multidrug resistant (MDR) and extensively drug resistant (XDR) cases of tuberculosis and which acts to block ATP synthase and shuts down bacterial metabolism and directly leads to bacterial death (citation). 

\subsection{The Mycobacterial Cell Wall}

Given that inhibition of cell wall biosynthesis is a common and highly effective mechanism of action for many antimycobacterial drugs, this structure is of clear importance to the survival and pathogenic success of these bacteria. \textit{In vitro}, mycobacteria are unique microbes that grow in intricate serpentine cords along agar plates. These cords were among the first observations that helped to classify diverse mycobacterial species together and defining the ontogeny of these cords was of immense concern to early mycobacteriologists (citation). By the 1950s they had identified what they called the cord factor --�� an isolable molecule required for the cording effect seen in mycobacteria and, indeed, able to replicate key features of cording when isolated, even in the absence of bacteria. The chemical composition of this cord factor was determined and this allowed it to be given a name -- trehalose 6-6'-dimycolate or TDM. TDM features a trehalose head group and two long mycolic acid ester tails that can number up to C\textsubscript{100} in length, creating an incorrigibly hydrophobic molecule that forms an extremely thick amphipathic bilayer at the surface of the mycobacteria with the trehalose moieties facing the outside world and attached to the arabinogalactan layer below with a dimensionally thick\footnote{~40 nm in thickness, representing approximately 30\% of the total volume of the bacteria, if we take the size of a single bacillus as 0.2 \textmu m in depth and 2 \textmu m in length} interior of interleaved mycolic acid chains. TDM is the predominant mycolic acid species in this cell wall layer and has been studied since its discovery for the ways in which it contributes to mycobacterial fitness in a diverse range of environments.  

Mycobacteria do not fit into standard binary classifications of bacteria within the Gram staining system. While Gram-negative bacteria feature both an inner and outer phospholipid membrane, Gram-positive bacteria have only a single plasma membrane but are encased in a thick layer of peptidoglycan. Although evolutionarily derived from the Gram-positive bacterial phylum \textit{Actinomycetota}\footnote{This large phylum of bacteria includes incredible diversity and a number of other important human pathogens with varying degrees of relatedness to \textit{Mycobacterium}. A notable example is \textit{Corynebacterium diphtheriae}, the causative agent of diphtheria, which also produces mycolic acids, albeit shorter in length. The existence of a highly effective vaccination to diphtheria while no effective vaccine exists against tuberculosis is emblematic of the divergent strategies these species use to undermine their hosts \textit{C. diphtheriae} produces a classical toxin, diphtheria toxin, that is responsible for much of the pathology of disease while M. tuberculosis was thought to lack toxins until the discovery of the tuberculosis necrotizing toxin (TNT), although this is only selectively expressed and not thought to be absolutely essential for disease (citation).}, mycobacteria possess features of both Gram-postive and Gram-negative bacteria; they have a single phospholipid bilayer and a thick peptidoglycan layer, but also have an additional membrane�� comprised of mycolic acids which is occasionally referred to as the \textit{mycomembrane} (citation).

The mycomembrane and its primary constituent, TDM, have many well-defined roles in providing tolerance to environmental stress, detoxifying reactive oxygen species, providing dehydration resistance, and modulating host immune responses. TDM, for instance, is able to block a key step in phagosomal maturation, which would normally be able to kill the bacteria after uptake into phagocytic immune cells, including macrophages and neutrophils. The broad ability of TDM to mediate mycobacterial interactions with the environment is one of the critical dimensions of the evolution of mycobacteria and the ability to then utilize novel modifications on this same molecular framework to undermine host immune responses appears to have been essential for their transition to a pathogenic or commensalistic\footnote{This notion of commensal mycobacteria warrants a vast degree of additional study. Although the laboratory model of non-pathogenic mycobacteria, \textit{Mycobacterium smegmatis}, was isolated from syphilitic chancres and, later, smegma, very little is known about the niche of these commensal mycobacteria, how they maintain a neutral or neutral-positive relationship with their (often transient) hosts, and how their presence impacts host immunity to future encounters with pathogenic mycobacteria (citation).} lifestyle in association with eukaryotic hosts ranging from amoeba to fish to humans.

\subsection{Trehalose 6-6'-Dimycolate (TDM)}

TDM, in many ways, defines the lifestyle of mycobacteria. As mentioned previously, this remarkably hydrophobic (indeed, wax-like) structure provides bacteria a potent tool in surviving both harsh environmental conditions but also the conditions likely to be encountered in a host. This structure has been thoroughly dissected over the past decades of research, and a range of modifications are known that influence both the biochemical properties of the cell wall, but also the ways in which host organisms response to this structure. 

Along the length of the mycolic acid tails, there are four main classes of modifications that may be present in two major locations. These modifications include methoxy, methyl, keto, and cyclopropyl groups, which can be located at either proximal or distal locations. Of these, the most research interest has centered on the very unusual cyclopropane modifications, which add a great deal of energetic ring strain to the molecule and is, generally, an unusual biological modification due to its inherent instability and energy investment required to create.

Cyclopropane modification of the proximal modification site has been identified to exist in both \textit{cis} and \textit{trans} isomers, each with distinct immunological properties. The cis modification was described first and is added to TDM by the protein product of the bacterial gene \textit{pcaA}. \textit{M. tuberculosis} deficient in \textit{pcaA} are hypoinflammatory in a mouse model of infection, suggesting that cis-cyclopropane modified TDM is pro-inflammatory. Loss of this gene results in an overall reduction in bacterial survival. This somewhat contradictory result indicates that aspects of the host inflammatory response are important for bacterial survival and replication, findings that have since been replicated in a variety of other contexts in respect to tuberculosis disease. Alternately, trans-cyclopropane modification of TDM is catalyzed by CmaA2 and this orientation was found to be hypoinflammatory. Similar to \textDelta \textit{pcaA} \textit{M. tuberculosis}, loss of \textit{cmaA2} resulted in a bacterial growth defect and prompt clearance of the bacteria, but by an alternative mechanism. Instead of a muted inflammatory response, \textDelta \textit{cmaA2} \textit{M. tuberculosis} induced hyperinflammation. This body of work, largely from the Glickman lab, established a variety of important roles for related but enantiomerically distinct versions of the same biomolecule that differ at only a single chemical site. This specificity is evocative of the high degree to which mycobacterial species have adapted to their hosts by developing novel modifications and mechanisms to perturb the immune response in their favor.

Models of TDM-host cell interactions are often lacking by virtue of the underlying chemistry of TDM. The profound hydrophobicity of TDM limits the avenues by which it can be experimentally presented to cells. On the surface of mycobacteria, TDM is (a) mixed with a range of other co-stimulatory molecules that may be important for the function of TDM, (b) presented along the curved surface of a roughly-cylindrical bacillus, and (c) constantly subject to remodeling as the chemically reactive components are oxidized. \textit{In vitro}, these are difficult aspects to model and two major methods have emerged to agonize cultured cells with TDM: on the surface of polystyrene microbeads and through evaporative monolayers on the surface of tissue culture plastic. Interestingly, these two routes of administration result in profound differences in the overall response from the exposed cells. Surface monolayers of TDM are cytotoxic to cells and trigger a highly inflammatory pyroptotic response; on the other hand, TDM on the surface of beads (although with some variation based on the diameter) tends to drive a more regulated response that still differs in some regards from that induced by whole, metabolically inactive mycobacteria. While whole mycobacteria undoubtedly have other molecular patterns that augment the overall immune response, it is likely that the full breadth of the immune response to TDM has yet to be fully uncovered on account of deficient models to do so. The physiological relevance of these monolayer-like configurations of TDM is up to some debate, but there is some evidence that planes of TDM from dead mycobacteria can form in vivo. 

\subsection{Moonlighting}

Pathogenic microorganisms are often constrained by genomic size -- too small and too few essential functions can be encoded; too large and the risk of duplication errors and cost of maintenance becomes prohibitory. There is therefore a great deal of evolutionary pressure to economize and multitask -- why have two proteins to do two functions if one can do both? That is the precise logic underlying many bacterial toxins, secreted effectors, and structural features. One of the most famous of these multifunctional proteins,�� often dubbed ��moonlighting proteins\footnote{Conceptually, of course, moonlighting is purely orientational. While the given example is one instance where a historically well-defined enzyme has additional functions based on localization, other multifunctional enzymes that can target both bacterial and host substrates or that have distinct functions when cytosolic or periplasmic or secreted are unlikely to be given this title unless they bear high homology to universally conserved proteins.}, is the alpha-enolase from \textit{Streptococcus pneumoniae}. Enolase is an enzyme critical to glycolysis and converts 2-phosphoglycerate to phosphoenolpyruvate, which is essential for the breakdown of glucose into pyruvate. However, \textit{S. pneumoniae} also secretes this normally cytosolic enzyme onto the surface of the outer membrane, which allows it to interact with host plasminogen and catalyze its conversion into active plasmin. Plasmin degrades host fibrin clots, leading to enhanced tissue invasion and pathogenicity through avoidance of host containment by fibrin and increased dissemination. By evolutionary addition of plasminogen-binding properties, fusion of two unrelated proteins into a single protein, alterations of protein localization, or novel layers of regulation, bacteria can, in a very efficient way, exert multiple essential functions from single biological products.

Similar to protein examples, which tend to be more obvious, the structural features of the bacteria can also serve important "moonlighting" functions in the sense that single elements can play key roles in seemingly unrelated phenomena. TDM is an excellent example of this - it is a conserved feature of non-pathogenic mycobacterial species, suggesting that this feature likely emerged to address environmental needs that preceded the need to engage with host immunity. Indeed, TDM serves such a wide array of important functions in the physiology of (especially pathogenic) mycobacteria that to assign it a "major" function would be rather fallacious. As a major structural component of the cell wall, defense from the environment is clearly the overarching theme of this sophisticated glycolipid, but what does that really mean? 

Strictly in the context of host immunity, TDM had been generally ascribed a few major roles: blockade of lysosome-phagosome fusion, alteration in expression of major immunoregulatory cytokines, induction of humoral immunity, and mediation of granuloma formation. Delipidation of mycobacteria results in a profound alteration of the overall inflammatory response \textit{in vitro} and results in efficient bacterial killing by macrophages but perturbed expression of IL-1\textbeta, TNF\textalpha, IL-6, and IL-12. It is now though that many of these functions are mediated by recognition of TDM by surface host receptors, a topic that will be returned to shortly. However, the expression of these critical cytokines (among many others) regulated by TDM results in profound changes in the overall tone and tempo of the inflammatory response that, in aggregate, contribute to granuloma formation, a process we now know to be dependent on both pro- and anti-inflammatory signaling molecules, including IL-4, IL-3, IFN\textgamma, and TNF\textalpha. These processes are intimately linked with the phenotype that will be further explored throughout this work: the TDM-dependent induction of VEGFA and resultant angiogenesis.

\section{C-Type Lectin Signaling}

For over 100 years, the cells of the innate immune system were thought to be essentially blind scavengers that existed solely to pick up debris for presentation to and activation of adaptive immune responses. How could cells saddled strictly with their somatic genotype go about "intelligently" identifying pathogens? The first clues came from gene homology between the cloned human IL-1 receptor and a developmental protein from \textit{Drosophila}, Toll. These similarities between IL-1R and Toll first informed the mechanism by which IL-1\textalpha/\textbeta{} act on cells and then led to the identification of a range of proteins that shared these homologous domains required to activate NF-\textkappa B. These proteins, dubbed the Toll-like receptors (TLRs) proved to be the foundation of the modern understanding of a very sophisticated innate immune system that, in the vast majority of cases, is solely responsible for the prevention of disease. Additionally, these discoveries opened doors into the study of evolutionarily conserved mechanisms of pathogen defense, as TLRs are conserved across both \textit{Animalia} and \textit{Plantae}. 

Another notable multipurpose biological product is the lipopolysaccharide (LPS) of Gram-negative bacteria. LPS is a critical component of the outer leaflet of the outer membrane in Gram-negative bacteria and a central interface with their hosts, for host-associated species. As a result, diverse eukaryotes, including both plants and animals, have developed a family of receptors known as Toll-like receptors (TLRs), one of which -- TLR4 -- induces an inflammatory transcriptional response in many vertebrates. LPS, while often stated as a monolithic entity, is in fact a whole family of diverse lipoglycans that vary widely in saccharide antigen and lipid composition, which has become an active area of study. The precise composition alters the ability for the lipid to bind to TLR4 and induce inflammatory responses. Pathogenic species of Gram-negative bacteria tend to have six (6) lipid tails on LPS that activate TLR4 while commensal or environmental species have five (5) or fewer lipid tails that do not activate TLR4\footnote{For instance, the oral opportunistic pathogen \textit{Porphyromonas gingivalis} produces a tetraacylated LPS that actually inhibits TLR4 activation by hexaacylated LPS from \textit{Escherichia coli} (citation).}. Precisely why and how these differences have emerged and evolutionary rationales for the failure of pathogenic species to adopt immune evading tetra- or penta-acyl LPS is the subject of ongoing work, but it seems undoubted that some aspects of the TLR-dependent response pathway must offer benefit to the bacteria and are an avenue for bacterial subversion of the host immune response.

\subsection{Diversity of Outcomes to Receptor Activation}

All the major families of pattern recognition receptors\footnote{Those being Toll-like receptor (TLR), NOD-like receptor (NLR), RIG-I-like receptor (RLR), and C-type lectin receptor (CLR) families of receptors.} are known to induce the activation of NF-\textkappa B, but many of them have specific additional pathways that they are known to induce that drive a particular kind of immune response that depends on the cell type, the particular receptor activated, the specific ligand, the duration of activation, other physiological variables, and more. For instance, RIG-I-like receptor activation after detection of pathogen-derived nucleic acids drives the nuclear translocation of IRF3 and IRF7 to produce type I interferons (IFN\textalpha/\textbeta), which induces both a paracrine (in neighboring cells) and autocrine (self) response to protect against viruses. 

Additionally, particular ligands can have multiple means of detection based on their particular presentation. The canonical example is lipopolysaccharide (LPS) from Gram-negative bacteria. Extracelluarly, detection can occur through cooperation of CD14 and TLR4, which coordinate the activation of MYD88 and subsequent activation of NF-\textkappa B. Intracellarly, detection is mediated by caspases 4 and 5, which drives inflammasome assembly to process pro-IL-1β and pro-IL-18 into their active, secreted forms, which also drives both paracrine and autocrine signaling cascades to defend against intracellular Gram-negative bacterial pathogens.

TDM, at least compared to LPS, is a relatively understudied molecule as far as the precise mechanisms of detection and response. This has led to there remaining a degree of uncertainty in the field over the contributions of either CLR signaling through MCL and MINCLR or TLR signaling through TLR2 and MARCO to the overall effect of TDM detection on the cellular response. Additionally, there is relatively little known about different physiological presentations and their impact on the response to TDM. In vitro, TDM has been demonstrated to adopt different conformational states based on surface composition and geometry. On beads of a small (exact number) diameter, it adopts a bilayer configuration similar to that seen on live bacilli; on larger beads or on a plane, it acts as a monolayer. The ��monolayer configuration is more inflammatory but was also thought unlikely to exist in vivo. Recent hypotheses have challenged this notion, but what is clear is that TDM must be presented to cells in particular arrangements to have an effect, which is seen in head-to-head comparisons between heat-killed Mtb and gamma-irradiated Mtb. While gamma-irradiated Mtb maintain their shape and structure, heat-killed Mtb are broken down and the presentation of TDM is no longer able to activate CLRs even though it becomes a very potent TLR-mediated vaccine adjuvant. Thus, across different types of bacterial ligands, the context of their presentation to a host is a key determinant of their overall effect on the immune response. This will be a key point in the development of several of our assays in the next chapter. 

TLR4 was the first of these to be characterized and remains the most thoroughly studied. The ostensible ligand (albeit indirectly via CD14) for TLR4 was identified as LPS, a highly conserved pattern common to Gram-negative bacteria. This notion of conserved molecular patterns serving as receptor ligands continues to ignite fields of discovery, which have since identified several major families of pattern recognition receptors (PRRs) that have a diverse set of ligands, commonly known as pathogen associated molecular patterns, or PAMPs. Notably, these receptors do not directly activate NF-\textkappa B, they must signal through a series of adaptor and regulatory proteins, the most notable of which is MYD88, which is required for signaling from all of the TLR proteins except TLR3 and a subset of TLR4 responses. 

Broadly, these can be classified into nucleotide oligomerization domain (NOD)-like receptors (NLRs), retinoic acid-inducible gene I (RIG-I)-like receptors (RLRs), and C-type lectin receptors (CLRs). 

NLRs, which are conserved across metazoans, detect cytosolic patterns, most notably peptidoglycan\footnote{Peptidoglycan is a major component of the bacterial cell wall and is present in essentially all bacteria.} and flagellin\footnote{Flagellin is the monomeric subunit used to generate bacterial flagella in motile strains that use flagella for movement.}. NLRs, like TLRs, activate NF-\textkappa B, but also can activate mitogen-activated protein kinase (MAPK) signaling and induce the pro-IL-1\textbeta{} and pro-IL-18 processing by Caspase-1. 

RLRs are another class of cytosolic sensors and detect aberrant cytosolic nucleic acids, including double-stranded RNA and uncapped single-stranded RNA. These sensors then activate IRF3/7, which produce IFN\textalpha/\textbeta, which activates paracrine and autocrine JAK/STAT signaling.

\subsection{Signaling Mechanisms Downstream of C-Type Lectin Receptors}

While TLR activation \textit{per se} is a rather monotonal response that is predominantly driven by NF-\textkappa B, CLRs terminate in at least two known downstream signaling pathways. In addition to NF-\textkappa B, they are capable of activating the \underline{n}uclear \underline{f}actor of \underline{a}ctivated \underline{T} cells (NF-AT or NFAT) pathway. This ability to activate multiple layers of transcriptional regulation either at the same time or under different contexts (length of time, strength of agonism, particular ligand) offers CLRs a powerful additional mechanism of modulating the tone of the immunological response in response to particular insults. CLRs are known to respond primarily to carbohydrate-linked ligands, as they contain lectin domains able to recognize either glucose- or galactose-derived saccharides. Many biomolecules are sugar-modified, from bacteria, fungi, viruses, and eukaryotes (both self and pathogens). This allows CLRs to be a major pathway for the response to host-derived damage-associated molecular patterns (DAMPs) as well as microbe- or pathogen-associated molecular patterns (MAMPs, or more commonly, PAMPs). 

CLRs are a diverse class of pattern recognition receptors that are defined by their use of divalent calcium (Ca\textsuperscript{2+}) to coordinate the binding of carbohydrate patterns, generally segregated into two major classes: QPD (glutamine-proline-aspartate) motif lectins, which bind galactose-containing sugars, and EPN (glutamate-proline-asparagine) motif lectins, which bind mannose- or glucose-containing sugars. QPD-containing C-type lectins are, in general soluble or secreted proteins and include the likes of human tetranectin (CLEC3B), an extracellualr matrix-interacting protein, and herring antifreeze protein, which mediates the breakdown of ice crystals in the blood of cold-water fish (citations). By contrast, EPN C-type lectins play a diverse set of roles and many are the classical members of the CLR family, with many being transmembrane receptors. Most notable among these EPN-containing CLRs is DECTIN-1, the archetypal member of the family which has long been studies for its roles in antifungal immunity, but has now been discovered to have a diverse set of roles in other conditions, including to bacterial pathogens (including mycobacteria) and in autoimmunity. 

DECTIN-1 has provided the scientific foundation of much of the knowledge we have about the mechanisms of signaling downstream of CLR activation. DECTIN-1 is a single-pass transmembrane receptor that uses a large C-type lectin domain to engage with various ligands, most notably \textbeta-glucans, to stimulate responses in myeloid cells. DECTIN-1 itself possesses an intracellular YxxL/I\textsubscript{x\textsubscript{(6-8)}}YxxL/I motif that is then phosphorylated by an adaptor kinase, spleen tyrosine kinase (SYK). This sets off a complex series of signaling events that activate CARD9, ASC, and/or PLC\textgamma 2, eventually resulting in NF-\textkappa B activation in the former instances and NFAT activation in the latter. For DECTIN-1 specifically, notable roles have been defined for both of these branches in this signaling pathway, but much less is known about these pathways downstream of other, related receptors.

Activation of either TLRs or CLRs can terminate in the activation of NF-\textkappa B, a generally pro-inflammatory transcriptional immune pathway. While TLRs are expressed on a rather wide diversity of cell types, CLRs are often specific to myeloid cells,�� the broad category of innate immune cells that includes macrophages, neutrophils, and dendritic cells. Additionally, the precise outcomes of NF-\textkappa B activation can vary based on the particular cell type, the length of stimulation, and other factors. Interestingly, despite commonalities between these pathways, TLR activation follows a highly proscribed set of signaling cascades that, in varying ways and to varying degrees, are dependent upon NF-\textkappa B. For instance, the primary mode of signal transduction depends on MYD88 and/or TRIF, two adaptor proteins, which ultimately lead to the phosphorylation of inhibitor of nuclear factor kappa B (I-\textkappa B) and subsequent activation of the NF-\textkappa B subunit(s).  Often viewed in a monolithic way by those outside the direct field of study, NF-\textkappa B is comprised of a family of five independent transcription factors (NF-\textkappa B1, NF-\textkappa B2, RelA, RelB, and c-Rel) that generally act as heterodimers with one another; this raises the potential for subtle and as yet undescribed levels of regulatory complexity based on the active heterodimers in different contexts. To date, little has been done to fully characterize these distinctions, although some work has found different transcription factor bindings sites to be preferentially bound by some dimers and not others (citations). However, NF-\textkappa B activation writ large is able to activate transcription of a full spectrum of cytokines and chemokines, including IL-1\textbeta, IL-2, IL-6, IL-12, IL-17, IFN-\textbeta, TNF\textalpha, and IFN-\textgamma. This robust pro-inflammatory response is essential for effective clearance of many pathogens and also increases the resistance of surrounding tissue to further infection by intracellular pathogens.

CLR activation also result in the activation of ASC-dependent canonical inflammasomes, which process pro-IL-1\textbeta{} and pro-IL-18 for secretion and paracrine and autocrine signaling. This process ultimately depends on a common set of adaptors through IL-1R/MYD88, linking it into common sets of signals that are induced by TLR activation. An additional mode is through the activation of the ASC-dependent inflammasome signaling complex, which processes pro-IL-1\textbeta{} and pro-IL-18 for secretion . However, the IL-1 receptor also induces a MYD88-dependent signaling pathway that terminates in NF-\textkappa B. This single pathway thus plays a critical and somewhat circular role in various facets of the host response downstream of TLR activation, which unifies the response tone while potentially limiting response diversity; while TLRs are somewhat broad in their expression pattern, the induction of IL-1\textbeta{} secretion activates all neighboring cells that express IL-1R, which is practically ubiquitous in environment-facing tissues. This makes this pathway extremely powerful for increasing the local inflammatory tone to block the replication and spread of (especially intracellular) bacteria, but subject to a unified set of subversive mechanisms utilized by bacteria, fungi, and viruses (citations). Comparatively less work has been done to characterize the activating mechanisms induced by CLRs that result in inflammasome formation, but this could serve as an exacerbatory mechanism to heighten inflammation in response to particular classes of signal. Notably, LPS (the canonical TLR4 ligand) is also capable of activating inflammasomes, albeit by an entirely distinct Caspase-11-dependent mechanism.

Lastly, CLRs can activate NFAT signaling, which is a signaling pathway common across other developmental processes but is thought to be a unique consequence of CLR activation in the context of PRR activation (and is often used as a reporter for their activity). Despite the relative uniqueness of this pathway to CLRs, comparatively less work has been done to characterize it vis-\`{a}-vis the CARD9-NF-\textkappa B pathway. This pathway can induce the expression of EGR2, EGR3, COX2, IL-2, IL-10, and others, which is a more anti-inflammatory or inflammo-regulatory set of signals than the heavily type I signals induced by NF-\textkappa B (IL-1, IFN-\textgamma, etc.) The interplay between these divergent signaling consequences is poorly understood, but is likely important in determining the overall tone and kinetics of the response to infection.

Two additional members of the EPN-containing superfamily of CLRs are MCL and MINCLE. MCL, originally dubbed DECTIN-3\footnote{And for historical reasons, is still occasionally called this in the modern literature.}, is expressed by myeloid cells at baseline and is a comparatively desensitized receptor with low affinity for its primary known ligand, TDM. MINCLE, on the other hand, is tightly regulated and only induced after cellular priming by some other stimulus, including MCL activation. MINCLE has much higher affinity for TDM and, seemingly, a broader range of agonizing ligands, although the latter discrepancy may be a result of historical scientific focus rather than authentic biological difference.

\subsection{The TDM Receptor}

TDM exerts similarly diverse functions to LPS and is also detected by host pattern recognitions receptors (PRRs), including TLR2 -- another member of the Toll-like receptor family -- and two C-type lectin receptors (CLRs), MINCLE and MCL. As discussed in Section 1.1.4 and 1.1.5, TDM is a structurally essential component of mycobacteria; the absence of TDM renders the bacteria susceptible to immunological, chemical, and environmental stressors. In addition to the important structural aspects of TDM, it also possesses a number of chemical and biological functions in mycobacterial interactions with their hosts.

Chemically, TDM is radically different from nearly any other biomolecule that an organism is likely to encounter. Comprised of a trehalose head group -- an unusual di-glucose that is never synthesized by animals --�� attached to profoundly hydrophobic, extremely long, and diversely modified branched fatty acid tails, TDM is directly cytotoxic to cells through disruption of plasma membrane integrity. For over 50 years after its initial characterization as an important structural feature of mycobacteria, the host receptor, if any, remained unknown. It was only in 2009 that a trio of publications began to define two distinct signaling pathway families that could be activated by TDM - a TLR2-dependent pathway and a Mincle/Fc\textgamma R-dependent pathway. Further studies elaborated the latter pathway to incorporate a biphasic and ultimately heterodimeric model of MCL/MINCLE\footnote{In the literature, these protein products are often listed using mouse-specific nomenclature as Mincle and Mcl for the sake of being more word-legible. MINCLE and Mincle are the protein products of the genes CLEC4E and Clec4e; MCL and Mcl are the protein products of CLEC4D) and Clec4d, from humans and mice respectively.} activation, with MCL being activated first, inducing \textit{MINCLE} expression and then the two acting as a heterodimer over longer courses of signaling.

The TLR2/MARCO/CD14-dependent response axis was first described by David Russell's group in 2009 and established an important role for this complex in regulating part of the expression of TNF\textalpha, IL-6, and IL-1\textbeta. While this was a very thorough and comprehensive project that demonstrated many cell biological aspects of macrophage TDM exposure, it failed to account for the complete response to TDM and left open the possibility that other pathways may play critical roles in mediating this response. Preliminary results had indicated that a Fc\textgamma R-SYK-CARD9 signaling axis was important for innate immune activation by TDM. Knockout of Card9 resulted in nearly abolished expression of a far more expansive range of cytokines and chemokines and led to the hypothesis that some co-receptor was actually responsible for directly binding TDM and that this co-receptor likely lacked its own ITAM motif, as it required Fc\textgamma R to provide one in \textit{trans}. 

MINCLE, itself, was viewed with some interest due to its curious regulatory pattern, but until 2008, no ligands had been identified. The \underline{M}acrophage \underline{in}ducible \underline{C}-type \underline{le}ctin) was originally identified as a receptor for SAP130, a nuclear protein that is exposed to the extracellular milieu after necrotic cell death, which is then able to activate macrophages to scavenge cellular debris (citation). These early observations were, themselves, clues to the pleiotropic nature of Mincle activation as a strictly inflammatory response to cell death would be inappropriate in tone for the majority of innocuous programmed and incidental cell death events that occur almost constantly in the day-to-day lives of organisms comprised of billions of cells. This assigned an initial role to MINCLE, but this broke from the standard pattern of CLRs detecting PAMPs and many of these receptors had already been identified to bind seemingly unrelated ligands.

The intersection of the data on Fc\textgamma R in TDM detection and MINCLE's known dependence on Fc\textgamma R for signaling led to a logical hypothesis: that perhaps MINCLE was the sought-after TDM receptor. A seminal paper in 2009 established a direct ligand-receptor binding interaction for TDM and MINCLE, establishing this as the dominant pathway for TDM detection and host signal transduction. 

Following this, it remained somewhat unclear what pathways were capable of inducing \textit{CLEC4E} expression in response to pure TDM. Over evolutionary time, an ancestral CLR had undergone a tandem duplication to form the modern \textit{CLEC4E} (MINCLE) and \textit{CLEC4D} (MCL). Based on conserved sequence and differential expression patterns, MCL became a compelling candidate for the basal receptor for TDM. Indeed, although it took four years, MCL was identified as a low-affinity TDM binding receptor that could mediate the upregulation of MINCLE, and later, was identified to act in a heterodimer with MINCLE to mediate signaling when both are present.

% \subsection{Discovery and Characterization of C-type Lectin Receptors}

% \subsection{Ligand Presentation and Pattern Recognition Receptor Responses}

\section{Nuclear Factor of Activated T Cells (NFAT)}

NFAT signaling activation results in diverse cell- and context-dependent outcomes. This gene family was first described as a transcription factor that regulated the production of IL-2 in T cells. This pathway could be inhibited by blocking the phosphatase activity of calcineurin, suggesting that NFAT was regulated by calcineurin. While it was initially studied for its diverse roles in regulating lymphoid biology, it has since had widely diverse roles ascribed to it in nearly all cell types including cardiomyocytes, endothelial cells, skeletal muscle, \textbeta{} islets, oligodendrocytes, keratinocytes, and myeloid cells. 

The foundational work on NFAT was done almost entirely in T cells, where it was found to be important not only for intra-T cell differentiation into T\textsubscript{H}2 cells, but also in dendritic cells for the initial production of IL-2, suggesting that NFAT is an inducer of anti-inflammatory signaling cascades that drive type II responses. NFAT is also essential for induction of IL-4/IL-13, the archetypal anti-inflammatory (or inflammation-resolving) cytokines. However, a binary type I/II classification for this pathway is ultimately elusive, as it is also critical for regulating the expression of TNF\textalpha{} and IFN\textgamma, critically important pro-inflammatory cytokines. The pleiotropic nature of this pathway makes it of especial interest in the context of host-pathogen interactions where a robust inflammatory response is needed to kill the invading pathogen, but moderation is required to prevent excessive tissue damage. 

NFAT was discovered relatively early on to be one of the major and defining responses to CLR activation. Defined by Goodridge et al. in 2007 as an important response mechanism, it has been co-opted over the years as an experimental tool to measure CLR activation because TLRs do not activate NFAT (citations). By using either NFAT proteins fused to fluorescent proteins to monitor nuclear localization or the DNA regulatory elements for NFAT to drive luciferase or GFP from a minimal promoter, it is possible to capture a report of NFAT activation with high sensitivity and with rapid response times. This has been used dozens of times in the literature of define the specificity of a response for a particular receptor and ligand. Despite the ironic ubiquity of this approach as experimental tool, very little additional work has been done to define the functional consequences of NFAT activation downstream of CLR activation, especially in the specific context of MINCLE or MCL agonism. Given the specificity of the NFAT response, there must be important biological consequences of this pathway being activated during infection, but these have been broadly neglected. 

One of the major reasons for this neglect has been a unitary focus on the importance of CARD9-BCL10-MALT1 (CBM) signalosomes as another unique consequence of CLR agonism. Despite this method of activation that has more in common with B cell receptor activation than TLRs, the functional downstream consequence is the same: nuclear translocation of NF-\textkappa B and associated induction of immune response genes. Furthermore, the evidence is extremely strong that CBM-dependent signaling is critical for the response to a variety of fungal pathogens and that these generally type I responses are a potent defense against infection. However, numerous datasets have provided evidence of a range of genes that depend on CLR activation but are CARD9-independent. Some of these genes are likely to be NFAT-dependent while others may be activated by as-yet unidentified pathway or through more indirect mechanisms. 

NFAT has many features that make it a transcription factor family of broad basic as well as translational interest. The NFAT family is comprised of five members: NFATC1 (also known as NFAT2), NFATC2 (NFAT1), NFATC3 (NFAT4), NFATC4 (NFAT3), and NFAT5. Historical reasons have resulted in a convoluted nomenclature\footnote{As often happens in science when multiple independent lab groups discover proteins at the same time, the naming can become a challenge as the field as whole reconciles two distinct naming schema. In this case, no resolution has ever come about. The NFATc subnomenclature was meant to designate that they are calcium-responsive and calcineurin-dependent and distinct from NFAT5, the modern homolog of the ancestral protein with high sequence similarity from humans to sponge. In choosing to maintain the NFATc nomenclature, I take no position on the relative merits of the two systems. Additional, now largely outdated, naming schemes had an additional name for each of the isoforms that I will address only as needed throughout this document.}, so for the sake of consistency, the NFATCx naming scheme will be used throughout this document. NFAT5 is a special, and distantly related, member of this family that appears to be important for the transcriptional response to osmotic stress, but unlike all of the other members, is constitutively nuclear and not regulated by changes in cytosolic calcium concentration via calcineurin.

The four calcium responsive members have long been passively assumed to be functionally redundant, with their roles defined by their patterns of tissue expression (citations). All of them are derived from an ancestral single isoform that was duplicated over the course of evolution (although intermediate representatives with greater than one but fewer than four isoforms are unknown among modern species). However, evolution has provided each of these isoforms distinctive biophysical properties that allow them to have non-redundant roles even in cell types where more than one is expressed simultaneously. Most notable is their alterations in sensitivity to changes in calcium: while NFATC2 has a persistent response after strong activation, NFATC3 rapidly traffics in and out of the nucleus in response to small magnitude changes in calcium (citations). 

NFAT requires the phosphatase calcineurin for their activation. Upon an increase in calcium, calcineurin dephosphorylates NFAT to expose a nuclear localization sequence (NLS); once in the nucleus, kinases (including GSK3 proteins and protein kinase A) phosphorylate NFAT to drive it back into the cytosol in inactive form. This shuttling behavior allows existing pools of NFAT to rapidly modulate host responses, including developmental, immunological, and pathological responses. This also allows for rapid tuning of the longevity of the response, presumably allowing for the induction of different genes and to different degrees based on the length of activation. Although no work has ever been done to define such distinctions, the principles of biochemical affinity dictate that more accessible chromatin with more NFAT binding sites would be activated prior to those in less accessible configurations or with fewer sites more distal from the transcriptional start site, which may require long periods of strong activation to be induced. Defining these different classes of genes in different cell types would provide a far greater depth of understanding for the consequences of NFAT activation and timing of intervention for maximum medical benefit.

Recently, and concurrently with the present work, others have identified \textit{Vegfa}\footnote{In the majority of this document, human gene nomenclature is used when referring to pathways in the abstract. However, when relevant to the literature being discussed, the appropriate model organism'��s field-appropriate nomenclature will be used. Later, when work specifically done in zebrafish is discussed, the nomenclature will use zebrafish nomenclature.} as an NFAT-dependent transcriptional target in myeloid cells downstream of Dectin-1 activation through the use of genetic knockouts of \textit{Card9} in mice and in vitro use of NFAT inhibitors after Dectin-1 agonism. This was among the first published works in over a decade to identify a discrete effect downstream of CLR activation that is NFAT-dependent and Card9-independent. Furthermore, there is somewhat of an NFAT renaissance occurring in the literature at the time of writing. Several new papers have emerged in the past several months identifying novel new roles for NFAT signaling in a variety of (predominantly hematopoietic) tissues, giving new emphasis to this long-neglected pathway. The work discussed in later chapters adds to this body of NFAT-dependent responses and, hopefully, encourages additional future work to define the roles of this important but understudied pathway in the response to not only tuberculosis but the full range of human diseases that engage CLR signaling, especially fungal diseases and additional autoimmune disorders.

\subsection{Review of Known Roles for NFAT}

\subsection{Clinical Utility of NFAT Inhibitors}

The imminent importance of NFAT became obvious in the organ transplant era. The recipient immune system will wage immunological war against non-self organs, which requires recipients to undergo life-long immunosuppressive therapy. One of the most successful approaches to preventing organ transplant rejection has been the use of calcineurin inhibitors, including cyclosporine and tacrolimus. These calcineurin inhibitors block the NFAT-mediated transcription of IL-2, IL-4, TNF-\textalpha, and IFN-\textgamma, which dampens the adaptive immune response (especially T cell mediated responses) and dramatically extends the useful lifespan of the transplanted organ. 

Although many immunosuppressive therapies have markedly increased risks for various opportunistic infections, calcineurin inhibitors are comparatively spared from this disadvantage. For instance, alemtuzumab, which depletes B and T cells, increases the risk of a number of bacterial, viral, and fungal infections, including \textit{Staphylococcus}, Hepatitis B, and \textit{Cryptococcus}. By constrast, tacrolimus has potent antifungal activity and no significant relationship between tacrolimus monotherapy and infectious disease risk. This evidence would suggest compensatory mechanisms are available to fend off many pathogens while maintaining enough targeted immunosuppression to prevent organ rejection. This provides evidence of the potential for the use of NFAT inhibitors in an infectious disease context without overtly inhibiting the overall immune response to the infection. 

Tacrolimus especially has found use in other realms of medicine. For atopic dermatitis and psoriasis, it has become a major topical monotherapy to treat these disorders with astonishing results. These inhibitors are site-localized with minimal skin absorption and exhibit far fewer side effects than comparable use of corticosteroids. Approaches focused on targeted inhibition of misregulated pathways have clear promise in improving overall disease outcomes, both in the context of NFAT inhibition, and many other diseases.

New approaches have begun to be developed for NFAT inhibition. One notable example is the development of a comparatively more selective calcineurin-NFAT inhibitor, INCA-6. While tacrolimus acts through binding of FK506-binding proteins, that then complex with and inactivate calcineurin, INCA-6 selectively blocks the interaction between calcineurin and NFAT, sparing some of the other functions of calcineurin while maintaining potent immunosuppression. While this inhibitor has not been exhaustively trialed, this general theme offers some promise for more targeted therapies that can overcome some of the side effects of traditional calcineurin inhibitors while maintaining most of the benefits. 

Another approach that has more recently begun to gain traction is the development of isoform-selective inhibitors. The four NFAT isoforms have selective expression and activation profiles that make them conceivably differentiable biochemically. Select recent work has begun to do exactly that, by taking a structure-guided approach to identifying regions of the proteins unique to particular isoforms and targeting them with small molecules. While no isoform-specific inhibitors are yet available, this will likely change in the coming years.

Beyond small molecule based approaches, the age of personalized medicine opens possibilities for gene therapy approaches to ameliorate pathology caused by one or a combination of NFAT isoforms in particular tissues. For instance, a T cell-targeted NFATC2 mutation may result in comparable graft-sparing immunosuppression to tacrolimus while limiting deleterious consequences to a single cellular compartment. Alternately, new delivery mechanisms may make it possible to selectively deliver traditional small molecules to particular cell types, perhaps through molecular caging approaches or liposomal delivery to discrete tissues. These possibilities and others foreshadow a future of greater specificity in targeting NFAT signaling, on both the isoform and tissue fronts. 

\subsection{Differentiation of Individual Isoforms}

As alluded to previously, the different NFAT isoforms have begun to be assigned discrete functions in different contexts. There is a great deal of interesting biology that derives from the different expression patterns and properties of these proteins, which will be briefly reviewed here.

\subsubsection{NFATC1}
\subsubsection{NFATC2}
\subsubsection{NFATC3}
\subsubsection{NFATC4}

\subsection{New Roles for NFAT}

The central role of NFAT in the immune system has long been appreciated, albeit in a rather limited context, via the widespread application of NFAT inhibitory drugs in the clinic. Two drugs are widely used to block calcineurin activation and suppress immune responses: cyclosporine A and tacrolimus. These drugs were discovered and developed for clinical use in order to target the T cell response and prevent organ transplant rejection by blocking the affinity maturation and proliferation of anti-graft T cells. The profound and global immune suppression that accompanies the use of these drugs has prevented their use in other contexts for fear of increase susceptibility to infectious diseases. The weakness of these drugs is that they block all calcineurin activity in all cell types, leading to a vast range of collateral targets -- a better approach would be to find a way to locally target only the disease-relevant target of calcineurin (in this case, NFAT). Halfway approaches have emerged using tacrolimus (and derivatives) through its use as a topical ointment for atopic dermatitis, but this is inherently limited to skin conditions. What is needed is a generalizable mechanism to deliver potent and localized cellular inhibition of NFAT. Future efforts toward this end may apply adeno-associated virus (AAV) vectors, liposomes, or other delivery mechanisms to drive the expression of VIVIT or CRISPR/Cas9 in specific tissues at particular times.

In the modern era, further roles have been investigated for NFAT that remain somewhat mysterious in mechanism and ontogeny. NFAT activation alters the behavior of platelets and drives inflammatory cascades during Gram-negative sepsis. Mammalian platelets are anucleated, so it is not clear how NFAT is able to modulate cellular behaviors in the absence of its canonical function as a transcriptional activator. The mechanisms of this are certain to be a fruitful avenue of future investigation and are likely to be applicable to nucleated cells as well --�� new tools and deeper understandings of NFAT protein topology will be required to differentiate these classes of functions in these cells. 

\section{Host-Microbe Interactions to Study Cell Biological Processes}

\subsection{Host-Directed Therapies: History and Promise}

One of these defining characteristics is the formation of caseating granulomas. These granulomas, formerly known as tubercles\footnote{Hence, \textit{tubercul}-osis.}, are the most notable and ubiquitous pathology of human tuberculosis. These granulomas are a highly conserved immunological response to any object – pathogen or otherwise – that the immune system is unable to clear and are an imminently visible and clinically definitive manifestation of tuberculosis\footnote{A large body of work exists on the mechanisms that Schistosoma eggs use to induce parasite-beneficial granuloma formation. However, even in the absence of active biological induction of granulomas, sterile but indigestible objects will induce granuloma formation, albeit with some distinguishing characteristics.}. For reasons that remain poorly understood, but likely related to the inflammatory biases of the C57BL/6 and other mouse models, these mice do not form granulomas\footnote{Strangely, these mice do form granulomas in response to Schistosoma and other stimuli, suggesting something distinguishing about mycobacterial infection and perhaps offering clues as to the unique characteristics of the tuberculous granuloma.} after being infected with Mycobacterium tuberculosis and mice do not harbor a strain of Mycobacterium that infects them in the wild. This has set the mouse on an evolutionary trajectory where potentially adaptive – or maladaptive – responses to mycobacterial infection fail to occur. No matter the relative costs or benefits to the host of granuloma formation, the inability of any as yet known mouse model (with the partial exception of the C3H/FeJ model) to form granulomas compromises their ability to serve as a physiologically relevant model of some, but not all, aspects of human tuberculosis.

A major challenge has been the specific identification of diseases, stimuli, and biological consequences that drive angiogenic effects. While the angiogenic response to tumors is thought to be mediated strictly through a hypoxia-dependent mechanism, the angiogenic response to other stimuli are far less homogeneous. For instance, in the context of the tuberculous granuloma, these structures initially form in the oxygenated environment of the human lung, which encounters 21\% oxygen in air approximately 16 times per minute – not an environment that would generally facilitate a hypoxia response. While it is certainly possible in occluded sites to create acute hypoxia, the angiogenic response within the lung would be assumed to rapidly and efficiently alleviate this stressor. No systematic comparison has been done to truly measure the precise oxygen tension in these granulomas from either humans or non-human primates, so it remains difficult to make sweeping assertions. Regardless, the experimental identification of particular mycobacterial components able to induce angiogenesis suggests more sophisticated immunological mechanisms at play than simple hypoxia.

This bacteria-centric approach to treatment of tuberculosis seems logical, as bacteria possess many functions that humans lack entirely that are necessary for their pathogenicity, making these appealing targets for drugs. However, this opens the door to the emergence of resistance when treatment is unable to clear the infecting bacteria and a tolerant or resistant population then expands anew. This makes a compelling niche for a new approach to the treatment of chronic bacterial (and fungal and viral) infections: the host-directed therapy. Host-directed therapies have long been used in cancer. Indeed, anti-angiogenic therapy is one of the earlier examples of a host-directed therapy to cancer. But  translating such therapies to infectious disease has, thus far, proven difficult or impractical. One of the reasons is a lack of understanding of the underlying mechanisms that could be targeted to benefit the host to bacterial detriment; another is the difficulty in interfering with host processes in ways that are specific to the site of infection while minimizing overall toxicity. While host toxicity is generally acceptable collateral damage in cancer treatment, this is often viewed less favorably when treating infectious diseases for which pathogen-targeting therapies are thought superior. Despite these challenges, mycobacterial infections, as a product of the unique intersectionality of host and bacterial biology in the granuloma, offer a spectacular opportunity to develop host-directed therapies that shorten time to cure, abbreviate the current drug regimen, prevent the emergence of antibiotic resistance, and, ultimately, fulfill the World Health Organization’s goal of eradicating tuberculosis by 2050\footnote{Disease eradication has long been a stated goal of many public health campaigns, but has thus far been successful precisely twice: against the scourge of smallpox (in 1977) and against rinderpest (a disease of cattle, in 2011). Current campaigns show promise in the eradication of dracunculiasis (or guinea worm) in the immediate future, with cases down to 14 in 2021. Others, including polio, yaws, and rabies, remain elusive despite all having effective vaccines or treatments, are human-exclusive (or have a known, discrete reservoir), and declining case counts. In the eyes of many, polio is an exceptional disappointment given how close we have come, but the continued need for the use of the oral polio vaccine makes eradication all but impossible in the immediate term.}. 

These conflicting responses are indicative of the importance of other factors in determining the overall inflammatory tone of a particular response to a particular insult, a theme that will emerge throughout this dissertation.

Among the guiding themes of this thesis is that immune responses are never solely one thing or the other. There is growing acceptance that biological responses in general are far more complex than has been generally acknowledged in the literature to date. In the context of mycobacterial infection, the balance of inflammatory and anti-inflammatory responses determines the ability of the host to survive infection. Beyond infection, the balance of signals creates human predisposition to allergies, autoimmunity, cancer, heart disease, and many other disorders. A deeper understanding of the ways that individual signal transduction cascades can drive both type I and type II responses is essential for the development of better therapeutics to treat diseases with underlying ontogenies from either type of response. 

\section{Angiogenesis}

Tissue perturbations, such as those caused by granulomas, often drive the invasion of blood vessels toward the site as a mechanism to facilitate tissue repair. However, these blood vessels can serve as a maladaptive response in many contexts. Most famous is the context of tumor biology, where these vessels serve as a supply of oxygen and glucose, a route of dissemination to distal sites, and a paradoxical barrier to the effective delivery of curative chemotherapeutics. In the transition toward chemotherapeutic options with lessened toxicity, a number of kinase inhibitors and monoclonal antibodies were developed that target a specific receptor on those blood vessels required for their growth and maintenance: the vascular endothelial growth factor receptor 2 or VEGFR2. This tyrosine kinase receptor triggers a downstream transcriptional response cascade that results in endothelial proliferation and directed growth toward the source of the ligand: the vascular endothelial growth factor, or VEGF\footnote{VEGF, or VEGFA, is the "canonical" VEGFR2 ligand, but there are four independent VEGF genes (VEGFA-D), each with different affinity for different VEGFRs (VEGFR1-3) and with different properties. VEGFB modulates VEGFA signaling in angiogenesis, while VEGFC and VEGFD are lymphangiogenic chemokines, which stimulate the growth of lymphatic vessels. This process remains unstudied in the context of mycobacterial infection, but seems likely to play important roles in regulating the course of infection.}. By inhibiting either the enzymatic activity of the receptor using kinase inhibitors or blocking the interaction between the receptor and the ligand using monoclonal antibodies, effective regression of the vascular webs around tumors can be achieved. This therapy has become standard of care for a subset of tumor types and physiological locations, but the mystery remains why this therapeutic strategy targeting a highly conserved (indeed nearly ubiquitous) feature of tumors is not more broadly applicable and generally successful.

The most common of the anti-angiogenic therapies targeting VEGFR2 signaling is bevacizumab. Bevacizumab is a humanized monoclonal antibody that very potently (K\textsubscript{D}=1.1nM) blocks the interaction between VEGFR2 and VEGF and induces vascular regression by binding and sequestering VEGFA. However, the physiological stress that this causes appears to drive a compensatory upregulation of VEGF production by the tumor itself -- the escalating hypoxia in the local region drives rapid amplification of VEGF production to alleviate such detrimental hypoxia. By this mechanism it is proposed that tumors increase the local VEGF concentration beyond the stoichiometry between bevacizumab and VEGF to promote vascular relapse and renewed angiogenesis toward the site (citations). 

Thus, despite the initial promise of anti-angiogenic therapy, the current implementations have several shortcomings that need to be addressed before this can be a viable and widespread strategy to treat solid cancers. However, by analogy, the same challenges exist with using anti-angiogenic therapies to treat other vascularized disorders. Given the central role of the hypoxia response driven by HIF1\textalpha{} to the induction of angiogenesis through the regulation of VEGF, efforts at inducing vascular regression inevitably drive a reduction in local oxygen tension and a corresponding increase in HIF-1\textalpha{} activity and VEGF production. This has logically led to investigation into HIF-1\textalpha-targeting therapeutics, despite the many challenges associated with targeting transcription factors. 

HIF-1\textalpha-directed therapeutic options remain limited in 2022, with none having been approved for broad use and none directly targeting the activity of the protein itself. All existing approaches impact HIF-1\textalpha{} activity indirectly, by altering the activity of upstream regulatory factors or major downstream response pathways. HIF-1\textalpha, aside from its role in inducing angiogenesis, exerts a panoply of pro-tumor effects, including enhancing the epithelial-mesenchymal transition, which enhances the metastatic capacity of the cells (Sharma et al. 2022). The most promising drug candidates are actually those that agonize HIF-1\textalpha\footnote{This is typically by inhibiting the upstream prolyl hydroxylase enzymes that hydroxylate HIF-1\textalpha{} in the oxygen-dependent degradation domain, resulting in its proteosomal degradation. In the absence of effective activity of these oxygen-dependent hydroxylases, HIF-1\textalpha{} is able to remain active.} and drive \textit{increased} local angiogenesis, which is rather beneficial for a number of disorders, including major burns and diabetes (citations). However, existing inhibitors, through either direct or indirect mechanisms, remain either impotent or excessively toxic \textit{in vivo}. However, it has long been established that other transcriptional pathways\footnote{Including SMAD3/4, ERG, p300, and a series of AP-1 transcription factors including c-Fos and Jun-D.} are important for the production of VEGF and these may prove to be a more fertile ground for discovery of anti-angiogenic therapies that target required upstream factors.

\subsection{Developmental Angiogenesis}

Much of what is known about developmental angiogenesis has been established in the zebrafish model, which facilitates live imaging of these processes at high spatial and temporal resolution. Embryos begin to develop a vascular system over the course of development in a process known as vasculogenesis. These initial vascular endothelial cells (VECs) derive from the primordial mesoderm and are then differentiated into endothelial progenitors. These progenitors are responsible for the development of major vessels, but this is insufficient to fully vascularize the developing embryo. By xxx hours post fertilization in the larval zebrafish, developmental angiogenesis begins, which is capable of forming the intersomitic vasculature and elaborating the vascular web in the brain. This develomental angiogenesis strictly requires VEGF-VEGFR2 signaling to occur; in the absence of both of the \textit{VEGFA} homologs in the fish, \textit{vegfaa} and \textit{vegfab}, the somitic vasculature is completely absent and only the dorsal aorta, the caudal vein, and other vasculogenic vessels are able to form. Notably, these phenomena are profoundly difficult to study in the mouse, which must overcome tissue hypoxia as an environmental stressor from very early stages of embryonic development while the externally developing larval zebrafish are perfused with oxygen until many days post fertilization.

Much of what is known about pathological angiogenesis was first established in the context of development as many of the same processes are at play in inducing vascular growth and remodeling. While developmental angiogenesis proceeds according to a tightly scripted set of signaling events that drive vascular normalization and limit vascular permeability, there are a variety of other vascular signals present during pathological events that alter the physiology of these vessels in various ways. 

\subsection{Angiogenesis in Cancer}

One of the foundational observation in solid tumors was that they were often encased in vascular webs that differed substantially from those seen in neighboring, unaffected tissues. The size and cellular density of these tumors created a hypoxic environment at the center of the tumor that drove the activation of HIF-1\textalpha{} and subsequent transcriptional responses geared toward the alleviation of this hypoxia. One of the most efficient means of alleviating cellular hypoxia is to shorten the distance between the cell and arteries, which are able to supply the cell with oxygenated blood. Under homeostatic conditions, no cell is more than 1 mm from the nearest blood vessel. Tumors are thus under profound selective pressure to increase the availability of oxygen or perish. However, the disregulated transcriptional responses of these tumors drive the production of new blood vessels that fail to properly mature and remain highly permeable to blood solutes. This vascular permeability is advantageous to the tumor by allowing glucose and other nutrients to diffuse out of the vessels and into the tumor microenvironment. Such leakiness would logically appear to be a boon to chemotherapeutic delivery, but actually serves as a monumental barrier to achieving sufficiently high drug concentrations at the site of the tumor to read an efective dose. 

These factors and others led to active develompent of vascular targeting therapies throughout the 1990s and early 2000s, which culminated in the approval of bevacizumab to treat colon cancer in 2004. The promise of bevacizumab was to adjunctively enhance the efficacy of other drugs while starving the tumor of needed oxygen -- a theoretical one-two punch at the cell biology of these tumors. However, the story ultimately proved more complicated and bevacizumab has been only a limited success and other vascular-targeting therapies have fared little better.

\subsection{The Relative Failure of Bevacizumab}

Bevacizumab is a humanized monoclonal antibody targeting VEGFR2, blocking receptor activation and resulting angiogenesis. \textit{In vitro} this approach proved very successful and clinical trials demonstrated a favorable safety profile and positive patient outcomes, leading to its FDA approval in XXXX. Initially approved only for (colorectal?) cancer, it began to accumulate other indications for some time, until further trials in other cancers found limited efficacy over the long term. Such a therapy could, in theory, have been a universally beneficial adjunct but proved to be comparatively limited, which is thought to be the product of the biology of the tumor itself. As the antibody begins to induce vascular regression when the vessels are no longer constantly agonized by VEGF, the tumor begins to experience ever-escalating hypoxia and produces more and more VEGF. Eventually, the relative affinity of the antibody and the ligand for the receptor lead to a functional stalemate and a return to \textit{status quo ante}, with the tumor returning to a homeostatic condition unless effectively killed by additional chemotherapies (to which many tumors develop resistance). These sophisticated cellular dynamics are the foundation of the basic scientific interest in tumor biology, but also serve as monumental barriers to effective clinical treatment. 

New approaches to targeting tumor-associated vasculature are clearly needed. As mentioned previously, one approach would be to target VEGF production at the root -- by blocking transcription factors needed to induce VEGF. Additional approaches would be to target axillary pathways also important for VEC biology in tumors -- the interaction between angiopoietins and the TIE receptors is one primary possibility. Other options include somehow mimicking normoxia for the tumor to block HIF-1\textalpha{} activation, if HIF-1\textalpha{} is truly the dominant means of inducing VEGF in this context. Lastly, one of the major hazards of this tumor vasculature is the leakiness, which strongly contributes to tumor growth. By inducing vascular normalization, rather than regression, through a combination of VEGFR- and TIE2-targeting therapies, it may be possible to simultaneously improve chemotherapeutic delivery without inducing compensatory responses in the tumor. 

\subsection{Historical Observations of Angiogenesis in Tuberculosis}

Histopathological characterization of tuberculous granulomas have served as the foundation of our knowledge of the structural features of tuberculosis disease. All of the major features were described on the basis of this histological analysis, including the necrotic core, the epithelioid macrophage layer\footnote{This layer was known as such for decades prior to a proper characterization in (Cronan et al. 2016). These cells pseudo- (or properly) differentiate into macrophages possessing notable epithelial characteristics.}, the lymphocytic cuff, and - important for this work - a surrounding web of vasculature. 

There was rife speculation on the role of this vasculature in the pathology of the disease. Bloody sputum had long been recognized as a primary differentiating factor between tuberculosis and other potential sources of lung disease, suggesting that vascular damage was a major consequence of cavitation and transition to active disease. While some viewed the angiogenesis as circumstantail -- granulomas at sterile wounds surrounding foreign objects also become vascularized -- others speculated that this may be a host-protective effect, allowing for the recruitment of additional immune cells to the site of the infection. There lingered the third possibility that rather than being either incidental or host-protective, that this effect may be a maladaptive response, serving to benefit the bacteria to the detriment of the host. Answers to this question would have to wait for quite some time as the tools to properly dissect the contributions of pathological angiogenesis to tuberculosis disease outcome remained many years in the future.

\subsection{Modern Studies on Granuloma Angiogenesis}

After a decades-long period of relative inactivity, a series of papers in the early-2000s began to explore mechanisms underlying the granuloma angiogenesis, although these early studies lacked effective tools to dissect this biology in the context of infection \textit{sensu stricto}. A pair of early papers investigated the contribution of TDM to the induction of angiogenesis . The chronological first to be published, Sakaguchi et al. did extremely thorough work to dissect the chemokines required for angiogenesis and the cellular source of these factors using an air pouch model in the mouse along with complementary work \textit{in vitro}. This study was interested primarily in the potential for these purified mycolic acids to be used for pro-angiogenic effects, potentially for treatment of burns or diabetes, although little discussion is provided on this front. Saita et al. developed a rat corneal model of angiogenesis and applied purified TDM to induce an effective angiogenic response that depended on VEGF and there was some speculation that TDM may contribute to the neovascularization effect seen in the proximity of tuberculosis granulomas in human patient samples, but little follow-up work was able to be done at that time. These efforts combined a targeted characterization of the cytokine response and piecewise inhibition of these factors to identify both VEGF and IL-8 as important pro-angiogenic factors caused by TDM. With relatively limited tools and the lack of an effective model, no work was able to be done to thoroughly characterize this angiogenesis phenotype in the context of infection or the consequences of blocking angiogenesis on disease progression.

The need for a tractable model required innovative new approaches to studying this phenomenon. Classical mouse models fail to form epitheloid granulomas with associated vasculature and many other mammalian models lack the exhaustive set of tools required for a thorough mechanistic dissection of this process. While a more comprehensive discussion of the zebrafish-\textit{Mycobacterium marinum} model will wait for Section 2.1, this model offered a set of unique advantages -- most notably in regards to genetic tractability and optical transparency -- that enabled pioneering studies on the mechanisms underlying the angiogenic response and the functional consequences of inhibiting this process. This study from Oehlers et al. identified that growth of vasculature toward the site of infection increased bacterial burden and that inhibition of this process may serve as a useful host-directed therapeutic approach. Concurrently, Datta et al. developed a rabbit model of infection, which is a model in which granulomas form along with associated vasculature. This work sought to comprehensively characterize the vasculature and identified notable commonalities between the heterogeneous and structurally abnormal vessels seen in tuberculosis and those seen in solid tumor cancers. The structural deficits of these vessels serve as a barrier to the effective delivery of small molecules because they leak out of the vessels and form a hazy corona around the granuloma without effectively penetrating it. Normalization and regression of these vessels with bevacizumab was able to improve effective small molecule delivery to the inner layers of the granuloma, suggesting that this may be an option for adjunctive therapy to improve drug delivery. Shortly thereafter, Polena et al. interrogated granuloma angiogenesis as a potential mechanism of dissemination and, using Matrigel plugs impregnated with \textit{M. tuberculosis}, identified an angiogenic response to infection in SCID\footnote{Severe combined immunodeficiency mice lack functional lymphocytes.} mice that depended on macrophage-derived VEGF and which facilitated spread from the plug to distal organs. One of the challenges with these latter studies is that they only observe the functional consequences of angiogenesis inhibition for a limited period of time (three or eight days in the Datta et al. study and two weeks in Polena et al.) and Polena et al. utilize a severely immunocompromised mouse model, where granulomas fail to mimic the human pathology and any protective role for the adaptive immune system is neutralized. These studies also raised several important questions:

\begin{itemize}
\item How does VEGF inhibition interact with the use of existing therapies? Can an additive or synergistic effect be achieved by combining the bacterial burden reduction effects seen in Oehlers et al. and Polena et al. with the improved drug delivery identified by Datta et al. in the context of standard-of-care anti-tuberculosis therapy?
\item Does this vascularization effect extend to other granulomatous diseases? For instance, it is known that \textit{Schistosoma} interacts directly with the endothelium, but does the granuloma response itself facilitate growth or transmission via modulation of the surrounding vasculature? Similarly, the fungal pathogens \textit{Cryptococcus} and \textit{Histoplasma} induce granuloma formation in the course of disease and the role for angiogenesis in these contexts remains unknown. Conversely, sarcoidosis granulomas are avascular; what functional distinctions might exist that discriminate between whether or not a granuloma sub-type is heavily vascularized or avascular given that macrophages play the defining role in all cases.
\item Pathogens must actively subvert host immunity in order to survive and transmit and have evolved sophisticated mechanisms for doing so. Given our knowledge that mycobacteria benefit from host angiogenesis, could it be the case that they have developed specific mechanisms for inducing this process? 
\end{itemize}

This last point is notable given the previously discussed foundational findings on the ability of TDM to induce angiogenesis. Could the bacteria be actively modifying TDM in order to enhance the degree of vascularization? Observations from Sakaguchi et al. attempted to dissect some elements of this using TDM from \textit{Rhodococcus} spp., but was unable to functionally differentiate these configurations. Additionally, Polena et al. utilized heat killed \textit{M. tuberculosis} and, while their thesis was that VEGF upregulation was dependent on ESX-1, found a substantial upregulation of VEGF in the absence of active metabolism or structural integrity. These findings suggested a heat-stable factor independent of secreted effectors could mediate at least some percentage of the overall angiogenesis phenotype and that it might depend on TDM.

To this end, Walton et al. genetically and biochemically interrogated the mechanisms underlying the TDM-dependent angiogenesis phenotype. This work identified TDM as sufficient to induce angiogenesis using a novel application of the zebrafish model, wherein TDM emulsified in incomplete Freund's adjuvant is injected into the trunk of the larval zebrafish, mirroring the approach used in Oehlers et al. for live mycobacterial infection. Furthermore, this work identified cis-cyclopropyl modification of TDM as necessary for the induction of this phenotype, a process mediated by the bacterial PcaA enzyme. \textDelta \textit{pcaA} \textit{M. marinum} is attenuated for \textit{in vivo} growth due to deficient angiogenesis, a process that can be complemented in \textit{trans} by co-infection with wild-type \textit{M. marinum}. This work serves as the foundation of the subsequent work presented here as it left a major question as yet unaddressed: \textbf{what signal transduction pathways within responding macrophages are capable of detecting and responding to TDM to drive VEGF production and downstream angiogenesis?}

% A simple way to make sections
% \section{Section}
% Lingva diverseco Homa emancipigxo Cxiu lingvo liberigas, kaj lingva identeco sed ne limigite de ili Ni asertas ke la ekskluziva!\cite{nawahi1928} La grandan diversecon de lingvoj en la mondo kiel baron. Profitus el la scio de dua lingvo Ni estas movado por efika. Etna lingvo estas ligita al difinita perspektivo pri la. 

% Gxi ne estas bazita sur respekto al kaj subteno de cxiuj. Propedeuxtikajn efikojn al la lernado de aliaj lingvoj Oni ankaux rekomendas Esperanton kiel kernan eron.

% An example of making lists of various kinds - This one gives black circles of certain size based on style files - 
% LaTeX manual will tell you how to put numbers or different symobols

%Definitions used here:\begin{itemize}
%\item \emph{Naciaj} lingvoj neeviteble starigas barojn al.
%\item \emph{Starigas} barojn al, cxe granda parto de la monda logxantaro.
%\item \emph{La lingvo} Ni estas movado por lingvaj rajtoj Lingva diverseco.
%\item Ni asertas ke la ekskluziva uzado de naciaj lingvoj \emph{hoarder}.
% \end{itemize}

% Ah! a little bit of math - all math is between two "$" signs

% Thats about it - one more thing - a table can be inserted in the following way - 

%\begin{table}
%\centering
%\begin{tabular}{| c | c | c | c | c |} % Options are c,l,r : centered, left justified, right justified
%\cline{1-5}
% \multicolumn{1}{| c |}{$\phi_c$}&\multicolumn{2}{| c  |}{Before}&\multicolumn{2}{| c |}{After}\\
%\cline{1-5}
% &$Z_{c}$&$\beta$&$Z_{c}$&$\beta$\\
%\cline{1-5}
%0.84058 &$2.390\pm 0.135$ &$0.5166 \pm 0.064$&$1.198 \pm 0.310$&$0.5024 \pm 0.093$\\
%\cline{1-5}
%0.84075&$2.512 \pm 0.138$&$0.5472 \pm 0.073$&$1.071 \pm 0.359$&$0.4601 \pm 0.090$\\
%\cline{1-5}
%0.84172&$2.632 \pm 0.151$&$0.4935 \pm 0.077$&$0.9747 \pm 0.458$&$0.3631 \pm 0.083$\\
%\cline{1-5}
%0.84204&$2.858 \pm 0.127$&$0.5637 \pm 0.086$&$1.183 \pm 0.413$ &$0.3665 \pm 0.079$ \\
%\cline{1-5}
%0.84236&$2.916 \pm 0.133$&$0.5555 \pm 0.093$&$1.744 \pm 0.298$&$0.445 \pm 0.088$\\
%\cline{1-5}
%0.84269&$3.003 \pm 0.124$&$0.5627 \pm 0.095$&$1.989 \pm 0.267$&$0.4691 \pm 0.092$\\
%\cline{1-5}
%0.84301&$3.075 \pm 0.12$&$0.5603 \pm 0.095$&$2.28 \pm 0.235$&$0.5245 \pm 0.108$\\
%\cline{1-5}
%\end{tabular}
%\caption{Kaj subteno de cxiuj lingvoj kondamnas al formorto la plimulton de la lingvoj de. Ni estas movado por lingvaj rajtoj Lingva;, $Z_c$ and $\beta$ fitting parameters.}
%\label{Table1} 
%\end{table}

%figure here please


