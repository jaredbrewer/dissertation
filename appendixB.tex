\newenvironment{QandA}{\begin{enumerate}[label=\bfseries\alph*.]\bfseries}
                      {\end{enumerate}}
\newenvironment{answered}{\par\normalfont}{}

\newenvironment{qanda}{\setlength{\parindent}{0pt}}{\bigskip}
\newcommand{\G}{\bigskip \normalfont}
\newcommand{\Q}{\bigskip\bfseries Q: }
\newcommand{\A}{\par\textbf \normalfont}

\begin{qanda}

\begin{adjustwidth}{0.5in}{0in}
	
\G Summary of the findings

\G Please provide a summary of the study, citing the central question being addressed and the key findings in the manuscript. (If you do not wish to use our Structured Review format, please place all comments to the authors in this box.)

\G Reviewer #1: This study addresses the mechanism used by pathogenic mycobacteria and the mycobacterial lipoglycan trehalose dimycolate (TDM), to stimulate angiogenesis in the region of granulomas. This group has previously reported that induction of angiogenesis is pathogen-beneficial in their model of M. marinum infection in zebrafish, and has now addressed the signaling pathway involved in regulating macrophage production of proangiogenic VEGFA. The key finding in the manuscript is that macrophage NFATc, especially NFATc2a, signaling is essential, while CARD9 signaling is less important, for the effects observed. Certain of the results generated in the zebrafish models are confirmed using a human macrophage-like leukemia cell line exposed to irradiated M. tuberculosis. The major innovative finding is that NFAT signaling can be involved downstream of C-type lectin receptor signaling; a finding that may be generalizable.

\G Reviewer #2: In this manuscript by Brewer et al. the authors examine the link between mycobacterial infection, host macrophages, and induction of host angiogenesis using zebrafish and human cell culture models. The authors cite a previously published report suggesting that mycobacterium-promoted increased vessel growth may facilitate bacterial growth and dissemination. They also note their own previously published study showing that a mycobacterium-derived cell wall component (TDM) promotes vegfaa production and angiogenesis at the site of mycobacterial infection and granuloma formation. Here, they seek to extend on their previous findings on TDM with new results showing that mycobacteria/TDM lead to activation of NFAT signaling, and NFAT-dependent induction of VEGF, in host macrophages. Although these findings are interesting and potentially highlight new targets for disrupting mycobacterial infection, the findings seem somewhat incremental in significance and a number of additional concerns detract from enthusiasm for this study.

\G Significance

\G How do these findings advance the thinking in the field? If there are concerns about conceptual advance (e.g., if the advance is limited by previous work), please provide primary references.
\G Reviewer #1: There are two ways to consider the significance. With regard to promoting the understanding of mycobacterial/TB pathogenesis, the findings might be considered incremental, since the investigators have provided ample evidence of angiogenesis and its significance in this model. On the other hand, the finding of NFAT signaling in C-type lectin induction of VEGFA by macrophages has the potential to be generalizable to other C-type lectin signaling contexts.

\G Reviewer #2: Many of the zebrafish in vivo findings presented in Figs 1-5 are based on image data whose resolution and/or magnification do not always allow clear interpretation or clear visualization of the authors conclusions. 

\end{adjustwidth}

\A We appreciate the commentary provided on the data presentation in Figures 2-4 and have reformatted these figures and selected additional new images that more poignantly demonstrate our observed effects. To enable whole organism visualization, we image these fish at 5x magnification on a Zeiss Z1 AxioObserver epifluorescent microscope; this limits our magnification clarity but allows for the rapid and robust observation of the whole organism effects during our assays. The presented images are maximum intensity projections of the images used for quantitation and this lossy projection strategy may obscure some effects that would be otherwise visible in individual stacks from the source images. For transparency, all of the raw images used to generate all of the data in this manuscript are included in our forthcoming Zenodo data release.

\begin{adjustwidth}{0.5in}{0in}

\G Despite statistical significance in the authors measurements, the vessel effects appear modest, particularly in the larvae, and it is sometimes difficult to see how they were quantitated from the image examples presented, or how these sorts of limited vessel sprouts and segments, which are unlikely to be carrying much if any additional flow from the authors images, could be significantly influencing mycobacterial growth or dissemination. 

\end{adjustwidth}

\A Our larval system demonstrates substantial heterogeneity between individual fish, which can obscure some subtle effects. However, our findings in this regard are both highly reproducible and of reasonable effect size. We apologize that our image selection in the initial submission obscured this effect.

\A It has been a long standing question about what aspects of the angiogenic effect are best modeled in the larva and how those translate to more established infections, where hypoxia is likely a major physiological factor. Thus, for the first time, we have included data from adult granulomas, which histopathologically mirror human tuberculosis granulomas. We hypothesize that, while the angiogenic sprouts themselves in the larvae may or may not be directly contributing to infection progression, they may be an indirect reflection other aspects of vascular biology (vascular permeability and remodeling) that is readily quantifiable and a proxy for the effect seen in the adult and that has been long noted in other model systems.

\begin{adjustwidth}{0.5in}{0in}

\G Since these measurements appear to rely on visual inspection and measurement of vessel/sprout length and are potentially more subjective than quantitative molecular measurement methods, it is also important that they are done fully blinded and that the blinding methods are fully described in the methods section. 

\end{adjustwidth}

\A Additional text has been added to the STAR Methods to reflect the blinding performed for the experiments and we apologize for any issues with clarity in the existing manuscript. We fully agree that the subjective nature of these types of measurements require blinding to be robust and reproducible. 
\A The blinding for these assays were conducted as follows: for assays where experimental blinding was difficult or infeasible (drug treatment assays, transgenic irg1 fish, adult infections), computational blinding was conducted using either the blindrename.pl script (Salter, 2016) or an in-house developed Python translation of the same (included in the Zenodo submission accompanying this manuscript). This script renames all the files to a random string of characters and generates a keyfile to allow for post hoc matching of blinded images to treatment groups or genotypes. For all assays necessitating post hoc genotyping (larval infection of host mutants), genotypic blinding was used. Measurements were collected on each of the images prior to matching them to the determined genotype. 

\begin{adjustwidth}{0.5in}{0in}

\G Besides vessel length and a few measurements of mycobacterial cfu’s in figure 5, no other quantitative data are provided for the zebrafish portion of this study (eg, rigorous molecular measurement of NFAT activation or VEGF induction in host macrophages), despite the technical feasibility of such measurements, using the transgenic lines available to the authors. 

\end{adjustwidth}

\A These are excellent ideas and ones that we have pursued throughout this course of experimentation. Despite repeated attempts at generating some form of transgenic reporter for NFAT activation, we repeatedly failed to do so; there were apparent toxic effects from whole-organism reporters and difficulty in identifying founders from macrophage-specific reporters. This is an area of active study and we hope to have such a tool available for future studies.
\A Measurements of VEGF induction are technically possible with the available tools (including the TgBAC(vegfaa:eGFPpd260) line) but we have generally found quantitation with this approach challenging. The high background fluorescence in the larvae narrows the dynamic range for detection while signal in the adult is extremely low (potentially due to insertional effects), an effect seen in Karra et al. 2018 as well. 

\begin{adjustwidth}{0.5in}{0in}

\G Furthermore, if the authors wish to confirm the requirement for macrophages in induction of host angiogenesis in zebrafish larvae in vivo, they could directly examine this by genetically ablating these cells, which can readily be accomplished in fish. The authors could also perform macrophage-specific targeting of nfatc2a and nfatc3a using irg:cas9 and guide RNAs targeting each of these genes, to confirm the specific requirement for nfatc2 in these cells in vivo.

\end{adjustwidth}

\A This excellent experimental idea has been well taken and is among our ongoing interested in the lab; however, this experiment is well beyond the scope of the revision process. 

\begin{adjustwidth}{0.5in}{0in}

\G There are some related concerns about the in vitro human macrophage quantitative data in Figures 6 and 7. The different replicates of the VEGFA qPCR and ELISA results shown in fig. 6 and supplemental Fig. 4 have strongly differing results, suggesting there may be other variables that are not being adequately controlled in these assays that may be having substantially larger effects then the MTB and INCA-6 effects, and making selection of a "�representative" result somewhat unclear. Is macrophage number equivalent in all these experiments? 

\end{adjustwidth}

We agree with the reviewer on the heterogeneity seen in the results from experiment to experiment. While the pattern and trends are identical across the replicates, the magnitude of difference varies. There are likely to be additional variables interfering in this assay beyond our experimental control. While we seed identical numbers of macrophages in each experiment and treat with the same amount of γMtb, we (and others) have observed substantial heterogeneity and senescence with THP-1 macrophages that make them a challenging model to work with (Stokes & Doxsee 1999; Spangenberg. et al. 2021). An observation that emerged in the course of our work is that the amplitude of the VEGF response to γMtb diminishes with increasing passage number and basal expression of VEGF increases. These effects make it challenging to generate true biological replicates, which is why we include 4 wells for each biological condition in each assay and perform technical duplicates on qRT-PCR and ELISA to account for the robustness of the effect within each assay. 

\begin{adjustwidth}{0.5in}{0in}

\G How are the values normalized? 

\end{adjustwidth}

\A For qRT-PCR, the values are normalized by 2-ΔΔCt and referenced to the amount of GAPDH transcript present. The ELISA assays are more challenging to normalize precisely, but we seed identical numbers of macrophages per well and collect the same volume of supernatant from each well for subsequent processing. The values are converted using a standard curve generated from included reference standards in the commercial assay kit. 

\begin{adjustwidth}{0.5in}{0in}

\G Despite activation of NFAT being a key point in this manuscript, the authors do not perform rigorous quantitative measurement of NFAT activation by careful measurement of nuclear/cytoplasmic compartmentalization or other methods (this should be done in conjunction with simultaneous quantitative measurement of VEGFA induction). 

\end{adjustwidth}

\A The reason for this omission from the initial manuscript was due to an attempt to conduct precisely this analysis in a computationally robust manner, but we were unable to find or develop a tool able to conduct this analysis properly. We therefore computationally blinded and manually counted the same subset of images used for previous Figure 6F to determine whether or not (a) γMtb treatment increased the degree of NFAT nuclear localization in these samples and (b) whether this increase in nuclear localization corresponded to an increase in VEGF+ cells. New Figure 6G shows a statistically significant increase in the percentage of cells in a given field of view demonstrating NFAT nuclear localization while Figure 6H shows an increase in the overall percentage of cells (normalized to the total number of cells visible) that have nuclear NFAT and VEGF production. We think this reanalysis offers convincing evidence that γMtb is able to induce NFAT activation and corresponding VEGF production. This critique from the reviewer has substantially improved Figure 6 and the overall narrative of the paper.

\begin{adjustwidth}{0.5in}{0in}

\G The data in Fig. 7 are notably deficient in this regard. 

\end{adjustwidth}

\A The results presented in Figure 7, being from pooled but puromycin-selected transduced cells, are difficult to quantitate in the same manner as Figure 6 due to the substantial heterogeneity present among the cells. We performed qPCR on these samples and see and average ~90\% reduction in NFATC2 mRNA expression across 12 different, randomly selected clones and ~75\% reduction  in mRNA on the entire bulk pool used in these experiments as a whole using undifferentiated, suspension THP-1 cells, suggesting that our CRISPR approach is able to specifically reduce expression of our target gene at the mRNA level.
\A The same concerns about senescence and passage number have applied to these assays as the ones described previously and in Figure 6. However, we sought to develop single clones from these pools of transduced cells. ….

\begin{adjustwidth}{0.5in}{0in}

\G The VIVIT experiments need validation of the efficacy and specificity of its effects on NFATC activation. 

\end{adjustwidth}

\A A large body of existing literature has established VIVIT as a potent and highly specific peptide inhibitor of NFAT-calcineurin interactions. We sympathize with the desire to experimentally demonstrate the potency and specificity of this inhibition strategy, but are unfortunately unable to do so in the allotted time frame.

\begin{adjustwidth}{0.5in}{0in}

\G And if the authors want assess the role of NFATC2 in their in vitro model, it would perhaps be better to select germline CRISPR knockouts of this gene in their cells, and perhaps perform siRNA knockdown to confirm.

\end{adjustwidth}

\A See response above.

\begin{adjustwidth}{0.5in}{0in}

\G Major concerns and limitations
\G List all concerns with the experimental and/or analytical approach(es) and data in the study, including the statistical analyses. Flag the 3 points (with an asterisk or another symbol) that you consider of central importance. If conceptual advance is a main concern, this can be indicated here as well as in above in the Significance questions.
\G Reviewer #1: There are no major concerns with the experimental, analytical, or data, in the study. *One concern is whether the finding that NFAT signaling downstream of C-type lectin engagement in macrophages contributes to angiogenesis is unique to this system, or if it is generalizable. More information on the possibility that the observations relate to signaling that is only engaged by a high concentration of agonist (ie numerous extracellular bacteria) could shed light on this possibility. *In isolation, the findings reported here do not solve a major question in mycobacterial-host interactions, but may provide a step toward addressing the larger question of how angiogenesis is pathogen-beneficial in mycobacterial infections.

\end{adjustwidth}

\A We, too, are intrigued by the possibility that this mechanism may be generalizable to other disease contexts, including other pathogens and in autoimmune conditions. While this is beyond the scope of the current publication, it is our hope that the present findings will provide impetus and a foundation for further study on the role of NFAT signaling in mediating pro-angiogenic macrophage responses. 

\begin{adjustwidth}{0.5in}{0in}

\G Reviewer #2: See "significance"
 
\G Minor points and recommendations
\G Please list any additional comments and/or suggestions.
\G Reviewer #1: 

\G The observation that induction of angiogenesis is delayed for several days post infection, when extracellular mycobacteria have accumulated, may be interesting and provide additional clues to the biological role of angiogenesis and mycobacterial infection. The observation suggests that the Clec-NFAT pathway might be due to low sensitivity and only activated by a high concentration of Clec ligand. It would be beneficial to address this possibility and consider its impact on mycobacteria-host interactions. The Discussion addresses the observation, but not its potential unique significance.

\end{adjustwidth}

\A Thank you for this excellent point of discussion. While this is a topic we attempted to broach with Figure 1, we have now expanded on this point in the Discussion.

\begin{adjustwidth}{0.5in}{0in}

\G The manuscript notes the involvement of calcium signaling in NFAT activation in classical systems, but doesn’t address it experimentally. It would be interesting to know if intracellular calcium responses to mycobacteria and/or TDM through C-type lectins require high ligand/agonist concentrations, or if the noncanonical role of NFAT is calcium-independent. Inhibition by calcineurin blockade provides indirect evidence against the latter possibility, but doesn’t prove it.

\end{adjustwidth}

\A The precise mechanism that activates NFAT is an interesting point that dives deeper into the cell biology of this process. Whether or not this is by traditional calcium influx or some alternative mechanism is intriguing and something worth exploring in the future.

\begin{adjustwidth}{0.5in}{0in}

\G As presented, it is difficult to relate the images and the quantitation of angiogenesis. The images shown are low power, and it seems unlikely that such low power images were used to generate the quantitative data used to interpret the results.

\end{adjustwidth}

\A We have significantly improved the data presentation in figures 2-4 in response to these critiques. We agree that the existing presentation was somewhat unclear and difficult for the general reader to discern. The images shown are derived from the actual images used to visualize these events; however, the process of generating maximum intensity projections for presentation in a publication necessarily loses some detail; the actual quantitation is performed stack-by-stack at high digital magnification. To best visualize the entire organism, we capture all of these larval zebrafish images at 5x magnification on a Zeiss Z1 AxioObserver with an AxioCam MRm at 1388x1040 pixel resolution. All of these images have been included in the associated Zenodo release in raw form.

\begin{adjustwidth}{0.5in}{0in}

\G Very minor point: the legend for Fig3A is, “NFAT CRISPR Screen”, but results of only two NFATs are shown.

\end{adjustwidth}

\A We apologize for any miscommunication about the nature of this experiment. The so-called screen was relatively limited in scope (as we had specifically selected these two isoforms for comparative study) and we have removed the language referring to this as a “screen.”

\begin{adjustwidth}{0.5in}{0in}

\G Reviewer #2: See "significance"
 
\G Clarity of reporting
\G Please comment on whether the paper adequately reports methods, number of samples and independent experiments, statistics, data, and if applicable, code.
\G Reviewer #1: Overall, the manuscript is very well written and organized. The methods, sample numbers, and analyses are all appropriate.
\G As noted above, the link between images demonstrating angiogenesis and the quantitative data could be improved.
\G The videos are a minor addition without additional labeling on the videos themselves, to highlight the points made.
\G Reviewer #2: Mostly yes, but more information on blinding methods for quantitattive measurements (most of which appear to involve visual scoring of image data) in this manuscript would be helpful.

\end{adjustwidth}

\A As referenced above, we have substantially improved the description of the blinding approaches used through the manuscript. 

\end{qanda}