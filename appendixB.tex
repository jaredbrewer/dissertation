The following is the cover letter that was provided along with the resubmission to \textit{Cell Reports} for \citet{Brewer2022}. Written together with Dr. David Tobin. 

\begin{flushright}
5 November 2022
\end{flushright}

Dear Editors,

At the advice of Faby Rivas, Deputy Editor of \textit{Immunity}, we are transferring our manuscript “Macrophage NFATC2 Drives [\textit{sic}] Angiogenic Signaling During Mycobacterial Infection” to Cell Reports. Dr. Rivas wrote with our initial \textit{Immunity} decision: ``However, given a recent change in editorial policy that enables me to speak for \textit{Cell Reports}, I can say that \textit{Cell Reports} would be interested in taking your study to publication upon clarification of the quantification concerns raised by referees (minor revisions).''

As detailed in the Response to Reviewers document, we have both clarified and expanded on the quantification that we had previously performed. We have made clear that our initial analyses were all performed blinded as well as performing additional extensive blinded and automated quantification as new Figure panels. These clarification as well as additional analyses have made our conclusions stronger. In addition, we performed one additional set of experiments to address variability in effect size that had concerned one of the reviewers. Our Response to Reviewers document describes this in more detail.

We think that this an exciting story that would be of broad interest to \textit{Cell Reports}' readers, as it touches on new or emerging concepts in mycobacterial pathogenesis, macrophage biology, engagement of endothelial cells during host immunity, immune cell/(non)\hyp{}immune cell crosstalk, and NFAT signaling downstream of C\hyp{}type lectin recognition of a specific bacterial lipid (a non\hyp{}canonical pathway).

We are excited about multiple aspects of this work, which defines a completely unknown role for macrophage\hyp{}specific NFAT signaling in mycobacterial pathogenesis, both in the zebrafish model and upon stimulation of a human cell line with M. tuberculosis. We had previously identified the impact of mycobacterium\hyp{}induced angiogenesis in infection \citep{Oehlers2015} as well as identifying an important bacterial modification of the mycobacterial cell envelope lipid trehalose dimycolate (TDM) that we showed is required for mycobacterium\hyp{}induced angiogenesis \citep{Walton2018}. In the new work, we establish \textit{in vivo} how TDM generates this host response, first ruling out \textit{in vivo} the key canonical signaling molecules like CARD9. Through intravital imaging and targeted, cell\hyp{}specific peptide\hyp{}mediated inhibition, we identify macrophage\hyp{}specific activation of NFAT signaling as essential to granuloma\hyp{}associated angiogenesis. NFAT signaling has been extensively studied in the context of T cell biology, but has not, to our knowledge, been implicated in the macrophage response to mycobacterial infection.

We then combined single cell transcriptional data that we had generated from zebrafish granulomas and genetic approaches to identify the specific NFAT isoform required \textit{in vivo}. In an \textit{in vivo} genetic screen of multiple isoforms via CRISPR/Cas9 targeting, we found that \textit{nfatc2a} in zebrafish was required for the response, while other NFATc isoforms were not. Finally, we extended our zebrafish \textit{in vivo} findings to human cell lines and show that human \textit{NFATC2} (the same isoform we identified in the zebrafish studies) mediates this response via macrophage production of the key pro\hyp{}angiogenic molecule VEGFA downstream of M. tuberculosis TDM activation. 

This conservation from fish to humans is particularly exciting and our results are consistent with single\hyp{}cell analysis by ourselves and others of TB granulomas in zebrafish, mammalian models, and humans. Indeed one prominent theme of the more recent single cell profiling (scRNA\hyp{}seq of zebrafish granulomas and non\hyp{}human primate granulomas and MIBI\hyp{}TOF analysis of human granulomas) has been a strong pro\hyp{}angiogenic signature. Using the zebrafish model, we now provide new insight into the molecular basis for and consequences of that angiogenesis. Delineating this pathway also provides new therapeutic targets for host\hyp{}directed approaches to treating TB.

Throughout, we use the tools available in the zebrafish to visualize the \textit{in vivo} kinetics and dynamics of \textit{Mycobacterium}\hyp{}driven angiogenesis and host response, including multi\hyp{}day timelapses using specific reporter lines, and CLARITY\hyp{}cleared organs to analyze the intricate neovasculature that is driven by mycobacterial granulomas and which leads to accelerated bacterial growth and dissemination.

We are excited about these findings and hope that this work will be of broad interest to TB biologists and clinicians as well as those developing potential host\hyp{}directed therapies for mycobacterial infections in pre\hyp{}clinical models. The identification of a specific NFAT isoform acting in macrophages suggests a new potential host target for therapies. In addition, because studies of NFAT have historically focused on T cells, uncovering an important role for NFAT signaling in macrophages should be particularly interesting to those studying macrophages and other myeloid cells in health and disease.

Thanks again for your consideration.

The following is the Response to Reviewers that accompanied the resubmission of \citet{Brewer2022} to \textit{Cell Reports}. Reviewer critiques from anonymous reviewers.

\emergencystretch=1em

\setlist[description]{style=multiline, labelwidth=\widthof{Reviewer \#2:   }, font=\normalfont, leftmargin=\labelwidth, align=right}  
                    
\begin{description}[style=multiline, labelwidth=\widthof{Reviewer \#2:   }, font=\normalfont, leftmargin=\labelwidth, align=right]
                    
\item[Response: ] \quad We are grateful to both reviewers for their thoughtful review of our manuscript and have made a number of changes and improvements, including new experiments and analyses as well as text changes to strengthen our conclusions and highlight specific places where this work contributes novel findings likely to impact our understanding of NFAT signaling in macrophages, the engagement of NFAT signaling pathways downstream of C\hyp{}type lectin recognition of microbial patterns, and specifically the role of these pathways in an important component of the host immune response to infection with pathogenic mycobacteria.

\item \quad In addition to our textual changes highlighting the novelty and potential impact of the work, we present new, clearer and more magnified representative images throughout to make clearer both the robust angiogenic response we observe \textit{in vivo} and our ability to quantitate \textit{in vivo} at baseline and upon perturbation of this response. Importantly these changes can be uniquely accessed \textit{in vivo} using the zebrafish model of mycobacterial infection.

\item \quad In response to both reviewers, we have made much clearer that all our quantitative analysis was performed blinded (further details are now provided in the STAR Methods section), and in the new version we perform a number of automated and blinded image analyses to further support our conclusions. In particular, we now better quantitate the mammalian cell culture/M. tuberculosis studies, relating them in a blinded, quantitative manner to NFAT nuclear localization and VEGFA induction. We have added multiple new panels to both Figures 6 and 7 to include this quantitation.

\item \quad In response to one reviewer’s concern about variability in effect size in mammalian culture, we have better optimized our assay and present more consistent results: three new independent biological replicates that are now presented as primary data instead. The findings are completely consistent with our previous findings, but with less variability in effect size. 

\item \quad Based on reviewer comments, we have also added further discussion of the significance of our findings, including some discussion of the importance of extracellular bacteria engaging this signaling pathway and the reported unique signaling kinetics of NFATC2, the NFAT paralog that we identify here – both \textit{in vivo} in the zebrafish and in mammalian cell culture experiments. We think that overall this work moves across two systems, one \textit{in vivo} animal model in both larvae and adults, and another in human cell culture with M. tuberculosis, to identify a previously unknown role for NFAT signaling during mycobacterial infection and granuloma angiogenesis, which we know to be an important host process manipulated by pathogenic mycobacteria to promote and sustain infection. Thus, this work also identifies a new host target for potential therapeutic interventions in TB, which, even in the age of COVID\hyp{}19, has now again become the leading infectious cause of death worldwide.

\item \quad Specific responses to each reviewer’s comments are included below.

\item[Reviewer \#1: ] \quad This study addresses the mechanism used by pathogenic mycobacteria and the mycobacterial lipoglycan trehalose dimycolate (TDM), to stimulate angiogenesis in the region of granulomas. This group has previously reported that induction of angiogenesis is pathogen\hyp{}beneficial in their model of \textit{M. marinum} infection in zebrafish, and has now addressed the signaling pathway involved in regulating macrophage production of proangiogenic VEGFA. The key finding in the manuscript is that macrophage NFATc, especially NFATc2a, signaling is essential, while CARD9 signaling is less important, for the effects observed. Certain of the results generated in the zebrafish models are confirmed using a human macrophage\hyp{}like leukemia cell line exposed to irradiated \textit{M. tuberculosis}. The major innovative finding is that NFAT signaling can be involved downstream of C\hyp{}type lectin receptor signaling; a finding that may be generalizable.

\item[Response: ] \quad Thank you for the thoughtful review and for this assessment of the work, which we have now strengthened, including new experiments and analysis. We agree with the reviewer that the question of NFAT signaling downstream of C\hyp{}type lectin receptor signaling and particularly in myeloid cells is almost entirely unstudied across fields, and that there is particular importance and impact to understanding the architecture of this response generally and, moreover, the details for the immune response to a major human pathogen. Our work in zebrafish and work from multiple others with human clinical samples and in non\hyp{}human primate models have identified the importance of VEGFA induction at the mycobacterial granuloma during pathogenesis, and, for some TB disease presentations, there are active clinical efforts to harness this pathway therapeutically. Moreover, as the reviewer points out, there is considerable novelty and generalizability in understanding myeloid engagement of NFAT signaling downstream of C\hyp{}type lectin receptor activation.

\item[Reviewer \#1: ] \quad There are two ways to consider the significance. With regard to promoting the understanding of mycobacterial/TB pathogenesis, the findings might be considered incremental, since the investigators have provided ample evidence of angiogenesis and its significance in this model. On the other hand, the finding of NFAT signaling in C\hyp{}type lectin induction of VEGFA by macrophages has the potential to be generalizable to other C\hyp{}type lectin signaling contexts.

\item \quad There are no major concerns with the experimental, analytical, or data [\textit{sic}], in the study. One concern is whether the finding that NFAT signaling downstream of C\hyp{}type lectin engagement in macrophages contributes to angiogenesis is unique to this system, or if it is generalizable. More information on the possibility that the observations relate to signaling that is only engaged by a high concentration of agonist (i.e. numerous extracellular bacteria) could shed light on this possibility\footnote{In isolation, the findings reported here do not solve a major question in mycobacterial\hyp{}host interactions, but may provide a step toward addressing the larger question of how angiogenesis is pathogen\hyp{}beneficial in mycobacterial infections.}.

\item[Response: ] \quad Thanks for this assessment and we agree that the finding of an important physiological role for NFAT signaling in macrophages, downstream of C\hyp{}type lectin signaling is a largely undescribed and important aspect of NFAT and macrophage biology. As detailed above, we respectfully disagree with the characterization that this work in the context of TB is only a limited advance. While it fits together well with previous work we have published 1) describing the impact of angiogenesis on mycobacterial granulomas \citep{Oehlers2015} and 2) identifying a specific bacterial enzymatic modification to a predominant cell envelope lipid that underlies this response \citep{Walton2018} – the downstream host steps that mediate this important aspect of TB pathogenesis had not been studied. Here we found – both in an \textit{in vivo} model that forms true granulomas and translating these findings into mammalian cell culture – that a previously undescribed engagement of NFAT signaling in macrophages drives granuloma\hyp{}associated angiogenesis.

\item \quad In this work, we provide the first real insight into how macrophage detection of the mycobacterial glycolipid TDM specifically manipulates the mycobacterium\hyp{}infected host, leading to VEGFA induction and angiogenesis, a response that enables the bacteria to survive, replicate and disseminate and is therefore of active clinical interest therapeutically for TB. Notably, standard models of TB disease (C57BL/6 mice, for example) do not recapitulate the granulomas seen in humans and non\hyp{}human primates, and so the zebrafish is one of the few genetically manipulable animal models where this important process can be dissected experimentally. This work 1) identifies activation of a previously unconsidered pathway that culminates in activation of NFATC2 as the key downstream transducer of VEGFA production during mycobacterial infection \textit{in vivo} in the zebrafish models and in a human cell line; and 2) assesses the consequences of perturbing NFAT signaling, both using loss\hyp{}of\hyp{}function mutants as well as macrophage\hyp{}specific inhibitory approaches in whole animals \textit{in vivo}. Generally, NFAT induction and transduction of pathogen\hyp{}associated signals have not been explored in myeloid cell biology, and so the link in both humans and zebrafish established here may also spur others to examine myeloid NFAT activation in different contexts during infection and disease.

\item[Reviewer \#1: ] \quad The observation that induction of angiogenesis is delayed for several days post infection, when extracellular mycobacteria have accumulated, may be interesting and provide additional clues to the biological role of angiogenesis and mycobacterial infection. The observation suggests that the Clec\hyp{}NFAT pathway might be due to low sensitivity and only activated by a high concentration of Clec ligand. It would be beneficial to address this possibility and consider its impact on mycobacteria\hyp{}host interactions. The Discussion addresses the observation, but not its potential unique significance.

\item[Response: ] \quad We agree with this suggestion and have highlighted the potential unique significance further in the Discussion. We have tried to address these observations throughout – first in Figure 1, where we put emphasis in the Results section on the fact that the induction of angiogenesis coincides with the emergence of extracellular bacteria, and, later in the paper, in the design of our mammalian cell culture assay. As we describe in the Discussion, it is notable that standard published THP\hyp{}1 infection assays (which aim to eliminate extracellular bacteria by antibiotic treatment) have not detected VEGF production at later timepoints. We have included discussion of this point and further emphasized this point throughout. 

\item[Reviewer \#1: ] \quad The manuscript notes the involvement of calcium signaling in NFAT activation in classical systems, but doesn’t address it experimentally. It would be interesting to know if intracellular calcium responses to mycobacteria and/or TDM through C\hyp{}type lectins require high ligand/agonist concentrations, or if the noncanonical role of NFAT is calcium\hyp{}independent. Inhibition by calcineurin blockade provides indirect evidence against the latter possibility, but doesn’t prove it.

\item[Response: ] \quad Thanks for this suggestion and we agree that inhibition with multiple compounds and with the VIVIT peptide, while highly suggestive, does not fully distinguish between these possibilities. In the scope of other revisions we were not able to address this question experimentally, but have added a new sentence into the Discussion to mention the possibility of a calcium\hyp{}independent potential role of NFAT and future research directions. We have also added additional quantitation in Figures 6 and 7 to make clear the dynamics and extent of the NFAT/INCA\hyp{}6 effects and the effects of inhibiting NFATC2 in cell culture.

\item \quad Interestingly, in considering the reviewer’s comments about the kinetics of NFATC2 activation, calcium dynamics, and exposure to extracellular bacteria, we came across literature implicating NFATC2 in a slower and more sustained process of activation relative to the other NFATc paralogs. This known property of NFATC2 and the associated references are now further highlighted in the discussion \citep{Kar2015, Kar2016} and may also relate to the reviewer’s idea of persistent or higher extracellular ligand concentrations.

\item[Reviewer \#1: ] \quad As presented, it is difficult to relate the images and the quantitation of angiogenesis. The images shown are low power, and it seems unlikely that such low power images were used to generate the quantitative data used to interpret the results.

\item[Response: ] \quad In order to assess angiogenesis comprehensively, we did image across the entirety of the animals for the larvae, but of course can magnify specific regions of angiogenesis as well as examining multiple planes. (What is shown are maximum intensity projections). For the larvae, we have now used higher magnification and clearer examples and images of the neovascularization in the figures (still representative of median data points). We also describe in additional detail our blinded quantitation for the larvae and automated (therefore also blinded) quantitation for the cell culture experiments, which revealed robust and significant differences in granuloma\hyp{}associated angiogenesis as well as NFAT\hyp{}dependence for VEGFA expression and angiogenesis in the cell culture experiments. The additional images and, in the case of the cell culture experiments, new automated quantitation, are now incorporated into the main figures, and the STAR Methods now describes more extensively the methods we used for quantitation.

\item[Reviewer \#1: ] \quad Very minor point: the legend for Fig3A is, “NFAT CRISPR Screen”, but results of only two NFATs are shown.

\item[Response: ] \quad We apologize for using the “screen” terminology, since this evokes large\hyp{}scale CRISPR screens. We have removed the term and had only mean to imply that, based on the VIVIT and inhibitor results, we considered multiple NFAT paralogs. By combining expression analysis from published datasets with functional analysis of zebrafish CRISPR mutants that we made, we were able to identify \textit{nfatc2a} as the key paralog required for the response.

\item[Reviewer \#1: ] \quad Overall, the manuscript is very well written and organized. The methods, sample numbers, and analyses are all appropriate. As noted above, the link between images demonstrating angiogenesis and the quantitative data could be improved.

\item \quad The videos are a minor addition without additional labeling on the videos themselves, to highlight the points made.

\item[Response: ] \quad Thanks for the suggestions, and we have reworked our magnification of the zebrafish images throughout the first few figures so that they are more apparent to the reader. We were trying to include the entirety of the animal to be transparent that we were not selectively showing only a part of each animal, but have now included more zoomed\hyp{}in images of the angiogenesis that we score. In addition, we have added detail regarding how we performed blinded quantitation in the STAR Methods section.

\item[Reviewer \#2: ] \quad In this manuscript by Brewer et al. the authors examine the link between mycobacterial infection, host macrophages, and induction of host angiogenesis using zebrafish and human cell culture models. The authors cite a previously published report suggesting that mycobacterium\hyp{}promoted increased vessel growth may facilitate bacterial growth and dissemination. They also note their own previously published study showing that a mycobacterium\hyp{}derived cell wall component (TDM) promotes vegfaa production and angiogenesis at the site of mycobacterial infection and granuloma formation. Here, they seek to extend on their previous findings on TDM with new results showing that mycobacteria/TDM lead to activation of NFAT signaling, and NFAT\hyp{}dependent induction of VEGF, in host macrophages. Although these findings are interesting and potentially highlight new targets for disrupting mycobacterial infection, the findings seem somewhat incremental in significance and a number of additional concerns detract from enthusiasm for this study.

\item[Response: ] \quad We appreciate the thoughtful feedback and perspective throughout. Although of course this is a judgement call, we disagree with the idea that these findings are incremental. NFAT signaling has not previously been implicated in granuloma\hyp{}mediated angiogenesis, a physiologically important response to mycobacterial infection. Our initial work in the zebrafish identifying the biological relevance of angiogenesis during pathogenesis \citep{Oehlers2015} came out just before concurrent work establishing the relevance of angiogenesis in TB disease in humans and non\hyp{}human primates, and there is active clinical interest in pursuing these pathways. Using the zebrafish model, we then established the specific mycobacterial lipid required for inciting angiogenesis \citep{Walton2018}, but how that lipid transduced the signal for VEGF production was not understood.

\item \quad In this work, we provide the first real insight into how macrophage detection of that lipid specifically manipulates the mycobacterium\hyp{}infected host, leading to VEGFA induction and angiogenesis, a response that enables the bacteria to survive, replicate and disseminate and is therefore of active clinical interest therapeutically for TB. Notably, standard models of TB disease (C57BL/6 mice, for example) do not recapitulate the granulomas seen in humans and non\hyp{}human primates, and so the zebrafish is one of the few genetically manipulable animal models where this important process can be dissected experimentally. This work 1) identifies activation of a previously unconsidered pathway via NFATC2 as the key downstream transducer of VEGFA production during mycobacterial infection \textit{in vivo} in the zebrafish models and in a human cell line; and 2) assesses the consequences of perturbing NFAT signaling, using both loss\hyp{}of\hyp{}function mutants as well as macrophage\hyp{}specific inhibitory approaches in whole animals \textit{in vivo}. Generally NFAT induction and transduction of pathogen\hyp{}associated signals have not been explored in myeloid cell biology, and so the link in both humans and zebrafish established here may also spur others to examine myeloid NFAT activation in different contexts during infection and disease.

\item \quad We detail below both experimental and analytical additions to address the reviewer’s additional concerns. In particular, we had not previously explained fully the rigorous blinding methodology that we used throughout, and we have now made this much more explicit up front and in the STAR Methods. In addition, we include new automated computational analysis for the human cell culture work that confirms and strengthens our findings, presenting new panels featuring automated blinded quantitation in Figures 6 and 7.

\item[Reviewer \#2: ] \quad Many of the zebrafish \textit{in vivo} findings presented in Figs 1\hyp{}5 are based on image data whose resolution and/or magnification do not always allow clear interpretation or clear visualization of the authors conclusions. Despite statistical significance in the authors measurements, the vessel effects appear modest, particularly in the larvae, and it is sometimes difficult to see how they were quantitated from the image examples presented, or how these sorts of limited vessel sprouts and segments, which are unlikely to be carrying much if any additional flow from the authors images, could be significantly influencing mycobacterial growth or dissemination. Since these measurements appear to rely on visual inspection and measurement of vessel/sprout length and are potentially more subjective than quantitative molecular measurement methods, it is also important that they are done fully blinded and that the blinding methods are fully described in the methods section.

\item[Response: ] \quad We appreciate raising these important points. We had not previously fully explained the rigorous blinding methodology that we had used throughout, and we have now made this much more explicit up front and in the STAR Methods. In addition, we include new automated computational analysis for the human cell culture work that confirms and strengthens our findings, presenting new panels featuring automated blinded quantitation in Figures 6 and 7.

\item \quad For the larvae, we have now used more highly magnified and clearer examples and images of the neovascularization in the figures (still representative of median data points). Regardless, our blinded quantitation reveals robust and significant differences in granuloma\hyp{}associated angiogenesis in blinded analysis. We also note that, because of the timing \citep{Oehlers2015, Walton2018}, there is a small window of about 2 days (from 3 dpi to 5 dpi) as the response begins only once the initial granuloma has assembled. Unfortunately, the larval system does not permit the longer\hyp{}term analysis that is available in adults. Although the larval effect sizes are reasonable and consistent, the adult effect size is more dramatic, as we can look over a two\hyp{}week time\hyp{}frame; use of the adults also allowed for traditional analysis of bacterial burden. The (blinded) quantitation in adults also certainly underestimates the effect size on angiogenesis, as it includes pre\hyp{}existing vessels throughout that are included in all analyses. We have no automated way of excluding these pre\hyp{}existing vessels in the adult and so have used the most conservative approach of including them, which necessarily drives up the background, but we also note that we do see qualitative differences in the \textit{nfatc2} mutants and the macrophage\hyp{}specific VIVIT line. Again, we had done all the quantitation previously in a completely blinded manner, and now have better described the process and blinding in the STAR Methods section as well as increased the magnification of some of the images to make the effect more apparent.

\item \quad Additional text has been added to the STAR Methods to reflect the blinding performed for the experiments and we apologize for any issues with clarity in the existing manuscript. We fully agree that the subjective nature of these types of measurements require blinding to be robust and reproducible. The blinding for these assays were conducted as follows: for assays where experimental blinding was difficult or infeasible (drug treatment assays, transgenic \textit{irg1} fish, adult infections), computational blinding was conducted using either the blindrename.pl script \citep{Salter2016} or an in\hyp{}house developed Python translation of the same (included in the Zenodo submission accompanying this manuscript). This script renames all the files to a random string of characters and generates a keyfile to allow for post hoc matching of blinded images to treatment groups or genotypes. For all assays necessitating post hoc genotyping (larval infection of host mutants), genotypic blinding was used. Measurements were collected on each of the images prior to matching them to the determined genotype. 

\item[Reviewer \#2: ] \quad Besides vessel length and a few measurements of mycobacterial CFU in figure 5, no other quantitative data are provided for the zebrafish portion of this study (e.g., rigorous molecular measurement of NFAT activation or VEGF induction in host macrophages), despite the technical feasibility of such measurements, using the transgenic lines available to the authors. Furthermore, if the authors wish to confirm the requirement for macrophages in induction of host angiogenesis in zebrafish larvae \textit{in vivo}, they could directly examine this by genetically ablating these cells, which can readily be accomplished in fish. The authors could also perform macrophage\hyp{}specific targeting of \textit{nfatc2a} and \textit{nfatc3a} using \textit{irg1}:\textit{Cas9} and guide RNAs targeting each of these genes, to confirm the specific requirement for \textit{nfatc2} in these cells \textit{in vivo}.

\item[Response: ] \quad These are good suggestions, and we had tried a number of these approaches previously but, in the context of the larva, we are particularly limited by some of the tool availability and development. In our hands the larvae are too small to dissect spatially, and, at this particular developmental stage, there are significant levels of vegfaa production throughout, making whole\hyp{}animal molecular measurements in this context uninformative. We have tried multiple approaches to generate NFAT reporter transgenic lines, including in\hyp{}frame fusions and lines based on NFAT binding sites, but have thus far been unable to generate such a line. There were apparent toxic effects from whole\hyp{}organism reporter transgenes and we were unable to identify transmitting founders from our attempts at macrophage\hyp{}specific reporters. We are also limited by the lack of antibodies that exist for fish.

\item \quad We also agree that the macrophage\hyp{}specific targeting of \textit{nfatc2a} would be a great experiment but do not currently have the technology to do so. We have tried for a number of years now to get a macrophage\hyp{}specific Cas9 to express reliably, including using the irg1 promoter, but without success. In collaboration with Qing Deng’s laboratory we have published on functional neutrophil\hyp{}specific Cas9 \citep{Wang2021}, but macrophages have proven more challenging. There is one report of a macrophage\hyp{}specific Cas9 from Graham Lieschke using a GAL4\hyp{}UAS system \citep{Isiaku2021}. The requisite two lines are in Australia and would be an additional number of months to formally import, get through quarantine in our facility and validate, and we (and the zebrafish community generally) have often had issues with transgene silencing using the GAL4\hyp{}UAS system. So while we recognize that this particular approach would add to the evidence presented in the paper, we don’t think it is reasonably within the scope of a revision.

\item \quad The suggested macrophage depletion experiment is one that we have done previously and published \citep{Oehlers2015} showing dependence on macrophages for angiogenesis. In the current work, instead of depleting all macrophages generally, we have used the VIVIT peptide, specifically expressed in macrophages, to show that \textit{in vivo}, macrophage\hyp{}specific expression of VIVIT abrogates granuloma\hyp{}associated angiogenesis. The conserved nature of the specific \textit{nfatc2} gene in the response to pathogenic mycobacteria in the human cell culture experiments with macrophage\hyp{}like THP\hyp{}1 cells also contributes to further validation of the findings.

\item[Reviewer \#2: ] \quad There are some related concerns about the \textit{in vitro} human macrophage quantitative data in Figures 6 and 7. The different replicates of the VEGFA qPCR and ELISA results shown in fig. 6 and supplemental Fig. 4 have strongly differing results, suggesting there may be other variables that are not being adequately controlled in these assays that may be having substantially larger effects then the MTB and INCA\hyp{}6 effects, and making selection of a “representative” result somewhat unclear. 

\item[Response: ] \quad We have tried to be complete and transparent throughout, showing each of the experiments in the initial submission (even the outlier that showed much higher induction) and we consistently see the INCA\hyp{}6 effect. We agree that there is variation in the level of VEGFA induction. We have been completely transparent about the extent of the effect in our original assays (from 5\hyp{}fold to 50\hyp{}fold induction, with our main figure representation showing ~12 fold induction). We believe that some of the heterogeneity in response may have come from the inherent “clumpiness” of the mycobacteria (they are lipid rich and tend to clump, providing uneven exposure) or properties in how the macrophages themselves were plated prior to infection. To confirm that we could improve on the control of other variables, we ran three additional independent experiments for the revision in which we now coated with poly\hyp{}D\hyp{}lysine before plating, as we had had to do for the immunofluorescence to see if this increased overall reproducibility. In the additional experiments, which we now present in Figure 6B and Supplemental Figures 4A and 4B, we see more consistency in the effect size, although the mean and median effect sizes are approximately the same as before.  Thus, under two slightly different experimental protocols (one without poly\hyp{}D\hyp{}lysine and one with) we observe a consistent induction of VEGFA and consistent inhibition of VEGFA production upon INCA\hyp{}6 treatment. For simplicity, we have incorporated the second set of experiments as the primary data. We have also included quantitation by three different methods: qRT\hyp{}PCR, ELISA, and immunofluorescence. To make this clearer, we have included new Figures 6G and 6H to better relate the nuclear localization of NFAT to VEGF production and quantitate (in a blinded fashion) the results of INCA\hyp{}6 exposure. To address the reviewer’s concern, we have included data from three independent biological replicates across >5000 cells.

\item[Reviewer \#2: ] \quad Is macrophage number equivalent in all these experiments? How are the values normalized? Despite activation of NFAT being a key point in this manuscript, the authors do not perform rigorous quantitative measurement of NFAT activation by careful measurement of nuclear/cytoplasmic compartmentalization or other methods (this should be done in conjunction with simultaneous quantitative measurement of VEGFA induction). The data in Fig. 7 are notably deficient in this regard. The VIVIT experiments need validation of the efficacy and specificity of its effects on NFATC activation. And if the authors want assess the role of NFATC2 in their \textit{in vitro} model, it would perhaps be better to select germline CRISPR knockouts of this gene in their cells, and perhaps perform siRNA knockdown to confirm. 

\item[Response: ] \quad For the cell culture experiments, we plate the same number of macrophages for each experiment (2.5x10\textsuperscript{5} per well in a 24\hyp{}well plate, normalized to 5x10\textsuperscript{5} cells per mL across all experiments regardless of plate size) and use aliquots of the gamma\hyp{}irradiated \textit{M. tuberculosis} and INCA\hyp{}6. As mentioned above, some of the variability between replicates likely arises from the heterogeneity of \textit{M. tuberculosis} exposure, with the extracellular bacteria tending to clump due to their incredibly lipid\hyp{}rich cell envelope, but we also have gained some consistency via coating the wells with poly\hyp{}D\hyp{}lysine. Notably, we think that the extracellular nature of the exposure may reflect conditions \textit{in vivo}. We now mention in the Discussion that, as we observe in the zebrafish model, engagement of this pathway may be driven by extracellular bacteria.

\item \quad For VIVIT validation, the peptide itself has been exhaustively validated over the last two decades and the zebrafish target site is 100\% conserved for zebrafish \textit{nfatc2a} as well as the other five \textit{nfatc} paralogs. We also have a reliable readout for macrophage\hyp{}specific VIVIT expression with the fluorescence construct. We agree that, in the future, a more thorough characterization of the effects that result from macrophage\hyp{}specific VIVIT expression would be of interest and may provide further validation of this tool. The fact that \textit{nfatc2a} mutants recapitulate the VIVIT phenotype is also notable in terms of the success of the VIVIT approach. As described above, we invested a lot of effort in trying to generate reporter lines that would give us an \textit{in vivo} readout of NFAT activation, using multiple approaches, but were unable to generate such lines, and there are no available antibodies that appear to cross\hyp{}react specifically with zebrafish NFAT (we have tried at least 4 different antibodies). So, while the zebrafish model provides important advantages in dissecting this process and the effects of mutants, drugs, and cell\hyp{}specific genetic inhibition on the process \textit{in vivo}, NFAT measurements that would be accessible in mammalian cell culture are thus far limited. But we think that the subsequent validation that we perform in mammalian cell culture highlights the relevance of the \textit{in vivo} findings and the conservation of this response during mycobacterial infection. 

\item \quad In terms of the siRNA knockdown idea, we purchased and used the four commercially available \textit{NFATC2} siRNAs from Qiagen. In contrast to the lentiviral approach that we ended up using (new Supplementary Figure 4J), we did not see efficient knockdown of \textit{NFATC2} with these siRNAs. The purchased siRNAs, although supposedly guaranteed by Qiagen, were manufactured in 2011 and so, discouraged from this approach, we pursued the lentiviral constructs to directly assess NFATC2 in mammalian experiments with gamma\hyp{}irradiated M. tuberculosis, to assess whether our \textit{in vivo} findings in the zebrafish extended to human macrophages and to M. tuberculosis.

\item \quad In both Figures 6 and 7, we have now used rigorous, blinded, quantitative computational analysis of immunofluorescence images of NFAT and VEGFA to precisely address the reviewer’s concern. The automated quantitation is now described further in the STAR Methods section, the Results section and the figure legends. 

\item \quad In the new Figure 6, panels E, G, and H now in an unbiased, blinded way present quantitative assessment of VEGFA production in conjunction with NFAT nuclear localization in response to gamma\hyp{}irradiated M. tuberculosis in the presence and absence of INCA\hyp{}6 drug treatment, with analysis of >5000 cells. 

\item \quad In the new Figure 7, panels C, F, G, and H now use the same blinded quantitative approach to simultaneously assess VEGFA induction and NFAT nuclear localization in the context of the specific genetic inhibition of \textit{NFATC2}. This blinded, quantitative analysis further confirms our initial findings and addresses concerns about the robustness of the effects and the dataset.

\item[Reviewer \#2: ] \quad (In re: clarity of reporting) Mostly yes, but more information on blinding methods for quantitative measurements (most of which appear to involve visual scoring of image data) in this manuscript would be helpful.

\item[Response: ] \quad As described in more detail above, we have better described our blinding methods for quantitative measurement (STAR Methods and description in Results) as well as applied new extensive and rigorous blinded and automated quantitation for Figures 6 and 7 in additional figure panels as requested by the Reviewer.

\end{description}