\section{Abstract}

As introduced in the previous chapters, pathogenic mycobacteria manipulate the host immune response to generate a productive infection that enables them to transmit to new hosts. To do so, they must both subvert protective immune responses and agonize the pathways that provide them benefit in the way of nutrients, susceptible host cells, or routes of transmission. This necessitates the engagement of both the immune cells themselves and the surrounding stromal cells, all of which much be subverted to generate the optimal pro-bacterial response. One of the mechanisms by which this can occur is through a specific modification of the specialized cell wall lipid, trehalose 6-6'-dimycolate (TDM). This mycobacterial glycolipid has been found to direct angiogenesis toward nascent granulomas to enhance overall bacterial burden. The present study utilizes the zebrafish-\textit{Mycobacterium marinum} infection model to define the signaling basis of the host angiogenic response. Through intravital imaging and targeted, cell-specific peptide-mediated inhibition, we identify macrophage-specific activation of NFAT signaling as essential to TDM-mediated angiogenesis in vivo.  Exposure of human cells to \textit{Mycobacterium tuberculosis} results in robust induction of VEGFA that is dependent on a signaling pathway downstream of host TDM detection and culminates in NFATC2 activation. As granuloma-associated angiogenesis is known to serve bacterial-beneficial roles, these findings identify potential host targets to improve tuberculosis disease outcomes.

\section{Introduction}

The host rejoinder to infection is driven by an intricately regulated, but occasionally discordant or maladaptive, immune response to pathogenic stimuli at the cell-intrinsic, innate, and adaptive levels \citep{Iwasaki2010, Finlay2006, Haldar2015}. Although an inflammatory immune response is essential to host survival and pathogen killing, an overly robust response is clearly deleterious to the host, as seen in sepsis \citep{Finethy2020}. The contributions of immune cells to host defense have been widely studied, but there is growing appreciation for the contributions made my non-immune populations, including stromal cells and the endothelium \citep{Honan2021, MacMicking2012, Worrell2021, Amersfoort2022} make in shaping the host response to both acute and chronic infections \citep{Mueller2009, Randow2013, Krishnamurty2020}. Pathogens have long been known to possess sophisticated mechanisms to undermine signaling pathways in immune cells and, more recently, have been shown to manipulate development and homeostatic tissue processes to force them toward pathogen-beneficial conditions \citep{Menzies1998, Guichard2013}.

\textit{Mycobacterium tuberculosis} is among history'€™s most widespread and successful pathogens. It has evolved an array of sophisticated mechanisms that manipulate its human host to enable bacterial survival, replication, and transmission. Upon infection, \textit{M. tuberculosis} sets in motion an intricate immune response wherein innate immune cells, consisting initially of macrophages, congregate at the bacterial focus and then undergo an epithelioid transformation and interdigitate to form an encased granuloma, the hallmark feature of tuberculosis, which provides both the replicative niche and the major host-pathogen interface of tuberculosis disease \citep{Cronan2016, Pagan2018, Cronan2021}. Granuloma-associated vasculature has long been noted in human and animal models of TB \citep{Cudkowicz1952, Russell2010} but the mechanisms of induction and precise contributions to infection are not yet fully understood.

Many of the major pathological features of mycobacterial granulomas, including associated vascularization, are conserved from zebrafish to humans \citep{Swaim2006, Bohrer2021}. Zebrafish can be infected with a natural pathogen, \textit{Mycobacterium marinum}, which induces a robust angiogenic response during granuloma formation. This process, much like that in humans, non-human primates, and rabbits, is associated with production of a pro-angiogenic chemokine, Vegfaa, at the site of infection \citep{Oehlers2015}. This chemokine has long been known to be a critical regulator of angiogenesis in both developmental and pathological contexts \citep{Chung2011, Leung1989, Adams2007}. Similarly, human granulomas have been shown to express VEGFA and are physically associated with blood vessels that penetrate the outer granulomatous layers \citep{Datta2015}. Subsequent work has demonstrated a role for these vessels in supporting bacterial growth and in dissemination of the bacilli from their primary site of infection \citep{Polena2016}. Additional roles for VEGFA in non-angiogenic processes have also been noted, suggesting that angiogenic signaling cascades can also alter the biology of granuloma macrophages to improve disease outcomes \citep{Harding2019}. Additional roles have been proposed for the related lymphangiogenesis \citep{Alitalo2005, Duong2012, Lerner2020}, where pharmacological blockade of these vessels of the lymphatic system also offer host-protective benefits \citep{Harding2015}. Recent profiling of human and non-human primate granulomas have confirmed the presence of aberrant vasculature associated with \textit{M. tuberculosis} granulomas \citep{Gideon2022, McCaffrey2022, Cronan2021}.

Pathogenic mycobacteria have evolved specialized mechanisms to promote and accelerate angiogenesis. Notably, the extensively modified and essential outer cell envelope component trehalose 6-6'€™-dimycolate (TDM) is cis-cyclopropanated by the enzyme PcaA \citep{Glickman2000, Rao2005}. Mutation of pcaA results in a reduction in granuloma angiogenesis and reduction in bacterial burden; correspondingly, cyclopropanated TDM alone is sufficient to induce host angiogenesis \citep{Saita2000, Sakaguchi2000, Walton2018}. As \textit{pcaA}-dependent vascularization supports bacterial growth, factors driving this represent potential sites of therapeutic intervention yet the signals that mediate this host process remain unclear.

TDM is an extraordinarily long-chain, hydrophobic (C\textsubscript{60}-C\textsubscript{100}) glycolipid \citep{Noll1956a, Noll1956b, Hunter2006a, Behling1993} that has been shown to be detected in cell culture and murine models by host C-type lectin receptors, most notably MCL (CLEC4D) and MINCLE (CLEC4E), as well as by Toll-like receptor 2 (TLR2) and MARCO \citep{Bowdish2009, Matsunaga2009, Miyake2013, Ishikawa2009}. Canonically, C-type lectin signaling is transmitted through a CARD9-NF-$\upkappa$B signaling pathway that results in the transcription and production of TNF-$\upalpha$, IL-1$\upbeta$, IL-6 and other cytokines \citep{Yamasaki2008, Goodridge2009, LobatoPascual2013, Zhao2014, Deerhake2021}. However, beyond CARD9, a number of other downstream signaling pathways are engaged by C-type lectin activation and likely control discrete aspects of signaling \citep{Goodridge2007, Deerhake2021}.

Here, we synthesize findings from zebrafish and cell culture models to define the \textit{in vivo} angiogenic response induced by pathogenic mycobacteria. Contrary to classical models of C-type lectin signaling, we find that cis-cyclopropanated TDM exerts its pro-angiogenic effects through an alternative NFAT-driven pathway rather than canonical CARD9-NF-$\upkappa$B signaling. We use peptide-mediated, cell-specific inhibition of NFAT to demonstrate that both early and mature granuloma angiogenesis are dependent upon macrophage-NFAT signaling. We identify \textit{NFATC2} as the predominant isoform mediating \textit{VEGFA} induction and angiogenesis. These findings define the basis of granuloma-associated angiogenesis during pathogenic mycobacterial infections and suggest new targets for host-directed therapeutic interventions during tuberculosis.

\section{Results}

\subsection{Macrophage Induction of \textit{vegfaa} and Angiogenesis during Mycobacterial Infection}

Injection of live \textit{Mycobacterium marinum} into the dorsal trunk of the zebrafish larva is sufficient to induce a robust angiogenic response adjacent to nascent granulomas in a macrophage-dependent manner \citep{Oehlers2015} (Fig. 1A). The stereotyped vasculature along this region of the larva allows facile quantitation of neovascularization during and after granuloma formation or other insult \citep{Lawson2002, Jin2005, Gore2012, Matsuoka2018}. We demonstrated previously that cis-cyclopropanated trehalose 6-6'€™-dimycolate (TDM) is required for the induction of \textit{vegfaa} and angiogenesis at the site of infection. Furthermore, we found that genetic blockade of Vegfaa signaling was sufficient to abolish angiogenesis during infection with wild-type mycobacteria \citep{Walton2018}. Taken together, these findings suggest that the failure to induce \textit{vegfaa} is a major contributor to the loss of angiogenesis in \textit{pcaA}-deficient granulomas.

To study this phenomenon further, we began by examining the kinetics of \textit{vegfaa} induction to identify the cellular source of Vegfaa during granuloma formation. To test whether macrophages were playing this role, we developed a macrophage-specific reporter using the previously described \textit{acod1} promoter (also known as \textit{irg1}), Tg(\textit{irg1}:\textit{tdTomato\textsuperscript{xt40}}) (from here, \textit{irg1}:\textit{tdTomato}). \textit{irg1} has been found to be expressed specifically in zebrafish macrophages and is upregulated during infection \citep{Sanderson2015, Kwon2022}. We then crossed this line with the \textit{vegfaa} reporter line TgBAC(\textit{vegfaa}:\textit{eGFP\textsuperscript{pd260}}) (\textit{vegfaa}:\textit{eGFP} throughout) \citep{Karra2018} and infected double transgenic \textit{irg1}:\textit{tdTomato}; \textit{vegfaa}:\textit{eGFP} progeny with \textit{M. marinum} expressing eBFP2 (\textit{Mm}-eBFP2) to simultaneously visualize bacteria, macrophage localization, and \textit{vegfaa} production \textit{in vivo} \citep{Takaki2013}.

We began imaging at a time point that preceded robust induction of \textit{vegfaa}:\textit{eGFP} but would allow us to capture the maximum time span of these events. We observed an increase in \textit{vegfaa} reporter signal over time that appeared largely localized to macrophages (Fig. 1B). We observed that bacteria initially grew primarily intracellularly within individual macrophages at 36 hours post infection but began to grow in characteristic extracellular cords by approximately 84 hours post infection with little to no intracellular containment at this site by 96 hours post infection (Fig. 1C). The increase in extracellular growth coincided with the induction of eGFP signal in macrophages at ~64 hours (Fig. 1B), suggesting that, at low overall burden, intracellular detection is unable to induce \textit{vegfaa} expression while extracellular engagement correlates with \textit{vegfaa} expression during early stages of granuloma formation (Fig. 1B; Supplemental Movie 1). 

We next visualized the production of angiogenic vessels throughout infection in parallel to our characterization of \textit{vegfaa} induction. Due to an inability to separate discrete emission wavelengths using two GFP reporter lines, we were unable to examine all four components (bacteria, \textit{vegfaa} induction, macrophages, and vasculature) simultaneously. To relate this process directly to the angiogenesis observed in mycobacterial granulomas, we crossed the \textit{irg1}:\textit{tdTomato} macrophage reporter to the Tg(\textit{kdrl}:\textit{eGFP\textsuperscript{s843}}) (from here, \textit{kdrl}:\textit{eGFP}) line, which labels vasculature (\textit{irg1}:\textit{tdTomato}; \textit{kdrl}:\textit{eGFP}) \citep{Jin2005}. Under the same conditions and burden at which we infected the \textit{vegfaa} and macrophage dual reporter line, we observed robust vascularization at approximately 96 hours post-infection, subsequent to initial granuloma formation and \textit{vegfaa} induction (Fig. 1C, 1D, 1E, Supplemental Movie 2). 

\subsection{Genetic \textit{card9} Deficiency Does Not Compromise Angiogenesis During Mycobacterial Infection}

Given these observations suggesting that macrophages engaging extracellular bacteria are an important source of \textit{vegfaa} expression, we interrogated pattern recognition receptor (PRR) signaling pathways that had been implicated in host responses to TDM, a major external component of the mycobacterial cell envelope. We had previously found that \textit{myd88} was dispensable for the induction of angiogenesis in response to TDM in vivo \textbf{FIGURE} \citep{Bowdish2009, Walton2018}. This suggested that the described TLR2-mediated responses that function downstream of TDM detection in some contexts were unlikely to be required for this process. Rather, we found that the Fc$\upgamma$ R homologs in zebrafish, \textit{fcer1g} and \textit{fcer1gl}, are required for the full angiogenic response to TDM \citep{Walton2018}, implicating C-type lectin receptors signaling in mediating this response \citep{Richardson2014, Zhao2014}.

As many of the downstream activities of C-type lectin receptors have been ascribed to the activation of CARD9-NF-$\upkappa$B signaling \citep{Goodridge2009, LobatoPascual2013, Zhao2014, Williams2017, Deerhake2021}, we assessed what role this pathway might play in angiogenesis during mycobacterial infection. We developed a card9 knockout zebrafish line using CRISPR/Cas9 that carries a 28 bp insertion, resulting in an early stop after 59 amino acids (\textit{card9\textsuperscript{xt31}}) (Supp Fig. 1A). We then assayed these animals in the \textit{kdrl}:\textit{eGFP} transgenic background by incrossing \textit{kdrl}:\textit{eGFP}; \textit{card9\textsuperscript{xt31/+}} animals and infecting the resulting offspring with tdTomato-fluorescent \textit{M. marinum} (\textit{Mm}-tdTomato) at 2 days post fertilization (dpf) \citep{Jin2005, Oehlers2015}(Fig. 2A, 2B). We quantitated the resulting aberrant vasculature at 4 days post-infection (dpi) under genotypic blinding and post hoc matched these measurements to genotype. There were no significant differences between the three genotypes (Supp. Fig. 1B, 1C), suggesting either redundancy between multiple established pathways or the existence of an alternative pathway downstream of TDM detection that was \textit{fcer1g}/\textit{fcer1gl}-dependent, but independent of both \textit{myd88} and \textit{card9}. 

\subsection{Pharmacological Inhibition of NFAT Activation Limits Angiogenesis in Response to Mycobacteria and TDM}

Although many of the physiological consequences of C-type lectin receptor induction are often ascribed to CARD9-NF-$\upkappa$B signaling, this PRR class is also known to activate a distinct transcription factor family with known roles in immunity --€“ the nuclear factor of activated T cells, or NFAT \citep{Goodridge2007, Deerhake2021}. This calcium-responsive transcription factor pathway is best described in its role regulating T cell biology, but there are numerous reports describing various roles for the members of this pathway in other cell types, including macrophages (see \autoref{NFAT}) \citep{Symes1998, Jones2000, Crabtree2002, Horsley2002, Elloumi2012}. Given that there are four mammalian members of this pathway and six zebrafish homologs with potentially overlapping functions, we began with a pharmacological approach to globally inhibit NFAT signaling through all six zebrafish isoforms. Although this comes with caveats with specificity -- as described in \autoref{}, NFAT has important roles in angiogenesis within the endothelium -- it offers an opportunity to assess the general roles of this pathway as a first-pass approach. 

We first infected 2 dpf \textit{kdrl}:\textit{eGFP} larval zebrafish with \textit{Mm}-tdTomato in the trunk and treated them with 125 nM FK506, a clinically utilized calcineurin inhibitor that blocks NFAT activation, for the duration of the experiment \citep{Ellis1995}. This modest dose of FK506 was chosen due to developmental toxicities observed at higher doses and is in line with the dosage used by others \citep{Kujawski2014}. We imaged the fish at 4 dpi and quantitated the degree of vasculature induced in the presence and absence of inhibitor under computational blinding. Even with at this low dose of FK506, we noted a small, but statistically significant reduction in the mean degree of neovascularization at this time point, consistent with a role for NFAT in controlling angiogenesis in response to \textit{M. marinum} infection (Fig. 2C, 2D) \citep{Kujawski2014}. To ask whether this effect was specific to recognition of TDM, we injected purified TDM or vehicle (incomplete Freund'€™s adjuvant; IFA) alone into the trunks of 2 dpf larvae. Treatment with FK506 resulted in a statistically significant reduction in the degree of angiogenesis induced at 2 days post-injection (Fig. 2E, 2F), suggesting that this pathway was relevant specifically to TDM-dependent angiogenesis.

\subsection{The Isoform \textit{nfatc2a} is Specifically Required for Angiogenesis During Infection}

Combining our observations on the correspondence of granuloma formation and the induction of \textit{vegfaa} with our data implicating the NFAT pathway, we sought to identify NFAT isoforms that were enriched in granuloma macrophage populations. Aside from investigations made into \textit{nfatc1}, which is restricted to the endocardium, lymphatic vessels, and the notochord during much of zebrafish development \citep{Pestel2016, Shin2019, Bagwell2020}, little is known of the expression patterns of these genes in zebrafish\footnote{As mentioned previously in \autoref{nfatc4}, the NFATC4 isoform is thought to not be expressed in the hematopoietic compartment, an observation backed up by some more recent data as well \citep{Peuker2022}.}, especially in the context of infection. We first made use of published scRNA-seq datasets from mycobacterial granulomas in zebrafish and non-human primates for \textit{nfat} transcripts that were robustly expressed in granuloma macrophages at the population level and identified both zebrafish \textit{nfatc2a} and \textit{nfatc3a} as plausible candidates \citep{Cronan2021, Gideon2022}.

To examine potential roles for \textit{nfatc2a} and \textit{nfatc3a} in granuloma-associated angiogenesis in vivo, we first screened F\textsubscript{0} CRISPR-injected mosaic knockouts (''crispants'') to rapidly evaluate these candidate genes. Using this approach, similar to that used previously by other groups, we assessed the relative roles of these two isoforms individually and in tandem, measuring the angiogenic response to mycobacterial infection in the \textit{kdrl}:\textit{eGFP} background \citep{Jao2013, Hoshijima2016, Wu2018, Hoshijima2019, Kroll2021}. We found that \textit{nfatc2a} inhibition resulted in a $\sim$50-80\% reduction in angiogenesis. In contrast, \textit{nfatc3a} had no effect on the length of ectopic blood vessels present. The dual targeted double mosaics were statistically indistinguishable from the \textit{nfatc2a} injected fish alone (Fig. 3A, Supp. Fig. 2A, 2B). This allowed us to prospectively identify \textit{nfatc2a} as an NFAT isoform required for full angiogenic response to mycobacteria while \textit{nfatc3a}, despite expression in overlapping cell populations, appeared to be entirely dispensable for this process at this timepoint (Fig. 3A; Supp. Fig. 2A, 2B). 

We then established stable, germline transmitting indel mutant alleles for both genes to validate our results from mosaic animals. Recapitulating our results in the F0 generation, the \textit{nfatc3a\textsuperscript{xt59}} mutation carrying a 22 bp deletion (leading to an early stop codon at amino acid 9 in exon 1) had no effect on angiogenesis at 4 dpi (Fig. 3B, Supp. Fig. 2C). We then developed a knockout line of \textit{nfatc2a} bearing a net 4 bp insertion leading to an early stop codon in the second exon (at amino acid 273, frameshifted after amino acid 247), prior to the DNA-binding domain (\textit{nfatc2a\textsuperscript{xt69}}) (Supp. Fig. 2D). We repeated our angiogenesis assay using larvae from in-crosses of \textit{kdrl}:\textit{eGFP}; \textit{nfatc2a\textsuperscript{xt69/+}} animals that produced expected Mendelian ratios of wild-type, heterozygous, and homozygous mutant offspring. Consistent with the results from mosaic animals, homozygous knockout of \textit{nfatc2a} was sufficient to reduce the degree of angiogenesis present in larval zebrafish at 4 dpi (Fig. 3C, 3D; Supp. Fig. 2E, 2F). Importantly, given the known role of NFAT isoforms in T cell function, these defects emerged prior to the developmental emergence of functional T cells, which does not occur until approximately 6 days post fertilization as the thymus develops \citep{Trede2004}. However, whole animal knockouts could not address potential roles for cell types other than macrophages in mediating this process. 

\subsection{Macrophage-NFAT is Essential for Angiogenesis Induction \textit{in vivo}}

Given our observations on \textit{vegfaa} induction in macrophages at the granuloma, we tested whether NFAT signaling was required specifically in macrophages for granuloma-associated angiogenesis. For \textit{in vivo} inhibition of macrophage NFAT signaling during infection, we applied an approach that takes advantage of the NFAT-inhibitory peptide, VIVIT, which competitively inhibits calcineurin-dependent activation of all the NFATc isoforms \citep{Aramburu1999}. This approach has been successfully used as an exogenous treatment in cell culture \citep{Deerhake2021} (Deerhake et al., 2021) and mice \citep{Noguchi2004, Elloumi2012, Rojanathammanee2015}, through ectopic overexpression in cell culture \citep{McCullagh2004},  and, more recently, in mice \citep{Poli2022, Peuker2022}. We developed a transgenic zebrafish line in which VIVIT is expressed specifically in macrophages, Tg(\textit{irg1}:\textit{VIVIT-tdTomato\textsuperscript{xt38}}) (from here, simply \textit{irg1}:\textit{VIVIT}) (Fig. 4A, 4B) \citep{Sanderson2015}. We assessed whether the macrophage-specific expression of VIVIT would be sufficient to reduce the degree of angiogenesis during infection in the trunk with wild-type \textit{M. marinum} expressing mCerulean (\textit{Mm}-mCerulean). We found that macrophage-specific VIVIT expression significantly reduced angiogenesis in response to infection (Fig. 4C, Supp. Fig. 2G, 2H). This suggested a macrophage-specific role for NFAT signaling downstream of mycobacterial detection that was necessary to induce angiogenesis, presumably through the \textit{nfatc2a} isoform.

To ask more directly whether the decreased angiogenesis observed in the NFAT-deficient macrophages was via the TDM-mediated pathway, we used the TDM injection assay we had developed previously. We injected TDM or the IFA vehicle  into the trunk of 2 dpf larval zebrafish (Fig. 4D) and measured the resulting angiogenesis at 2 dpi \citep{Walton2018}. TDM was sufficient to induce angiogenesis in vivo and this effect was dependent upon functional NFAT signaling, with the degree of TDM-induced angiogenesis reduced to the level of the vehicle alone in \textit{irg1}:\textit{tdTomato} animals compared to \textit{irg1}:\textit{tdTomato} controls (Fig. 4E, Supp. Fig. 2I, 2J).

\subsection{NFAT Activation is Essential for Angiogenesis in Adult Granulomas}

Adult zebrafish are equipped with both innate and adaptive immunity and form mycobacterial granulomas that histologically mirror epithelioid human tuberculosis granulomas \citep{Swaim2006}, including induction of a surrounding vascular network. To assess whether our findings in the larvae translated to a longer-term context in the presence of adaptive immunity, we infected adult \textit{kdrl}:\textit{eGFP}; nfatc2a\textsuperscript{xt69/xt69} zebrafish and \textit{kdrl}:\textit{eGFP}; nfatc2a\textsuperscript{+/+} siblings with Mm-tdTomato and examined their peritoneal organs at 18 dpi after CLARITY-based clearing \citep{Chung2013, Cronan2015}. Cleared organs were then imaged by spinning disk confocal microscopy (Fig. 5A). We measured the total vascular network surrounding the granulomas in a programmatically blinded fashion \citep{Salter2016} and \autoref{chap4} and found that nfatc2a\textsuperscript{xt69/xt69} fish had a significant reduction (~50\%) in the length of the vascular network compared to wild-type siblings, further validating this gene as important for the angiogenic response in vivo (Fig. 5B, 5C; Supp. Fig. 3A, 3B). These putatively neovascular vessels tend to be highly branched and to be comprised of a limited number of cells with small or non-existent luminal volume, indicating that they are still in the sprouting stage of angiogenesis and suggesting a potential failure to mature. We observed robust effects that are likely understated in our quantitation, as we could not formally make any distinction between thicker, existing vasculature present at baseline that happens to fall nearby the granuloma and the characteristic neovascularization more intimately associated with the granuloma and present in wild-type but reduced in \textit{nfatc2a} mutants (Fig. 5A; Supp. Fig. 3C).

\subsection{\mbox{Macrophage-specific} NFAT Inhibition in Mature Granulomas Reduces Angiogenesis}

We next evaluated whether macrophage-specific NFAT inhibition had similar effects on vascularization in adult zebrafish. We infected adult \textit{irg1}:\textit{tdTomato}; \textit{kdrl}:\textit{eGFP} and \textit{irg1}:\textit{VIVIT-tdTomato}; \textit{kdrl}:\textit{eGFP} double transgenic zebrafish with Mm-mCerulean and examined visceral organs at 14 dpi. We used confocal imaging to visualize individual CLARITY-cleared organs and measured the total length of granuloma-proximal vasculature under blinding as above (Salter, 2016). We found that the degree of vascularization was significantly reduced around granulomas from \textit{irg1}:\textit{VIVIT-tdTomato} fish as compared to \textit{irg1}:\textit{tdTomato} fish (Fig. 5D, 5E, Supp. Fig. 3D, 3E). The extent of the vascular network in the \textit{irg1}:\textit{VIVIT-tdTomato} condition was notably restricted in cases or solely comprised of more mature, luminal vessels, suggesting a total failure to induce an angiogenic response (Fig. 5D). These findings, consistent with our previous data from both larval zebrafish infections in the \textit{irg1}:\textit{VIVIT-tdTomato} background and in the \textit{nfatc2a} mutant adult fish, point to a critical role for macrophage-specific NFAT activation in inducing the angiogenic response at mycobacterial granulomas. Furthermore, this establishes that NFAT function is broadly conserved from early larval infection through to the mature necrotic granulomas that characterize adult infection.

\subsection{Inhibition of NFAT Signaling Results in Decreased Bacterial Burden}

We had previously shown that inhibition of granuloma-associated vascularization is associated with decreased bacterial burden. Mycobacterial mutants unable to induce vascularization ($\upDelta$pcaA), and either genetic or pharmacological inhibition of VEGF signaling all result in lower bacterial burden, presumably due to functions of the aberrant vasculature promoting bacterial growth and/or inhibiting bacterial killing \citep{Rao2005, Glickman2000, Oehlers2015, Walton2018}. To examine the effect on burden of inhibition of NFAT signaling, we performed colony forming unit (CFU) assays at timepoints after the induction of angiogenesis and granuloma maturation. We infected \textit{nfatc2a\textsuperscript{+/+}} and \textit{nfatc2a\textsuperscript{xt69/xt69}} adult zebrafish with \textit{Mm}-tdTomato and plated them for CFU at 24 dpi. We found that knockout of \textit{nfatc2a} resulted in a ~50\% decrease in median colony number compared to wild-type after extended infection (Fig. 5F). 

Finally, we evaluated the impact of macrophage-specific NFAT inhibition on whole organism bacterial burden. We infected adult zebrafish possessing either the \textit{irg1}:\textit{tdTomato} or \textit{irg1}:\textit{tdTomato} transgenes with Mm-tdTomato and then homogenized and plated these fish at 18 dpi. We found that macrophage expression of the VIVIT peptide resulted in a median reduction of ~60\% of the bacterial burden in these fish at this time point relatively soon after the formation of necrotic granulomas and robust induction of angiogenesis (Fig. 5G).
 
\subsection{Pharmacological Inhibition of NFAT in Human \mbox{THP-1} Macrophages Limits VEGFA Induction by \textit{Mycobacterium tuberculosis}}

The zebrafish mycobacterial infection model shares important conserved features with \textit{M. tuberculosis} infection of humans, host response and granuloma angiogenesis \citep{Swaim2006, Datta2015, Oehlers2015, Cronan2021}. In addition, important aspects of the response to cyclopropanated TDM appears to be largely maintained between zebrafish and humans \citep{Walton2018}. We next asked whether our findings discovered \textit{in vivo} with the zebrafish-\textit{M. marinum} model were conserved in human cells exposed to \textit{M. tuberculosis}. We developed a cell culture model of macrophage-\textit{M. tuberculosis} interactions using differentiated THP-1 monocytic cells exposed to $\upgamma$-irradiated \textit{Mycobacterium tuberculosis} H37Rv ($\upgamma$\textit{Mtb}), which produces the full spectrum of TDM species, presented to the cell in their native configuration (as compared to heat-killed \textit{M. tuberculosis}, which disrupts cell envelope structure and organization) \citep{Romero2014, SecanellaFandos2014} (Fig. 6A). We found that exposure of differentiated THP-1 macrophages to $\upgamma$\textit{Mtb} was sufficient to induce VEGFA transcription as well as VEGFA secretion (Fig. 6B, 6C). To examine whether NFAT signaling is required for production and secretion of VEGFA we treated THP-1 macrophages with the small molecule inhibitor INCA-6, which specifically disrupts the interaction between the NFAT family members and their activating phosphatase, calcineurin \citep{Roehrl2004}. Strikingly, treatment of THP1 cells with INCA-6 during $\upgamma$\textit{Mtb} exposure significantly inhibited transcriptional induction of \textit{VEGFA} (Fig. 6B, Supp. Fig. 4A, 4B), as well as VEGFA secretion (Fig. 6C, Supp. Fig. 4C, 4D). Immunofluorescence revealed robust translocation of NFAT (using an NFATC2 antibody) that was broadly correlated to VEGFA signal (Fig. 6D). Taken together these experiments suggest that human NFAT signaling is required for VEGF production in response to \textit{M. tuberculosis} exposure.

\subsection{Requirement of human NFATC2 for VEGFA induction}

To identify functionally important NFAT human isoforms, we exposed THP-1 macrophages to $\upgamma$\textit{Mtb} and subsequently used the secretion inhibitor brefeldin A to lock VEGFA within secreting cells. Simultaneous staining for each of the four human NFATc proteins along with VEGFA allowed us to identify NFAT isoforms that underwent changes in expression and localization and correlate this with VEGFA production (Fig. 7A). While THP-1 macrophages express all of the isoforms to varying degrees, the most intense co-staining with VEGF was found with NFATC2 (Fig. 7B). Additionally, while each of the isoforms showed alterations after $\upgamma$\textit{Mtb} exposure, only NFATC2 showed robust nuclear localization that appeared to correspond to VEGFA induction in individual cells (Supp. Fig. 4E). While some NFAT isoform translocation was observable with at least NFATC1 and NFATC3, this generally had no correspondence to the degree or presence of VEGFA production. Given the strong correlation for NFATC2 with nuclear localization and VEGFA production after $\upgamma$\textit{Mtb} exposure, expression data from zebrafish and non-human primate granulomas, as well as the \textit{in vivo} zebrafish results implicating macrophage \textit{nfatc2a} in \textit{vegfaa} production and angiogenesis, we focused on human NFATC2 as a key isoform.

To test a functional role for human NFATC2 in macrophage induction of VEGFA during $\upgamma$\textit{Mtb} exposure, we used a lentivirus-mediated CRISPR/Cas9 approach to introduce high-efficiency disruption of NFATC2. We compared these cells to those transduced with lentiviruses expressing safe-targeting control sgRNAs. (Supp. Fig. 4F-4H) \citep{Kabadi2014, Sanjana2014, Morgens2017, Kitamura2021}. Using techniques inspired by the zebrafish and backported to cell culture, we simultaneously expressed four distinct guide RNAs targeting NFATC2 or safe-targeting controls, to maximize the percentage of puromycin-resistant cells possessing complete null mutations \citep{Wu2018}. Due to technical challenges associated with long-term culture of THP-1 cells and to address heterogeneity among cellular responses, we focused these assays on VEGFA induction in these cells by immunofluorescence after $\upgamma$\textit{Mtb} exposure. Because the N-terminal epitope recognized by our NFATC2 antibody was upstream of the targeted sites, we were unable to examine functional protein levels directly and simultaneously in the immunofluorescence images (Supp. Fig. 4I). However, we found that transduced cells targeted by NFATC2 lentivirus generally failed to induce VEGFA while safe-targeting control lentivirus-transduced cells responded normally (Fig. 7D, 7E). Thus, macrophage NFATC2-mediated induction of VEGFA downstream of mycobacterial TDM exposure is conserved from zebrafish to human cells exposed to \textit{M. tuberculosis}.

\section{Discussion}

This work uncovers an unexpected role for macrophage NFAT activation in immune responses to pathogenic mycobacteria and the maladaptive angiogenic responses that occur during infection. This activation of NFAT is driven through recognition of bacterial  cyclopropanated trehalose 6-6'-dimycolate, a major constituent of the cell envelope in pathogenic mycobacteria, that we have previously found is necessary and sufficient to drive pathological angiogenesis (Walton et al., 2018). Identifying this unexpected role for NFAT in angiogenesis expands our understanding of the mechanisms governing mycobacterial pathogenesis and offers targets for potential host directed therapeutics. Traditionally, work on TDM-mediated C-type lectin activation has focused on CARD9 and NF-$\upkappa$B signaling. Here, in contrast, we describe a specific role for alternative C-type lectin signaling responses through the NFAT pathway to drive VEGFA production and granuloma-associated angiogenesis. 

VEGFA induction is a prominent feature of tuberculosis in human disease as well as in a number of animal models, including non-human primates, rabbits, mice, and zebrafish \citep{Datta2015, Oehlers2015, Polena2016, Harding2019, Cronan2021, Gideon2022}. We found that VEGF was produced specifically within newly arrived macrophages at nascent granulomas. Macrophage populations are critical to VEGF induction, as macrophage-specific inhibition of NFAT signaling as well as deletion of \textit{nfatc2a} result in reductions in granuloma-associated angiogenesis. Using a human cell culture model, we found that NFATC2 was similarly engaged in human cells as amongst all NFAT isoforms, only NFATC2 underwent robust nuclear translocation in response to \textit{M. tuberculosis} stimulation. Correspondingly, pharmacological inhibition of NFAT signaling in human cell culture as well as genetic inhibition of \textit{NFATC2} resulted in reduced VEGFA production.

Although animal models of tuberculosis generally report high VEGF levels, there are few studies that center on VEGFA induction in cell culture infection models. Through high-resolution time-lapses and reporter lines, we found that \textit{vegfaa} induction generally does not occur until the formation of initial granulomas and generally correlated with the appearance of extracellular bacteria that could be recognized by incoming, likely uninfected macrophages. This concentration-dependent effect on signaling may reflect key aspects of the disease itself, wherein large masses of extracellular bacteria accumulate in the necrotic core of the granuloma, potentially triggering relatively insensitive and/or chronic C-type lectin signaling in this context.

Consistent with the recognition of extracellular bacteria, exposure of human macrophage-like cells to $\upgamma$-irradiated M. tuberculosis rapidly induced VEGF signaling in a NFATC2-dependent manner in a dose dependent manner. Standard cell culture infection models generally eliminate extracellular bacteria using gentamicin treatment and media changes, and so it is possible that engagement of this pathway by extracellular bacteria or TDM stimulation is a key component of this response. A survey of the literature and a variety \citep{Lee2019, Pisu2020, Hall2021, Looney2021, Pu2021} of RNA-seq datasets from macrophage-\textit{M. tuberculosis} infection experiments reveal modest or nonexistent induction of VEGFA, further supporting the notion that extracellular exposure to \textit{M. tuberculosis} may be an important element of the angiogenic response and may reflect some aspects of the macrophage-\textit{M. tuberculosis} interface within granulomas.

As its name suggests, the NFAT pathway plays an indispensable role in normal T cell biology. Accordingly, whole animal knockouts of NFAT in standard mouse models of \textit{M. tuberculosis} infection --€“ where granuloma formation itself may be limited --€“ may have obscured a role for myeloid-specific effects of NFAT signaling \citep{Via2012}. The zebrafish model, by looking at early timepoints, uncovered a role both in angiogenesis and, presumably as a consequence, bacterial control. Wholesale, longer-term inactivation of NFAT, which also plays important roles in T cells, would compromise important aspects of a productive adaptive immune response during mycobacterial infection. While genetically manipulable animal models allow for cell-specific separation, any host-based therapeutic approaches might require cell-specific macrophage delivery methods \citep{Hu2019, Mukhtar2020, Colombo2022}, NFATC2-specific targeting \citep{Kitamura2021}, and/or contend with the adaptive immune response\footnote{Although NFATC2 appears to be largely redundant with both NFATC1 and NFATC2 for many, but not all adaptive responses, it is still important to, as specifically as possible, target the desired pathway for minimal off-target side effects.}, an important aspect of host resistance during mycobacterial infection.

It remains unclear why NFATC2, but not any of the other isoforms, is specifically required in macrophages for the induction of VEGFA, given evidence that the others are present in resting macrophages (Fig. 7A). The functional distinctions between the isoforms have long been of basic interest, but relatively few specific differences between them have been identified beyond basal regulation to provide tissue-specificity and more recent findings describing layers of kinetic regulation with isoform-specific stimulation thresholds, nuclear retention, and more \citep{Lyakh1997, Rao1997, Kar2014, Kar2015, Kar2016, Yissachar2013}. These novel levels of regulation offer opportunities for uncovering new features of the cell biology of NFAT.

Here, we identify the unique requirement for this single isoform in macrophages to induce angiogenesis in response to mycobacterial infection. One hypothesis is that NFATC2 has binding partner(s) unique among NFAT isoforms required for its effect on the VEGFA promoter. Whether this is HIF-1$\upalpha$ (the canonical regulator of VEGFA) or one of the many previously described interacting partners is, as yet, unknown, but could be tested either in vitro or in vivo with genetic or chemical approaches. However, higher order regulatory mechanisms that result in the production of VEGF in the absence of overt hypoxia have been understudied and this work proposes at least one potentially generalizable mechanism whereby NFATC2 activation results in VEGFA transcriptional upregulation, a process that can be inhibited with chemical and genetic intervention. Despite the widespread presence of putative NFAT binding motifs (\seqsplit{5'€™-GGAAA-3'€™}) (Supp. Fig. 4J) in the proximal VEGFA promoter \citep{Gearing2019}, their influence on VEGFA transcription has been relatively unexplored as this specific effect is generally not seen in T cells or other cell types \citep{Chang2004}. NFAT involvement in the induction of a variety of cytokines is well-documented, but which, if any, are at play in the macrophage-\textit{M. tuberculosis} interaction is a promising subject for future research. 

A more comprehensive characterization of NFAT-dependent innate immune responses has begun in recent years \citep{Deerhake2021, Peuker2022, Poli2022} (see also \autoref{nouveaunfat}), but this pathway has remained unstudied in the context of macrophage signaling during mycobacterial infection. Furthermore, this work draws a connection between the induction of calcium fluctuations --€“ which can occur in response to many different developmental, homeostatic, and pathological stimuli, including to mycobacterial infection \citep{Kusner2001, Jayachandran2007, Jayachandran2008, Matty2019} --€“ to the angiogenic response to that stimulation. Our identification of NFAT regulation of VEGFA offers a novel approach to both pro- and anti-angiogenic intervention in various pathological contexts.

\subsection{Limitations of the Study}

While we have identified interesting macrophage biology mediating an important host immune response during mycobacterial infection, there is no data as to whether this might translate to other disease contexts, especially those with a prominent role for C-type lectin signaling. Whether or not this mechanism is broadly generalizable is important to understanding key aspects of pro-angiogenic macrophage behavior. Additionally, we have validated important aspects of our observations in the zebrafish with a mammalian cell culture model, but subsequent studies may warrant further integration of mammalian models of tuberculosis infection where angiogenesis is present or human patient samples to better understand certain aspects of the underlying biology.


% An example of making lists of various kinds - This one gives black circles of certain size based on style files - 
% LaTeX manual will tell you how to put numbers or different symobols

