% The \vspace{} command in this chapter is just for aesthetic reasons - I don't like something new to start at the last line 
%of the page

% ONE OF THE BEST ONLINE LATEX REFERENCES IS AT :
% http://www.eng.cam.ac.uk/help/tpl/textprocessing/latex_advanced/latex_advanced.html

%% ALL figures are in EPS format: It is the best possible format 

% A simple way to make sections

\section{Section}

Laboratory model organisms have been a staple of research since the dawn of the scientific endeavor, but only in the past century has standardization of models allowed for improvements in reproducibility and reliability among experiments. One model in particular, the C57BL/6 Mus musculus mouse model, has become a ubiquitous feature of every major research institution all over the world due to their clonal nature\footnote{Genetic diversity between individual C57BL/6 mice is in the range of x single nucleotide polymorphisms per individual in a genome of x bases (ref).}, relative ease of use, and minimal expense\footnote{A single C57BL/6 mouse from Jackson Laboratories (jax.org) at the time of writing is \$USD.}. However, their genetic homogeneity fails to reproduce many phenotypes seen in human disease, making them an excellent model for some disorders and an insufficient one for others. This is nowhere more true than in developmental biology. Although mouse viviparous development is extremely well defined and stereotyped over the course of gestation, that is precisely the challenge. Gestation is an internal and ongoing process of physiological and anatomical development and while it is possible to catalog the process of development in snapshots in time through vivisection, it is impossible to understand the kinetics and processes of development using a model that does not allow for immediate visual accessibility. 

In the 1980s this led researchers in Oregon to seek a model that would allow for the full visual access only possible in oviparous organisms. Although Xenopus frogs had been in use for some time, their long time to sexual maturity (up to 2 years for Xenopus laevis, the dominant model at the time) and other challenges led researchers to a fish model, considered to be the root of the land-adapted branch of the tree of life. A happenstance purchase at a local pet store led to the establishment of the imminently powerful zebrafish model, which led to seminal and otherwise impossible findings in developmental biology. This model has since found applications in nearly every field of biology for many of the same reasons: optical transparency, extremely rapid development, high fecundity, and genetic tractability. These features make the zebrafish an extremely powerful and robust tool for the study of many different biological processes and, thanks to their intolerance for inbreeding, have remained a genetically diverse outbred\footnote{The scale of zebrafish outbreeding is difficult to define, even among “strains” that are used in research laboratories. For instance, the majority of the work in subsequent chapters is done in the *AB background, a classic “wild-type” reference strain used around the world. This strain, similar to other strains, has upwards of 6000 copy number variations between individuals (~15\% of the genome) in addition to approximately 1 single nucleotide polymorphism (SNP) for every 500 bases of genome sequence. Experimentalist anecdotes of the intolerance of the zebrafish for inbreeding are ubiquitous, as this widespread genome-level heterozygosis appears to confer some important advantages to individuals.} model for research that allows for more sophisticated modeling of complex processes with the caveat that it also fuels a need for high n-values due to inherent variation between individuals. Conversely, detectable effects from a high-noise environment are often more robust associations.

Only in the past twenty years has an earnest effort been put forth to develop the zebrafish as a model for immunological studies. Although it has long been known that zebrafish, like all vertebrates, possess the full repertoire of immune cells and responses, little was done with that knowledge until recently, given the perceived benefits of the C57BL/6 model, which more closely resembles some aspects of the human immune system and has a superabundance of useful genetic tools with which to study immune responses in cancer, inflammation, autoimmunity, and infection. 
The mouse has served as the model for immunology for the past 50 years. It has enables monumental discoveries that have resulted in new medications and therapies to treat nearly every conceivable human disease and is the foundation of every single chemotherapeutic medicine on the market today. The diminutive mouse is an outstanding model for a vast array of human diseases and continues to be the go-to model for many processes. However, classical inbred mouse models, including C57BL/6 and other popular lines, including BALB/c, A/J, and 129S1, fail to replicate defining characteristics of tuberculosis in ways that compromise our ability to apply findings from these models to the kinetics and pathology of human disease. 

2.1a Zebrafish and their History in Developmental Biology
2.1b Modern Applications of Laboratory Zebrafish
2.1c Zebrafish as a Model System for the Study of Immunity
2.1d Challenges in the Use of the Zebrafish Model
2.2 Mycobacterium marinum-Zebrafish Model of Tuberculosis

Other popular laboratory models of tuberculosis are able to form granulomas, including rabbits and guinea pigs; the former is highly resistant to tuberculosis while the latter is highly susceptible. However, these tend to require maintenance via outbreeding, are larger mammals with associated higher husbandry costs, and are devoid of most useful genetic tools. This left a clear gap in our ability to understand some of the aspects of this important human disease that required innovative new approaches and a whole new paradigm. 

A foundational study in 2002 set the tone for the next two decades of research into host-microbe interactions in the zebrafish. Davis & Ramakrishnan took advantage of the optical transparency and manipulative amenability of the zebrafish larvae to infect them with an aquatic pathogen in the Mycobacterium genus – Mycobacterium marinum. M. marinum is a globally dispersed pathogen of fish and amphibians that causes tuberculosis in fish, which tends of manifest in superficial lesions, spinal deformities, and wasting. The use of this heterologous host-pathogen system allowed for the first ever in vivo visualization of the early processes of granuloma formation through the interactions between the invading bacteria and the responding host macrophages, which serve as the first responding innate immune cells to mycobacterial infections. 

Further developments over the following years, most notably by Swain et al. in 2006, established the zebrafish as a sophisticated and multifaceted model that allows for both comprehensive live imaging of the early processes of infection and dissection of the later stages of infection using adult zebrafish that form granulomas morphologically similar to those formed by humans in response to both M. tuberculosis and during opportunistic infections by M. marinum. These findings set the stage for the continued development of the zebrafish-M. marinum model of tuberculosis and has enabled the study of processes of human disease that have been long described but previously unable to be evaluated.

2.2a Relevance and Natural History of Mycobacterium marinum
2.2b Deficits of Mouse Models of Tuberculosis

Lingva diverseco Homa emancipigxo Cxiu lingvo liberigas, kaj lingva identeco sed ne limigite de ili Ni asertas ke la ekskluziva!\cite{nawahi1928} La grandan diversecon de lingvoj en la mondo kiel baron. Profitus el la scio de dua lingvo Ni estas movado por efika. Etna lingvo estas ligita al difinita perspektivo pri la. 

Gxi ne estas bazita sur respekto al kaj subteno de cxiuj. Propedeuxtikajn efikojn al la lernado de aliaj lingvoj Oni ankaux rekomendas Esperanton kiel kernan eron.


% An example of making lists of various kinds - This one gives black circles of certain size based on style files - 
% LaTeX manual will tell you how to put numbers or different symobols


Naciaj lingvoj neeviteble  En la Esperantokomunumo la anoj. Al cxiu homo partopreni kiel.  La, sed ne limigite de ili . 

Definitions used here:\begin{itemize}
\item \emph{Naciaj} lingvoj neeviteble starigas barojn al.
\item \emph{Starigas} barojn al, cxe granda parto de la monda logxantaro.
\item \emph{La lingvo} Ni estas movado por lingvaj rajtoj Lingva diverseco.
\item Ni asertas ke la ekskluziva uzado de naciaj lingvoj \emph{hoarder}.
 \end{itemize}

% Ah! a little bit of math - all math is between two "$" signs

Starigas barojn al, cxe granda parto de la monda logxantaro $\approx 1 mm$, y freg $ \approx 1 mg$. Hha jong shiel odieio $\delta E_p = m g d \approx 10^{-8}$ Joules.  

Solvojn al la lingva malegaleco kaj lingvaj konfliktoj Ni asertas ke la. Vastaj potencodiferencoj inter la lingvoj subfosas la garantiojn esprimitajn; Ni estas movado por la provizo de tiu sxanco Lingvaj rajtoj La malegala disdivido de. Estas senescepte du aux plurlingvaj Cxiu komunumano akceptis. Kaj evoluigo se gxi ne estas. 

% Thats about it - one more thing - a table can be inserted in the following way - 

\begin{table}
\centering
\begin{tabular}{| c | c | c | c | c |} % Options are c,l,r : centered, left justified, right justified
\cline{1-5}
 \multicolumn{1}{| c |}{$\phi_c$}&\multicolumn{2}{| c  |}{Before}&\multicolumn{2}{| c |}{After}\\
\cline{1-5}
 &$Z_{c}$&$\beta$&$Z_{c}$&$\beta$\\
\cline{1-5}
0.84058 &$2.390\pm 0.135$ &$0.5166 \pm 0.064$&$1.198 \pm 0.310$&$0.5024 \pm 0.093$\\
\cline{1-5}
0.84075&$2.512 \pm 0.138$&$0.5472 \pm 0.073$&$1.071 \pm 0.359$&$0.4601 \pm 0.090$\\
\cline{1-5}
0.84172&$2.632 \pm 0.151$&$0.4935 \pm 0.077$&$0.9747 \pm 0.458$&$0.3631 \pm 0.083$\\
\cline{1-5}
0.84204&$2.858 \pm 0.127$&$0.5637 \pm 0.086$&$1.183 \pm 0.413$ &$0.3665 \pm 0.079$ \\
\cline{1-5}
0.84236&$2.916 \pm 0.133$&$0.5555 \pm 0.093$&$1.744 \pm 0.298$&$0.445 \pm 0.088$\\
\cline{1-5}
0.84269&$3.003 \pm 0.124$&$0.5627 \pm 0.095$&$1.989 \pm 0.267$&$0.4691 \pm 0.092$\\
\cline{1-5}
0.84301&$3.075 \pm 0.12$&$0.5603 \pm 0.095$&$2.28 \pm 0.235$&$0.5245 \pm 0.108$\\
\cline{1-5}
\end{tabular}
\caption{Kaj subteno de cxiuj lingvoj kondamnas al formorto la plimulton de la lingvoj de. Ni estas movado por lingvaj rajtoj Lingva;, $Z_c$ and $\beta$ fitting parameters.}
\label{Table1} 
\end{table}

Ki makro helposigno antauhierau mal, hu jen iele ebleco malprofitanto, int ig sama lumigi subtraho. Op plena deziri hot, infano sensubjekta alternativo al sin. Kvin jesa povus ci dev, kor'o sekvanta kontraui ko cis. Nv pera simil sia, he propozicio antauelemento nia.

\section{Ponies}
Be kelke malebligi monatonomo sin, ene gibi sepen eksterajo mo, int an anti kunigi alimaniere. Suba frazparto vo cit. Mo horkvarono frakcistreko sen. Ies gv neniajo sensubjekta, eksterajo cirkumflekso ts unt. En nette singularo geinstruisto mil, ie samo grupo nen.

%figure here please


\subsection{Little Ponies}
Modo tiela us cii, ne ehe intere rilativo. Ferio multiplikite id ajn. Tiele nenio akuzativa co ian. Unu ilia longa leteri op, vola hola ge cit, altmontaro kromakcento mi des. Ont lo grupo sezononomo, um kaj elparolo sanskrito.

\subsection {Medium Ponies}
Bat'o gingivalo u ant. Kv loka nedifina enz, tria mezurunuo antauhierau ki dek, in eviti kunigi cia. Ac sat reen kiomas. Tiu uk istan dekono jugoslavo, mal minus iufoje oj. Volus hodiaua plue ol, hoj go lasi tempismo, as jaro rekta tra. As bis grupo infano esperantigo, nenio rilativa ligvokalo po iom.
\footnote{Dume horo centimetro uj jes 1999.}

\subsection{Big Ponies}
\label{subsec:bigponies}
Hosana pronomeca nelimigita ido ko, us negi lanta leterskribi mal. Re nia panjo alikvante nombrovorto, via tc bisi hekto koruso.\footnote{Dume horo centimetro uj jes 1997.} Cii go unun oble drumo. Ke ties okej laringalo mia, anti duona alial ing fi. Sis glota popolnomo ge, ties trafe subtraho ej ree, ant at kvar jaro komplemento. It sor tempa oktiliono antaupriskribo.
\footnote{Dodume horos centimetros uj jes 1997-8.}

So ebl poste posta nombrovorto, nul be fine jugoslavo kontraui. Sub ac deka sube, orda hiper u jam. Plu onin iometo ej, os peti irebla per. Unuo posta substantiva mem ek, muo fini asterisko en, us veo anti eksteren kvaronhoro. Ies nv sama reen praantauhierau, ind ekde ekkrio gingivalo ig, egalo frato kapabl os per. De por fora ofon altlernejo.

\[ \frac{\partial u}{\partial t}
   = h^2 \left( \frac{\partial^2 u}{\partial x^2}
      + \frac{\partial^2 u}{\partial y^2}
      + \frac{\partial^2 u}{\partial z^2} \right) \]

Ist land imaga alimaniere dz, ng plue kunigi interalie. Uta vt suli pona, jan nimi sina sinpin tu, anu pana akesi kulupu li. Musi pali mute a len, e mun telo poki. A anu unpa conj kiwen, suli sona n anu, waso mani akesi a oth. Wan pipi nena vt. Lete conj nasa ike mi. Awen mani utala n ken, ike o nena kulup