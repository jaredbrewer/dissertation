\documentclass[PhD]{dukethesis2006}

%preamble here for options

%-----------------------------------------------------------------------------%
% DEFINITIONS:
%
% include \usepackage here
%-----------------------------------------------------------------------------%
\usepackage{amsmath}
\usepackage{amssymb}
\usepackage{amsthm}
\usepackage{array}
\usepackage{epsfig}
\usepackage{graphicx}
\usepackage{xy}
\usepackage{tabularx}
\usepackage{parskip}
\usepackage{setspace}
\usepackage{textgreek}
\usepackage{listings}

\input{bob_macros.tex} 
% This is where the shortened versions of Latex environments like figure, equation, etc 
% are defined. See the file for the shortcuts.

\author{William Jared Brewer}
\advisor{David M. Tobin}
\member{J. Andrew Alspaugh}
\member{John F. Rawls}
\member{Christopher Kontos}
\member{Dennis Ko}

\department{Molecular Genetics and Microbiology}
%\subject{xxx} If this is used, "subject" has to be un-commented in the cls file in several places
\title{Cellular Signaling Mechanisms Underlying the Angiogenic Response to Mycobacterial Infection}

%end of preamble, beginning of printable document

\begin{document}

{\parskip=0pt \maketitle}

{\parskip=0pt \makeabstract}

\Copyright

\begin{doublespace}

\abstract

Pathological angiogenesis is a widespread phenomenon that influences the progression of a variety of diseases, including autoimmune conditions, cancers, and microbial infections. One infection in particular, tuberculosis, induces a potent pro-angiogenic signaling cascade that increases bacterial burden and disease progression, but many of the underlying mechanisms remain unknown. Here, I have delineated a discrete host signaling pathway within responding macrophages that first detects a particular glycolipid on the surface of the bacteria, transduces an intracellular signaling cascade, and drives production of the master regulatory angiogenic chemokine, VEGFA. This signaling pathway is driven by activation of nuclear factor of activated T cells, cytoplasmic 2 (NFATC2) downstream of trehalose 6-6�-dimycolate (TDM) detection. Characterization of this pathway resolves a major unknown factor in the signaling mechanisms underlying this maladaptive host response and may offer opportunities for host-directed therapeutic intervention in mycobacterial infections as well as being potentially generalizable to other disease contexts.

\dedication

For my daily motivation and inspiration; for the person who taught me to read and write, who always believed I could do anything I set my mind to, and who dreamt of this day - my Mamaw Barb.

\end{doublespace}

\listoftables

\listoffigures

\begin{doublespace}

\acknowledgements

The number of people who warrant thanks exceeds my capacity to thank them and the space allotted. Suffice it to say that this entire journey would not have been possible without the multifold support systems that got me to where I am today.

First and foremost, my advisor, David Tobin, who was a constant and unconditional source of support and encouragement through the long experimentation that was required for much of the work presented in these coming pages. He provided me the academic and scientific freedom to find my path in science and to fail (repeatedly) with the unwavering faith that I would succeed.

In my darkest days of graduate school, Andy Alspaugh was the shining presence I needed to realize that I could go on and succeed and he has been there every step of the way, encouraging me to pursue my dreams and giving me support in the way that only he is capable. His kindness is an inspiration to all.

Lastly, I want to thank my loving partner, Kristen, who has believed in me and supported me throughout the past 4 years. As is often said of the long-suffering partner, she has tolerated the days, nights, and weeks that I was completely absent and, when I finally had a break, welcomed me home with no (visible) resentment and for that I will be forever thankful.

%A table of contents file is automatically generated in the same folder as the .tex file when
%the \tableofcontents is used

\end{doublespace}

\tableofcontents

%-----------------------------------------------------------------------------%
% replace FILE in \input{FILE} with name of tex file
% containing the given chapter, eg. for the introduction one could
% have FILE = intro if stored in intro.tex (.tex extension is assumed!).
%-----------------------------------------------------------------------------%

\chapter{Introduction}
\pagenumbering{arabic}

%A large document requires a lot of input. Rather than putting the whole input in a single large file, it's more efficient to split it into several smaller ones. Regardless of how many separate files you use, there is one that is the root file; it is the one whose name you type when you run LaTeX. In this template, the introduction chapter is an external file (intro.tex) and the second chapter is contained internally in the root file (in this file you are reading)

\begin{doublespace}
% The \vspace{} command in this chapter is just for aesthetic reasons - I don't like something new to start at the last line 
%of the page

% ONE OF THE BEST ONLINE LATEX REFERENCES IS AT :
% http://www.eng.cam.ac.uk/help/tpl/textprocessing/latex_advanced/latex_advanced.html

%% ALL figures are in EPS format: It is the best possible format 


% The \vspace{} command in this chapter is just for aesthetic reasons - I don't like something new to start at the last line 
%of the page

% ONE OF THE BEST ONLINE LATEX REFERENCES IS AT :
% http://www.eng.cam.ac.uk/help/tpl/textprocessing/latex_advanced/latex_advanced.html

%% ALL figures are in EPS format: It is the best possible format 

\section{Tuberculosis}

Of all the infectious agents to have ever afflicted humankind, \textit{Mycobacterium tuberculosis} is perhaps the most imminently successful. The primary cause of potentially greater than one billion human deaths since 1800 alone (approximately 9\% of all deaths in that time period) (citation), this disease has had profound impact on the cultural and political development of the modern world and continues to impact the lives of most people around the world today\footnote{For additional reading on this subject of how tuberculosis has impacted the development of human society, see (citation).}. Fallaciously considered a disease of antiquity, this disease manifests in active disease in greater than 10 million people each year and has killed greater than one million people per year each year since records or estimates have been available with the case and death burden rising due to health system neglect exposed by the COVID-19 pandemic ongoing at the time of this writing (citation). 

\subsection{History of Tuberculosis}

The overwhelming prevalence of tuberculosis in the 18th and 19th centuries led to an extreme degree of cultural salience for this disease in the daily lives of the people of those times. Responsible for the deaths of many preeminent public figures of these eras\footnote{The number of such public figures is far too great to list. From the 1840s and 1850s alone, tuberculosis was responsible for the deaths of Andrew Jackson (seventh president of the United States), Henry Clay (Secretary of State, Speaker of the House, three-time presidential candidate for the Whig Party), John C. Calhoun (Vice President, Secretary of State), Alexis de Tocqueville (famed French observer of American culture and author of the classic of political theory, Democracy in America), Henry David Thoreau (naturalist author of Walden), and Emily Bront\"{e} (author of Wuthering Heights).}, it is also a ubiquitous feature of the literature of those times as well. Perhaps most famously, tuberculosis is depicted as the disease that afflicts the Lowood School in Charlotte Bront��\"{e}'s Jane Eyre, among other novels depicting the disease then known as \textit{��consumption��} for the way in which it leads to cachexia, increasing pallor, hemoptysis, and ultimately death (citation).

This cachexia is a defining feature of tuberculosis across phylogenies; such progressive wasting unable to be ameliorated by improved nutrition is an unusual presentation strongly reminiscent of many cancers and rather dissimilar from most infectious diseases (citation). Indeed, as medical understanding of diseases progressed beyond concepts of humoral imbalance, a prevailing theory was that tuberculosis was a hereditary form of cancer due to the way it spread within families (citation). The functional and consequential similarities between tuberculosis and cancer are replete and will be a subject returned to throughout this document.

First documented in 1888 by Robert Koch, the tubercle bacillus, Mycobacterium tuberculosis, was a foundational instrument in the broader development of the field of microbiology and remains a major area of research today\footnote{For more on this, see (citation).} (citation). Once thought to have been banished to the annals of history, tuberculosis, after a steady decline in cases throughout the middle of the 20th century\footnote{This was coincident with, but likely unrelated to, the development of effective antibiotic therapies. Indeed, the modern disparity between tuberculosis rates between the United States and Western Europe and much of the rest of the world is thought to have more to do with improved living conditions, growing herd immunity, and improved nutrition rather than the use of antibiotics as downward trends actually began 100 years prior to the discovery of streptomycin in 1944 (citation).}, came roaring back in the late 20th century with the introduction of HIV into the human population in the 1980s and the corresponding increase in susceptibility to infection, disease, and death from tuberculosis due to the immunocompromising nature of HIV/AIDS (citation). 

\subsection{Pathogenic Features of \textit{Mycobacterium tuberculosis}}

In addition to the clear relevance of the study of tuberculosis to human health, the unique biological features of this acid-fast, non-motile, slow-growing mycobacterial species make is a fertile ground for basic scientific studies into the way that both saprophytic and pathogenic species of bacteria adapt to adverse environments and ultimately establish a productive niche (citation). The physiological features of the bacillus -- a thick, hydrophobic cell wall, unique export and import systems, and novel mechanisms for cell division and stress tolerance -- make this a fascinating case study in the evolutionary processes that drive niche adaptation and, indeed, niche creation (citation). That related members of the same genus of bacteria occupy such diverse infectious niches across a wide spectrum of organisms (from fish and amphibians to birds and mammals and in every major organ system) as well as possessing stages of growth in the environment is a testament to the flexibility and adaptability of many mycobacterial species. By contrast, some species, notably \textit{M. tuberculosis} and \textit{M. leprae} are profoundly adapted to a limited range of hosts and have lost the capacity for long-term survival outside of a mammalian host. This diversity within the genus offers abundant opportunity for gene-structure-function discovery to uncover factors both required for maintaining an environmental niche as well as those specifically required for either commensal or pathogenic association with hosts, an approach that has long been fruitful in the discovery of novel virulence factors (citations) but comparatively neglected in the basic bacteriological study of environmental mycobacteria.

\subsection{Treatments for Tuberculosis and their Mechanisms of Action}

Mycobacterial infections are uniquely integrative biological phenomena that require a careful balance between both the host and the bacteria. The host, seeking to eradicate the bacteria, needs a potent but highly specific immune response capable of sterilizing the invading bacilli while the bacteria, seeking to establish a replicative niche, must evade these host defenses. Historically, treatment for bacterial infections has been through the application of bacteria-targeting antibiotics, despite their mechanism of action rarely being understood at the time of clinical introduction. Modern tuberculosis infections are treated with a four-drug cocktail of antibiotics over the course of six to nine months: isoniazid, ethambutol, pyrazinamide, and rifampin (citation). Should the bacteria exhibit resistance to one or more of these, a state known as multi- or extensive-drug resistance (MDR/XDR), additional drugs with further host toxicity are used: kanamycin, ciprofloxacin, and cycloserine are common choices, although new drugs are slowly coming onto the market (citation). Of these, bedaquiline appears to have the most promise in improving the overall treatment of tuberculosis, but long-term impact remains to be seen (citation). 

The first modern, clinically effective treatment for tuberculosis was pioneered by the discovery of streptomycin from \textit{Streptomyces griseus} in 1944 (citation). Unlike its antibiotic predecessor, penicillin, streptomycin was effective in killing \textit{Mycobacterium tuberculosis} bacilli \textit{in vitro}. However, due to its lack of oral bioavailability, the use of this drug was limited to hospitals and clinics able to deliver the drug intravenously. Additionally, like many of the attempts at drug development for tuberculosis that had preceded it\footnote{One of these, para-aminosalicylic acid (PAS) is an interesting, if distracting tale in the history of microbiology. For more information, see (citation)}, it was not particularly effective at eliminating disease when used alone. Streptomycin is an aminoglycoside antibiotic that acts by interfering with protein biosynthesis by poisoning the 30S subunit of the ribosome as well as by inhibiting peptidoglycan biosynthesis through nucleophilic attack of the glycosidic bonds in peptidoglycan (citation). These mechanisms are common to all of the diverse bacteria against which streptomycin is effective, making it a good general purpose antibiotic, if somewhat limited in the face of the unique features of mycobacterial anatomy.

Thus, the introduction of a mycobacteria-specific antibiotic in the form of isoniazid in 1952 was a major breakthrough in the treatment of this disease. Orally bioavailable and highly effective at killing mycobacteria, it comes with the dose limiting side effects of peripheral neuropathy and occasionally fatal hepatitis that make it a less than perfect therapeutic option (citation). It is still in use today on account of its synergy with other antimycobacterials and independent efficacy. Isoniazid works by targeting mycobacterial cell wall synthesis and targets InhA to block earlier stages of fatty acid biosynthesis. This prevents the synthesis of the mycolic acids that comprise the outermost layer of the cell wall and which are essential for mycobacterial survival and growth (citation). 

Recognizing the inherent limitations to isoniazid, additional drugs came into use over the next twenty years. Next on the list of drugs was ethambutol, which entered into use in 1961. Ethambutol, like isoniazid, targets the synthesis of the cell wall, this time by inhibiting the enzymatic ligation of arabinogalactan to the lower peptidoglycan layer and the outer mycolic acid layer, which destabilizes the cell wall and increases bacterial susceptibility to killing. Interestingly, the precise mechanism of action of ethambutol remains unknown despite over 60 years of extensive study (citation).

Rifampin (1965) was the next addition and has an entirely novel mechanism of action compared to the previous entrants. Targeting multiple simultaneous essential biological pathways is an excellent and repeatedly proven way of killing pathogens and preventing the emergence of resistance to all of them simultaneously (citation). Rifampin targets RpoB, the major subunit of the bacterial DNA-dependent RNA polymerase, which is essential for gene transcription. Although mutations have arisen that confer resistance to rifampin, these have particular fitness costs on the bacteria under conditions lacking antibiotic pressure. Rifampin has proven to be an excellent antimycobacterial drug with a comparatively favorable side effect profile compared to the other commonly used drugs.

To round out the four drug cocktail generally recommended for the first-line treatment of tuberculosis today, pyrazinamide (1972) is the most mechanistically interesting of the drugs commonly used to treat tuberculosis. It appears to work by diffusion into the acidic necrotic center of the granuloma where protons activate the prodrug and allow it to be enzymatically converted into pyrazinoic acid, the active antimicrobial. The low pH maintains the stoichiometry in favor of the protonated pyrazinoic acid form over the conjugate base pyrazinoate, facilitating diffusion into the cytosol of the bacteria. Despite the knowledge that has been ascertained about the precise conditions under which pyrazinamide is active, the mechanism of action remains under hot contention with a variety of different mechanisms proposed and the most recent -- that it inhibits the synthesis of the essential fatty acid and metabolic carrier coenzyme A --�� still under dispute (citation). Pyrazinamide, entirely by accident, takes advantage of the particular biological environment of the infecting bacteria to specifically target the pathogen. As a relatively innocuous prodrug that is activated at the precise site of infection, it is able to reduce some of the toxic effects that would be associated with direct use of pyrazinoic acid while concentrating active drug where bacteria are actively growing with passive diffusion moving additional prodrug into the granuloma to be activated and trapped inside the necrotic caseum (citation).

Modern antibiotic development generally has been stymied by a lack of incentive for the development of high research and development cost, low profit drugs. As new antibiotics are likely to be reserved for cases with extensive antibiotic resistance and are likely to be cost-prohibitive, few have been developed despite pressing need. One of the success stories is that of bedaquiline, which was first approved in 2012. The development of bedaquiline required ~\$500 million in public investment compared to ~\$100 million in investment by the profiting corporation, Janssen Biotech (citation). Bedaquiline is a potent and highly effective drug reserved for use in multidrug resistant (MDR) and extensively drug resistant (XDR) cases of tuberculosis and which acts to block ATP synthase and shuts down bacterial metabolism and directly leads to bacterial death (citation). 

\subsection{The Mycobacterial Cell Wall}

Given that inhibition of cell wall biosynthesis is a common and highly effective mechanism of action for many antimycobacterial drugs, this structure is of clear importance to the survival and pathogenic success of these bacteria. \textit{In vitro}, mycobacteria are unique microbes that grow in intricate serpentine cords along agar plates. These cords were among the first observations that helped to classify diverse mycobacterial species together and defining the ontogeny of these cords was of immense concern to early mycobacteriologists (citation). By the 1950s they had identified what they called the cord factor --�� an isolable molecule required for the cording effect seen in mycobacteria and, indeed, able to replicate key features of cording when isolated, even in the absence of bacteria. The chemical composition of this cord factor was determined and this allowed it to be given a name -- trehalose 6-6'-dimycolate or TDM. TDM features a trehalose head group and two long mycolic acid ester tails that can number up to C\textsubscript{100} in length, creating an incorrigibly hydrophobic molecule that forms an extremely thick amphipathic bilayer at the surface of the mycobacteria with the trehalose moieties facing the outside world and attached to the arabinogalactan layer below with a dimensionally thick\footnote{~40 nm in thickness, representing approximately 30\% of the total volume of the bacteria, if we take the size of a single bacillus as 0.2 \textmu m in depth and 2 \textmu m in length} interior of interleaved mycolic acid chains. TDM is the predominant mycolic acid species in this cell wall layer and has been studied since its discovery for the ways in which it contributes to mycobacterial fitness in a diverse range of environments.  

Mycobacteria do not fit into standard binary classifications of bacteria within the Gram staining system. While Gram-negative bacteria feature both an inner and outer phospholipid membrane, Gram-positive bacteria have only a single plasma membrane but are encased in a thick layer of peptidoglycan. Although evolutionarily derived from the Gram-positive bacterial phylum \textit{Actinomycetota}\footnote{This large phylum of bacteria includes incredible diversity and a number of other important human pathogens with varying degrees of relatedness to \textit{Mycobacterium}. A notable example is \textit{Corynebacterium diphtheriae}, the causative agent of diphtheria, which also produces mycolic acids, albeit shorter in length. The existence of a highly effective vaccination to diphtheria while no effective vaccine exists against tuberculosis is emblematic of the divergent strategies these species use to undermine their hosts \textit{C. diphtheriae} produces a classical toxin, diphtheria toxin, that is responsible for much of the pathology of disease while M. tuberculosis was thought to lack toxins until the discovery of the tuberculosis necrotizing toxin (TNT), although this is only selectively expressed and not thought to be absolutely essential for disease (citation).}, mycobacteria possess features of both Gram-postive and Gram-negative bacteria; they have a single phospholipid bilayer and a thick peptidoglycan layer, but also have an additional membrane�� comprised of mycolic acids which is occasionally referred to as the \textit{mycomembrane} (citation).

The mycomembrane and its primary constituent, TDM, have many well-defined roles in providing tolerance to environmental stress, detoxifying reactive oxygen species, providing dehydration resistance, and modulating host immune responses. TDM, for instance, is able to block a key step in phagosomal maturation, which would normally be able to kill the bacteria after uptake into phagocytic immune cells, including macrophages and neutrophils. The broad ability of TDM to mediate mycobacterial interactions with the environment is one of the critical dimensions of the evolution of mycobacteria and the ability to then utilize novel modifications on this same molecular framework to undermine host immune responses appears to have been essential for their transition to a pathogenic or commensalistic\footnote{This notion of commensal mycobacteria warrants a vast degree of additional study. Although the laboratory model of non-pathogenic mycobacteria, \textit{Mycobacterium smegmatis}, was isolated from syphilitic chancres and, later, smegma, very little is known about the niche of these commensal mycobacteria, how they maintain a neutral or neutral-positive relationship with their (often transient) hosts, and how their presence impacts host immunity to future encounters with pathogenic mycobacteria (citation).} lifestyle in association with eukaryotic hosts ranging from amoeba to fish to humans.

\subsection{Trehalose 6-6'-Dimycolate (TDM)}

TDM, in many ways, defines the lifestyle of mycobacteria. As mentioned previously, this remarkably hydrophobic (indeed, wax-like) structure provides bacteria a potent tool in surviving both harsh environmental conditions but also the conditions likely to be encountered in a host. This structure has been thoroughly dissected over the past decades of research, and a range of modifications are known that influence both the biochemical properties of the cell wall, but also the ways in which host organisms response to this structure. 

Along the length of the mycolic acid tails, there are four main classes of modifications that may be present in two major locations. These modifications include methoxy, methyl, keto, and cyclopropyl groups, which can be located at either proximal or distal locations. Of these, the most research interest has centered on the very unusual cyclopropane modifications, which add a great deal of energetic ring strain to the molecule and is, generally, an unusual biological modification due to its inherent instability and energy investment required to create.

Cyclopropane modification of the proximal modification site has been identified to exist in both \textit{cis} and \textit{trans} isomers, each with distinct immunological properties. The cis modification was described first and is added to TDM by the protein product of the bacterial gene \textit{pcaA}. \textit{M. tuberculosis} deficient in \textit{pcaA} are hypoinflammatory in a mouse model of infection, suggesting that cis-cyclopropane modified TDM is pro-inflammatory. Loss of this gene results in an overall reduction in bacterial survival. This somewhat contradictory result indicates that aspects of the host inflammatory response are important for bacterial survival and replication, findings that have since been replicated in a variety of other contexts in respect to tuberculosis disease. Alternately, trans-cyclopropane modification of TDM is catalyzed by CmaA2 and this orientation was found to be hypoinflammatory. Similar to \textDelta \textit{pcaA} \textit{M. tuberculosis}, loss of \textit{cmaA2} resulted in a bacterial growth defect and prompt clearance of the bacteria, but by an alternative mechanism. Instead of a muted inflammatory response, \textDelta \textit{cmaA2} \textit{M. tuberculosis} induced hyperinflammation. This body of work, largely from the Glickman lab, established a variety of important roles for related but enantiomerically distinct versions of the same biomolecule that differ at only a single chemical site. This specificity is evocative of the high degree to which mycobacterial species have adapted to their hosts by developing novel modifications and mechanisms to perturb the immune response in their favor.

\subsection{Moonlighting}

Pathogenic microorganisms are often constrained by genomic size -- too small and too few essential functions can be encoded; too large and the risk of duplication errors and cost of maintenance becomes prohibitory. There is therefore a great deal of evolutionary pressure to economize and multitask -- why have two proteins to do two functions if one can do both? That is the precise logic underlying many bacterial toxins, secreted effectors, and structural features. One of the most famous of these multifunctional proteins,�� often dubbed ��moonlighting proteins\footnote{Conceptually, of course, moonlighting is purely orientational. While the given example is one instance where a historically well-defined enzyme has additional functions based on localization, other multifunctional enzymes that can target both bacterial and host substrates or that have distinct functions when cytosolic or periplasmic or secreted are unlikely to be given this title unless they bear high homology to universally conserved proteins.}, is the alpha-enolase from \textit{Streptococcus pneumoniae}. Enolase is an enzyme critical to glycolysis and converts 2-phosphoglycerate to phosphoenolpyruvate, which is essential for the breakdown of glucose into pyruvate. However, \textit{S. pneumoniae} also secretes this normally cytosolic enzyme onto the surface of the outer membrane, which allows it to interact with host plasminogen and catalyze its conversion into active plasmin. Plasmin degrades host fibrin clots, leading to enhanced tissue invasion and pathogenicity through avoidance of host containment by fibrin and increased dissemination. By evolutionary addition of plasminogen-binding properties, fusion of two unrelated proteins into a single protein, alterations of protein localization, or novel layers of regulation, bacteria can, in a very efficient way, exert multiple essential functions from single biological products.

Similar to protein examples, which tend to be more obvious, the structural features of the bacteria can also serve important "moonlighting" functions in the sense that single elements can play key roles in seemingly unrelated phenomena. TDM is an excellent example of this - it is a conserved feature of non-pathogenic mycobacterial species, suggesting that this feature likely emerged to address environmental needs that preceded the need to engage with host immunity. Indeed, TDM serves such a wide array of important functions in the physiology of (especially pathogenic) mycobacteria that to assign it a "major" function would be rather fallacious. As a major structural component of the cell wall, defense from the environment is clearly the overarching theme of this sophisticated glycolipid, but what does that really mean? 

Strictly in the context of host immunity, TDM had been generally ascribed a few major roles: blockade of lysosome-phagosome fusion, alteration in expression of major immunoregulatory cytokines, induction of humoral immunity, and mediation of granuloma formation. Delipidation of mycobacteria results in a profound alteration of the overall inflammatory response \textit{in vitro} and results in efficient bacterial killing by macrophages but perturbed expression of IL-1\textbeta, TNF\textalpha, IL-6, and IL-12. It is now though that many of these functions are mediated by recognition of TDM by surface host receptors, a topic that will be returned to shortly. However, the expression of these critical cytokines (among many others) regulated by TDM results in profound changes in the overall tone and tempo of the inflammatory response that, in aggregate, contribute to granuloma formation, a process we now know to be dependent on both pro- and anti-inflammatory signaling molecules, including IL-4, IL-3, IFN\textgamma, and TNF\textalpha. These processes are intimately linked with the phenotype that will be further explored throughout this work: the TDM-dependent induction of VEGFA and resultant angiogenesis.

\section{Angiogenesis}

Tissue perturbations, such as those caused by granulomas, often drive the invasion of blood vessels toward the site as a mechanism to facilitate tissue repair. However, these blood vessels can serve as a maladaptive response in many contexts. Most famous is the context of tumor biology, where these vessels serve as a supply of oxygen and glucose, a route of dissemination to distal sites, and a paradoxical barrier to the effective delivery of curative chemotherapeutics. In the transition toward chemotherapy options with lessened toxicity, a number of kinase inhibitors and monoclonal antibodies were developed that target a specific receptor on those blood vessels required for their growth and maintenance: the vascular endothelial growth factor receptor 2 or VEGFR2. This tyrosine kinase receptor triggers a downstream transcriptional response cascade that results in endothelial proliferation and directed growth toward the source of the ligand: the vascular endothelial growth factor, or VEGF. By inhibiting either the enzymatic activity of the receptor using kinase inhibitors or blocking the interaction between the receptor and the ligand using monoclonal antibodies, effective regression of the vascular webs around tumors can be achieved. This therapy has become standard of care for a subset of tumor types and physiological locations, but the mystery remains why this therapeutic strategy targeting a highly conserved (indeed nearly ubiquitous) feature of tumors is not more broadly applicable and generally successful.

The most common of the anti-angiogenic therapies targeting VEGFR2 is bevacizumab. Bevacizumab is a humanized monoclonal antibody that very potently (KD = ) blocks the interaction between VEGFR2 and VEGF and induces vascular regression. However, the physiological stress that this causes appears to drive a compensatory upregulation of VEGF production by the tumor itself – the escalating hypoxia in the local region drives rapid amplification of VEGF production to alleviate such detrimental hypoxia. By this mechanism it is proposed that tumors increase the local VEGF concentration beyond the binding affinity of bevacizumab for VEGFR2 and promote vascular relapse and renewed angiogenesis toward the site. 

Thus, despite the initial promise of anti-angiogenic therapy, the current implementations have several shortcomings that need to be addressed before this can be a viable and widespread strategy to treat solid cancers. However, by analogy, the same challenges exist with using anti-angiogenic therapies to treat other vascularized disorders. Given the central role of the hypoxia response driven by HIF1a to the induction of angiogenesis through the regulation of VEGF, efforts at inducing vascular regression inevitably drive a reduction in local oxygen tension and a corresponding increase in HIF1a activity and VEGF production. This has logically led to investigation into HIF1a-targeting therapeutics, despite the many challenges associated with targeting transcription factors. 

HIF1a-directed therapeutic options remain limited in 2022. The most promising drug candidates are actually those that agonize HIF1a and drive increased local angiogenesis, which is rather beneficial for a number of disorders, including major burns and diabetes. However, existing inhibitors, through either direct or indirect mechanisms, remain either impotent or excessively toxic in vivo. However, it has long been established that other transcriptional pathways are important for the production of VEGF and these may prove to be a more fertile ground for discovery.

\subsection{Developmental Angiogenesis}
\subsection{Angiogenesis in Cancer}
\subsection{The Relative Failure of Bevacizumab}
\subsection{Historical Observations of Angiogenesis in Tuberculosis}
\subsection{Modern Studies on Granuloma Angiogenesis}

\section{C-Type Lectin Signaling}
Another notable multipurpose biological product is the lipopolysaccharide (LPS) of Gram-negative bacteria. LPS is a critical component of the outer leaflet of the outer membrane in Gram-negative bacteria and a central interface with their hosts, for host-associated species. As a result, diverse eukaryotes, including both plants and animals, have developed a family of receptors known as Toll-like receptors (TLRs), one of which – TLR4 – induces an inflammatory transcriptional response in many vertebrates. Still an active area of study, the precise composition of LPS fundamentally alters the ability of TLR4 to bind and respond. Pathogenic species of Gram-negative bacteria tend to have six (6) lipid tails on LPS that activate TLR4 while commensal or environmental species have five (5) or fewer lipid tails that do not activate TLR4\footnote{For instance, the oral opportunistic pathogen Porphyromonas gingivalis produces a tetraacylated LPS that actually inhibits TLR4 activation by hexaacylated LPS from E. coli (citation).}. Precisely why and how these differences have emerged and evolutionary rationales for the failure of pathogenic species to adopt immune evading 5 tail version of LPS are an active area of research, but it seems probable that some aspects of the TLR-dependent response pathway offer benefit to the bacteria and are an avenue for bacterial subversion of the host immune response.

TDM exerts similarly diverse functions to LPS and is also detected by host pattern recognitions receptors (PRRs), including TLR2 – another member of the Toll-like receptor family – and two C-type lectin receptors (CLRs), MINCLE and MCL. TDM is a structurally essential component of mycobacteria – if it is experimentally stripped using organic solvents, the bacteria must either regenerate it de novo or are profoundly susceptible to any additional stressor and the most rudimentary of host immune responses. Thus, in addition to the important structural aspects of TDM, it also possesses a number of chemical and biological functions in mycobacterial interactions with their hosts.

Chemically, TDM is radically different from nearly any other biomolecule that an organism is likely to encounter. Comprised of a trehalose head group – an unusual di-glucose that is never synthesized by animals – with profoundly hydrophobic, extremely long, and diversely modified branched fatty acid tails, TDM is directly cytotoxic to cells through disruption of plasma membrane integrity. However, at physiologically relevant concentrations and (importantly) presentation, the primary mechanisms of host response are through the activation of the aforementioned PRRs – TLR2 and MINCLE/MCL\footnote{In the literature, these protein products are often listed using mouse-specific nomenclature as Mincle and Mcl for the sake of being more word-legible. MINCLE and Mincle are the protein products of the genes CLEC4E and Clec4e; MCL and Mcl are the protein products of CLEC4D) and Clec4d, from humans and mice respectively.}. Given the complexity of TDM and the different arrangements that it can adopt in vitro, many of the previous studies in the literature on the immune response to TDM are difficult to reconcile and integrate as presentation on plastic-adherent monolayers stimulate different types of responses to TDM presentation on polystyrene beads, which are somewhat more similar to the presentation on the surface of mycobacteria. By generalization, monolayer-stimulated responses are more pro-inflammatory than those in curved configurations, such as on beads or bacilli. The physiological relevance of these monolayer-like configurations of TDM is up to some debate, but there is some evidence that planes of TDM from dead mycobacteria can form in vivo. 

Activation of either TLR2 or MINCLE/MCL terminate in the activation of NF-κB, a generally pro-inflammatory transcriptional immune pathway. While TLR2 is expressed on a rather wide diversity of cell types, MINCLE and MCL are specific to myeloid cells – the broad category of innate immune cells that includes macrophages, neutrophils, and dendritic cells. Additionally, the precise outcomes of NF-κB activation can vary based on the particular cell type, the length of stimulation, and other factors. 

Interestingly, despite this commonality, TLR activation follows a highly proscribed set of signaling cascades that, in varying ways and to varying degrees, are dependent upon NF-κB. For instance, the primary mode of signal transduction depends on MYD88 and/or TRIF, two adaptor proteins, which ultimately lead to the phosphorylation of inhibitor of nuclear factor kappa B (I-κB) and subsequent activation of the NF-κB subunit. An additional mode is through the activation of the ASC-dependent inflammasome signaling complex, which processes pro-IL-1β and pro-IL-18 for secretion and paracrine and autocrine signaling. However, the IL-1 receptor also induces a MYD88-dependent signaling pathway that terminates in NF-κB. This single pathway thus plays a critical role various facets of the host response downstream of TLR activation, which unifies the response tone while potentially limiting response diversity; while TLRs are somewhat broad in their expression pattern, the induction of IL-1β secretion activates all neighboring cells that express IL-1R, which is practically ubiquitous in environment-facing tissues. 

By contrast, CLRs terminate in at least two known downstream signaling pathways. In addition to NF-κB, they are capable of activating the nuclear factor of activated T cells (NF-AT or NFAT) pathway. This ability to activate multiple layers of transcriptional regulation either at the same time or under different contexts (length of time, strength of agonism, particular ligand) offers CLRs a powerful additional mechanism of modulating the tone of the immunological response in response to particular insults. CLRs are known to respond primarily to carbohydrate-linked ligands, as they contain lectin domains able to recognize either glucose- or galactose-derived saccharides. Many biomolecules are sugar-modified, from bacteria, fungi, viruses, and eukaryotes (both self and pathogens). This allows CLRs to be a major pathway for the response to host-derived damage-associated molecular patterns (DAMPs) as well as microbe- or pathogen-associated molecular patterns (MAMPs, or more commonly, PAMPs). 

Indeed, MINCLE (from the gene CLEC4E, Macrophage inducible C-type lectin), was originally identified as a receptor for SAP130, a nuclear protein that is exposed to the extracellular milieu after necrotic cell death, which is then able to activate macrophages to scavenge cellular debris. These early observations were, themselves, clues to the pleiotropic nature of Mincle activation as a strictly inflammatory response to cell death would be inappropriate in tone for the majority of innocuous cell death events that occur almost constantly in the day-to-day lives of organisms comprised of billions of cells. While NF-κB is broadly considered a pro-inflammatory pathway, it induces the expression of several cytokines that are functionally pleiotropic. While entire dissertations could be, and have been, written about interleukin-6 (IL-6), suffice it to say that IL-6 can be both pro- and anti-inflammatory based on the particular circumstances in which the responding cells detect it, the length of exposure, and more. 

C-type lectin receptors, or CLRs, are a diverse class of pattern recognition receptors that are defined by their use of calcium (Ca2+) to coordinate the binding of carbohydrate patterns, generally segregated into two major classes: QPD motif lectins, which bind galactose-containing sugars, and EPN motif lectins, which bind mannose- or glucose-containing sugars. QPD-containing C-type lectins are, in general soluble or secreted proteins and include the likes of human tetranectin, an extracellualr matrix-interacting protein, and herring antifreeze protein, which mediates the breakdown of ice crystals in the blood of cold-water fish. By contrast, EPN C-type lectins play a diverse set of roles and many are the classical members of the CLR family. Most notable among these EPN-containing CLRs is DECTIN-1, the archetypal member of the family which has long been studies for its roles in antifungal immunity, but has now been discovered to have a diverse set of roles in other conditions, including to bacterial pathogens and in autoimmunity.

Two additional members of the EPN-containing superfamily of CLRs are MCL and MINCLE. MCL, originally dubbed DECTIN-3\footnote{And for historical reasons, is still occasionally called this in the modern literature.}, 

\subsection{History of Pattern Recognition Receptor Signaling}
\subsection{Discovery and Characterization of C-type Lectin Receptors}
\subsection{Ligand Presentation and Pattern Recognition Receptor Responses}
\subsection{Diversity of Outcomes to Receptor Activation}

All the major families of pattern recognition receptors\footnote{Those being Toll-like receptor (TLR), NOD-like receptor (NLR), RIG-I-like receptor (RLR), and C-type lectin receptor (CLR) families of receptors.} are known to induce the activation of NF-κB, but many of them have specific additional pathways that they are known to induce that drive a particular kind of immune response that depends on the cell type, the particular receptor activated, the specific ligand, the duration of activation, other physiological variables, and more. For instance, RIG-I-like receptor activation after detection of pathogen-derived nucleic acids drives the nuclear translocation of IRF3 and IRF7 to produce type I interferons (IFNα/β), which induces both a paracrine (in neighboring cells) and autocrine (self) response to protect against viruses. 

Additionally, particular ligands can have multiple means of detection based on their particular presentation. The canonical example is lipopolysaccharide (LPS) from Gram-negative bacteria. Extracelluarly, detection can occur through cooperation of CD14 and TLR4, which coordinate the activation of MYD88 and subsequent activation of NF-κB. Intracellarly, detection is mediated by caspases 4 and 5, which drives inflammasome assembly to process pro-IL-1β and pro-IL-18 into their active, secreted forms, which also drives both paracrine and autocrine signaling cascades to defend against intracellular Gram-negative bacterial pathogens.

TDM, at least compared to LPS, is a relatively understudied molecule as far as the precise mechanisms of detection and response. This has led to there remaining a degree of uncertainty in the field over the contributions of either CLR signaling through MCL and MINCLR or TLR signaling through TLR2 and MARCO to the overall effect of TDM detection on the cellular response. Additionally, there is relatively little known about different physiological presentations and their impact on the response to TDM. In vitro, TDM has been demonstrated to adopt different conformational states based on surface composition and geometry. On beads of a small (exact number) diameter, it adopts a bilayer configuration similar to that seen on live bacilli; on larger beads or on a plane, it acts as a monolayer. The “monolayer” configuration is more inflammatory but was also thought unlikely to exist in vivo. Recent hypotheses have challenged this notion, but what is clear is that TDM must be presented to cells in particular arrangements to have an effect, which is seen in head-to-head comparisons between heat-killed Mtb and gamma-irradiated Mtb. While gamma-irradiated Mtb maintain their shape and structure, heat-killed Mtb are broken down and the presentation of TDM is no longer able to activate CLRs even though it becomes a very potent TLR-mediated vaccine adjuvant. Thus, across different types of bacterial ligands, the context of their presentation to a host is a key determinant of their overall effect on the immune response. This will be a key point in the development of several of our assays in the next chapter. 

\subsection{MINCLE and MCL Detection of TDM}

\section{Nuclear Factor of Activated T Cells (NFAT)}

NFAT, by contrast, is widely accepted to be a pleiotropic pathway as a product of the foundational studies in the pathway conducted on T cell activation, where it is essential for both the expression of IL-2 by dendritic cells to activate TH2 cells and the differentiation of T cells into TH2 cells. It is also essential for IL-4 induction, which is widely considered the canonical anti-inflammatory (or inflammation-resolving) cytokine. Remarkably, it is also important for mediating the expression of TNFα and IFNγ, critically important pro-inflammatory cytokines. Other factors must intervene in the overall response, likely through the modulation of other pathways by the pathogen to drive particular types of responses to their overall benefit. 

NFAT was discovered relatively early on to be one of the major and defining responses to CLR activation. Defined by Goodridge et al. in 2007 as an important response mechanism, it has been co-opted over the years as an experimental tool to measure CLR activation because TLRs do not activate NFAT. By using either NFAT proteins fused to fluorescent proteins to monitor nuclear localization or the DNA regulatory elements for NFAT to drive luciferase from a minimal promoter, it is possible to capture a report of NFAT activation with high sensitivity and with rapid response times. This has been used dozens of times in the literature of define the specificity of a response for a particular receptor and ligand. Despite the ironic ubiquity of this approach as experimental tool, very little additional work has been done to define the functional consequences of NFAT activation downstream of CLR activation, especially in the specific context of Mincle or Mcl agonism. Given the specificity of the NFAT response, there must be important biological consequences of this pathway being activated during infection, but these have been broadly neglected. 

One of the major reasons for this neglect has been a unitary focus on the importance of CARD9-BCL10-MALT1 (CBM) signalosomes as another unique consequence of CLR agonism. Despite this method of activation that has more in common with B cell receptor activation than TLRs, the functional downstream consequence is the same: nuclear translocation of NF-κB and associated induction of immune response genes. Furthermore, the evidence is extremely strong that CBM-depenent signaling is critical for the response to a variety of fungal pathogens and that these generally type I responses are a potent defense against infection. However, numerous datasets have provided evidence of a range of genes that depend on CLR activation but are CARD9-independent. Some of these genes are likely to be NFAT-dependent while others may be activated by as-yet unidentified pathway. 

NFAT has many features that make it a transcription factor family of broad basic as well as translational interest. The NFAT family is comprised of five members: NFATC1 (also known as NFAT2), NFATC2 (NFAT1), NFATC3 (NFAT4), NFATC4 (NFAT3), and NFAT5. Historical reasons have resulted in a convoluted nomenclature\footnote{As often happens in science when multiple independent lab groups discover proteins at the same time, the naming can become a challenge as the field as whole reconciles two distinct naming schema. In this case, no resolution has ever come about. While NFAT was originally identified in \textit{Drosophila} as TonB, the NFATc subnomenclature was meant to designate that they are calcium-responsive and calcineurin-dependent and distinct from NFAT5, the modern homolog of the ancestral protein with high sequence similarity from humans to sponge. In choosing to maintain the NFATc nomenclature, I take no position on the relative merits of the two systems. Additional, now largely outdated, naming schemes had an additional name for each of the isoforms that I will address only as needed throughout this document.}, so for the sake of consistency, the NFATCx naming scheme will be used throughout this document. NFAT5 is a special member of this family that appears to be important for the transcriptional response to osmotic stress, but unlike all of the other members, is not regulated by changes in cytosolic calcium concentration via calcineurin.

The four calcium responsive members have long been assumed to be functionally redundant, with their roles defined by their patterns of tissue expression. All of them are derived from an ancestral single isoform that was duplicated over the course of evolution (although intermediates with greater than one but fewer than four isoforms are unknown among modern species). However, evolution has provided each of these isoforms distinctive biophysical properties that allow them to have non-redundant roles even in cell types where more than one is expressed simultaneously. Most notable is their alterations in sensitivity to changes in calcium: while NFATC2 has a persistent response after strong activation, NFATC3 rapidly traffics in and out of the nucleus in response to small magnitude changes in calcium. 

NFAT requires the phosphatase calcineurin for their activation. Upon an increase in calcium, calcineurin dephosphorylates NFAT to expose a nuclear localization sequence (NLS); once in the nucleus, kinases (including GSK3 proteins and protein kinase A) phosphorylate NFAT to drive it back into the cytosol in inactive form. This shuttling behavior allows for existing pools of NFAT to rapidly modulate host responses, including developmental, immunological, and pathological responses. This also allows for rapid tuning of the longevity of the response, presumably allowing for the induction of different genes and to different degrees based on the length of activation. Although no work has ever been done to define such distinctions, the principles of biochemical affinity dictate that more accessible chromatin with more NFAT binding sites would be activated prior to those in less accessible configurations or with fewer sites more distal from the transcriptional start site, which may require long periods of strong activation to be induced. Defining these different classes of genes in different cell types would provide a far greater depth of understanding for the consequences of NFAT activation and timing of intervention for maximum medical benefit.

Recently, and concurrently with the present work, others have identified Vegfa\footnote{In the majority of this document, human gene nomenclature is used when referring to pathways in the abstract. However, when relevant to the literature being discussed, the appropriate model organism’s field-appropriate nomenclature will be used. Later, when work specifically done in zebrafish is discussed, the nomenclature will use zebrafish nomenclature.} as an NFAT-dependent transcriptional target in myeloid cells downstream of Dectin-1 activation through the use of genetic knockouts of Card9 in mice and in vitro use of NFAT inhibitors after Dectin-1 agonism. This was among the first published works in over a decade to identify a discrete effect downstream of CLR activation that is NFAT-dependent and Card9-independent. Furthermore, there is somewhat of an NFAT renaissance occurring in the literature at the time of writing. Several new papers have emerged in the past several months identifying novel new roles for NFAT signaling in a variety of (predominantly hematopoetic) tissues, giving new emphasis to this long-neglected pathway. Some of the work discussed in later chapters adds to this body of NFAT-dependent responses and, hopefully, encourages additional future work to define the roles of this important but understudied pathway in the response to not only tuberculosis but the full range of human diseases that engage CLR signaling, especially fungal diseases and additional autoimmune disorders.

\subsection{Review of Known Roles for NFAT}
\subsection{Clinical Utility of NFAT Inhibitors}
\subsection{Differentiation of Individual Isoforms}
\subsection{New Roles for NFAT}
The central role of NFAT in the immune system has long been appreciated, albeit in a rather limited context, via the widespread application of NFAT inhibitory drugs in the clinic. Two drugs are widely used to block calcineurin activation and suppress immune responses: cyclosporine A and tacrolimus. These drugs were discovered and developed for clinical use in order to target the T cell response and prevent organ transplant rejection by blocking the affinity maturation and proliferation of anti-graft T cells. The profound and global immune suppression that accompanies the use of these drugs has prevented their use in other contexts for fear of increase susceptibility to infectious diseases. The weakness of these drugs is that they block all calcineurin activity in all cell types, leading to a vast range of collateral targets – a better approach would be to find a way to locally target only the disease-relevant target of calcineurin (in this case, NFAT). Halfway approaches have emerged using tacrolimus (and derivatives) through its use as a topical ointment for atopic dermatitis, but this is inherently limited to skin conditions. What is needed is a generalizable mechanism to deliver potent and localized cellular inhibition of NFAT. Future efforts toward this end may apply adeno-associated virus (AAV) vectors, liposomes, or other delivery mechanisms to drive the expression of VIVIT in specific tissues at particular times.

In the modern era, further roles have been investigated for NFAT that remain somewhat mysterious in mechanism and ontogeny. NFAT activation alters the behavior of platelets and drives inflammatory cascades during Gram-negative sepsis. Mammalian platelets are anucleated, so it is not clear how NFAT is able to modulate cellular behaviors in the absence of its canonical function as a transcriptional activator. The mechanisms of this are certain to be a fruitful avenue of future investigation and are likely to be applicable to nucleated cells as well – new tools and deeper understandings of NFAT protein topology will be required to differentiate these classes of functions in these cells. 

\section{Host-Microbe Interactions to Study Cell Biological Processes}
\subsection{Host-Directed Therapies: History and Promise}

One of these defining characteristics is the formation of caseating granulomas. These granulomas, formerly known as tubercles\footnote{Hence, \textit{tubercul}-osis.}, are the most notable and ubiquitous pathology of human tuberculosis. These granulomas are a highly conserved immunological response to any object – pathogen or otherwise – that the immune system is unable to clear and are an imminently visible and clinically definitive manifestation of tuberculosis\footnote{A large body of work exists on the mechanisms that Schistosoma eggs use to induce parasite-beneficial granuloma formation. However, even in the absence of active biological induction of granulomas, sterile but indigestible objects will induce granuloma formation, albeit with some distinguishing characteristics.}. For reasons that remain poorly understood, but likely related to the inflammatory biases of the C57BL/6 and other mouse models, these mice do not form granulomas\footnote{Strangely, these mice do form granulomas in response to Schistosoma and other stimuli, suggesting something distinguishing about mycobacterial infection and perhaps offering clues as to the unique characteristics of the tuberculous granuloma.} after being infected with Mycobacterium tuberculosis and mice do not harbor a strain of Mycobacterium that infects them in the wild. This has set the mouse on an evolutionary trajectory where potentially adaptive – or maladaptive – responses to mycobacterial infection fail to occur. No matter the relative costs or benefits to the host of granuloma formation, the inability of any as yet known mouse model (with the partial exception of the C3H/FeJ model) to form granulomas compromises their ability to serve as a physiologically relevant model of some, but not all, aspects of human tuberculosis.

A major challenge has been the specific identification of diseases, stimuli, and biological consequences that drive angiogenic effects. While the angiogenic response to tumors is thought to be mediated strictly through a hypoxia-dependent mechanism, the angiogenic response to other stimuli are far less homogeneous. For instance, in the context of the tuberculous granuloma, these structures initially form in the oxygenated environment of the human lung, which encounters 21\% oxygen in air approximately 16 times per minute – not an environment that would generally facilitate a hypoxia response. While it is certainly possible in occluded sites to create acute hypoxia, the angiogenic response within the lung would be assumed to rapidly and efficiently alleviate this stressor. No systematic comparison has been done to truly measure the precise oxygen tension in these granulomas from either humans or non-human primates, so it remains difficult to make sweeping assertions. Regardless, the experimental identification of particular mycobacterial components able to induce angiogenesis suggests more sophisticated immunological mechanisms at play than simple hypoxia.

This bacteria-centric approach to treatment of tuberculosis seems logical, as bacteria possess many functions that humans lack entirely that are necessary for their pathogenicity, making these appealing targets for drugs. However, this opens the door to the emergence of resistance when treatment is unable to clear the infecting bacteria and a tolerant or resistant population then expands anew. This makes a compelling niche for a new approach to the treatment of chronic bacterial (and fungal and viral) infections: the host-directed therapy. Host-directed therapies have long been used in cancer. Indeed, anti-angiogenic therapy is one of the earlier examples of a host-directed therapy to cancer. But  translating such therapies to infectious disease has, thus far, proven difficult or impractical. One of the reasons is a lack of understanding of the underlying mechanisms that could be targeted to benefit the host to bacterial detriment; another is the difficulty in interfering with host processes in ways that are specific to the site of infection while minimizing overall toxicity. While host toxicity is generally acceptable collateral damage in cancer treatment, this is often viewed less favorably when treating infectious diseases for which pathogen-targeting therapies are thought superior. Despite these challenges, mycobacterial infections, as a product of the unique intersectionality of host and bacterial biology in the granuloma, offer a spectacular opportunity to develop host-directed therapies that shorten time to cure, abbreviate the current drug regimen, prevent the emergence of antibiotic resistance, and, ultimately, fulfill the World Health Organization’s goal of eradicating tuberculosis by 2050\footnote{Disease eradication has long been a stated goal of many public health campaigns, but has thus far been successful precisely twice: against the scourge of smallpox (in 1977) and against rinderpest (a disease of cattle, in 2011). Current campaigns show promise in the eradication of dracunculiasis (or guinea worm) in the immediate future, with cases down to 14 in 2021. Others, including polio, yaws, and rabies, remain elusive despite all having effective vaccines or treatments, are human-exclusive (or have a known, discrete reservoir), and declining case counts. In the eyes of many, polio is an exceptional disappointment given how close we have come, but the continued need for the use of the oral polio vaccine makes eradication all but impossible in the immediate term.}. 

These conflicting responses are indicative of the importance of other factors in determining the overall inflammatory tone of a particular response to a particular insult, a theme that will emerge throughout this dissertation.

Among the guiding themes of this thesis is that immune responses are never solely one thing or the other. There is growing acceptance that biological responses in general are far more complex than has been generally acknowledged in the literature to date. In the context of mycobacterial infection, the balance of inflammatory and anti-inflammatory responses determines the ability of the host to survive infection. Beyond infection, the balance of signals creates human predisposition to allergies, autoimmunity, cancer, heart disease, and many other disorders. A deeper understanding of the ways that individual signal transduction cascades can drive both type I and type II responses is essential for the development of better therapeutics to treat diseases with underlying ontogenies from either type of response. 


% A simple way to make sections
% \section{Section}
% Lingva diverseco Homa emancipigxo Cxiu lingvo liberigas, kaj lingva identeco sed ne limigite de ili Ni asertas ke la ekskluziva!\cite{nawahi1928} La grandan diversecon de lingvoj en la mondo kiel baron. Profitus el la scio de dua lingvo Ni estas movado por efika. Etna lingvo estas ligita al difinita perspektivo pri la. 

% Gxi ne estas bazita sur respekto al kaj subteno de cxiuj. Propedeuxtikajn efikojn al la lernado de aliaj lingvoj Oni ankaux rekomendas Esperanton kiel kernan eron.


\vspace{0.2in}

% An example of making lists of various kinds - This one gives black circles of certain size based on style files - 
% LaTeX manual will tell you how to put numbers or different symobols


Naciaj lingvoj neeviteble  En la Esperantokomunumo la anoj. Al cxiu homo partopreni kiel.  La, sed ne limigite de ili . 

Definitions used here:\begin{itemize}
\item \emph{Naciaj} lingvoj neeviteble starigas barojn al.
\item \emph{Starigas} barojn al, cxe granda parto de la monda logxantaro.
\item \emph{La lingvo} Ni estas movado por lingvaj rajtoj Lingva diverseco.
\item Ni asertas ke la ekskluziva uzado de naciaj lingvoj \emph{hoarder}.
 \end{itemize}

% Ah! a little bit of math - all math is between two "$" signs

Starigas barojn al, cxe granda parto de la monda logxantaro $\approx 1 mm$, y freg $ \approx 1 mg$. Hha jong shiel odieio $\delta E_p = m g d \approx 10^{-8}$ Joules.  

Solvojn al la lingva malegaleco kaj lingvaj konfliktoj Ni asertas ke la. Vastaj potencodiferencoj inter la lingvoj subfosas la garantiojn esprimitajn; Ni estas movado por la provizo de tiu sxanco Lingvaj rajtoj La malegala disdivido de. Estas senescepte du aux plurlingvaj Cxiu komunumano akceptis. Kaj evoluigo se gxi ne estas. 

% Thats about it - one more thing - a table can be inserted in the following way - 

\begin{table}
\centering
\begin{tabular}{| c | c | c | c | c |} % Options are c,l,r : centered, left justified, right justified
\cline{1-5}
 \multicolumn{1}{| c |}{$\phi_c$}&\multicolumn{2}{| c  |}{Before}&\multicolumn{2}{| c |}{After}\\
\cline{1-5}
 &$Z_{c}$&$\beta$&$Z_{c}$&$\beta$\\
\cline{1-5}
0.84058 &$2.390\pm 0.135$ &$0.5166 \pm 0.064$&$1.198 \pm 0.310$&$0.5024 \pm 0.093$\\
\cline{1-5}
0.84075&$2.512 \pm 0.138$&$0.5472 \pm 0.073$&$1.071 \pm 0.359$&$0.4601 \pm 0.090$\\
\cline{1-5}
0.84172&$2.632 \pm 0.151$&$0.4935 \pm 0.077$&$0.9747 \pm 0.458$&$0.3631 \pm 0.083$\\
\cline{1-5}
0.84204&$2.858 \pm 0.127$&$0.5637 \pm 0.086$&$1.183 \pm 0.413$ &$0.3665 \pm 0.079$ \\
\cline{1-5}
0.84236&$2.916 \pm 0.133$&$0.5555 \pm 0.093$&$1.744 \pm 0.298$&$0.445 \pm 0.088$\\
\cline{1-5}
0.84269&$3.003 \pm 0.124$&$0.5627 \pm 0.095$&$1.989 \pm 0.267$&$0.4691 \pm 0.092$\\
\cline{1-5}
0.84301&$3.075 \pm 0.12$&$0.5603 \pm 0.095$&$2.28 \pm 0.235$&$0.5245 \pm 0.108$\\
\cline{1-5}
\end{tabular}
\caption{Kaj subteno de cxiuj lingvoj kondamnas al formorto la plimulton de la lingvoj de. Ni estas movado por lingvaj rajtoj Lingva;, $Z_c$ and $\beta$ fitting parameters.}
\label{Table1} 
\end{table}

Ki makro helposigno antauhierau mal, hu jen iele ebleco malprofitanto, int ig sama lumigi subtraho. Op plena deziri hot, infano sensubjekta alternativo al sin. Kvin jesa povus ci dev, kor'o sekvanta kontraui ko cis. Nv pera simil sia, he propozicio antauelemento nia.

\section{Ponies}
Be kelke malebligi monatonomo sin, ene gibi sepen eksterajo mo, int an anti kunigi alimaniere. Suba frazparto vo cit. Mo horkvarono frakcistreko sen. Ies gv neniajo sensubjekta, eksterajo cirkumflekso ts unt. En nette singularo geinstruisto mil, ie samo grupo nen.

%figure here please


\subsection{Little Ponies}
Modo tiela us cii, ne ehe intere rilativo. Ferio multiplikite id ajn. Tiele nenio akuzativa co ian. Unu ilia longa leteri op, vola hola ge cit, altmontaro kromakcento mi des. Ont lo grupo sezononomo, um kaj elparolo sanskrito.

\subsection {Medium Ponies}
Bat'o gingivalo u ant. Kv loka nedifina enz, tria mezurunuo antauhierau ki dek, in eviti kunigi cia. Ac sat reen kiomas. Tiu uk istan dekono jugoslavo, mal minus iufoje oj. Volus hodiaua plue ol, hoj go lasi tempismo, as jaro rekta tra. As bis grupo infano esperantigo, nenio rilativa ligvokalo po iom.
\footnote{Dume horo centimetro uj jes 1999.}

\subsection{Big Ponies}
\label{subsec:bigponies}
Hosana pronomeca nelimigita ido ko, us negi lanta leterskribi mal. Re nia panjo alikvante nombrovorto, via tc bisi hekto koruso.\footnote{Dume horo centimetro uj jes 1997.} Cii go unun oble drumo. Ke ties okej laringalo mia, anti duona alial ing fi. Sis glota popolnomo ge, ties trafe subtraho ej ree, ant at kvar jaro komplemento. It sor tempa oktiliono antaupriskribo.
\footnote{Dodume horos centimetros uj jes 1997-8.}

So ebl poste posta nombrovorto, nul be fine jugoslavo kontraui. Sub ac deka sube, orda hiper u jam. Plu onin iometo ej, os peti irebla per. Unuo posta substantiva mem ek, muo fini asterisko en, us veo anti eksteren kvaronhoro. Ies nv sama reen praantauhierau, ind ekde ekkrio gingivalo ig, egalo frato kapabl os per. De por fora ofon altlernejo.

\[ \frac{\partial u}{\partial t}
   = h^2 \left( \frac{\partial^2 u}{\partial x^2}
      + \frac{\partial^2 u}{\partial y^2}
      + \frac{\partial^2 u}{\partial z^2} \right) \]

Ist land imaga alimaniere dz, ng plue kunigi interalie. Uta vt suli pona, jan nimi sina sinpin tu, anu pana akesi kulupu li. Musi pali mute a len, e mun telo poki. A anu unpa conj kiwen, suli sona n anu, waso mani akesi a oth. Wan pipi nena vt. Lete conj nasa ike mi. Awen mani utala n ken, ike o nena kulup

\vspace{0.2in} 

% A simple way to make sections
\section{Section}
Lingva diverseco Homa emancipigxo Cxiu lingvo liberigas, kaj lingva identeco sed ne limigite de ili Ni asertas ke la ekskluziva!\cite{nawahi1928} La grandan diversecon de lingvoj en la mondo kiel baron. Profitus el la scio de dua lingvo Ni estas movado por efika. Etna lingvo estas ligita al difinita perspektivo pri la. 

Gxi ne estas bazita sur respekto al kaj subteno de cxiuj. Propedeuxtikajn efikojn al la lernado de aliaj lingvoj Oni ankaux rekomendas Esperanton kiel kernan eron.


\vspace{0.2in}

% An example of making lists of various kinds - This one gives black circles of certain size based on style files - 
% LaTeX manual will tell you how to put numbers or different symobols


Naciaj lingvoj neeviteble  En la Esperantokomunumo la anoj. Al cxiu homo partopreni kiel.  La, sed ne limigite de ili . 

Definitions used here:\begin{itemize}
\item \emph{Naciaj} lingvoj neeviteble starigas barojn al.
\item \emph{Starigas} barojn al, cxe granda parto de la monda logxantaro.
\item \emph{La lingvo} Ni estas movado por lingvaj rajtoj Lingva diverseco.
\item Ni asertas ke la ekskluziva uzado de naciaj lingvoj \emph{hoarder}.
 \end{itemize}

% Ah! a little bit of math - all math is between two "$" signs

Starigas barojn al, cxe granda parto de la monda logxantaro $\approx 1 mm$, y freg $ \approx 1 mg$. Hha jong shiel odieio $\delta E_p = m g d \approx 10^{-8}$ Joules.  

Solvojn al la lingva malegaleco kaj lingvaj konfliktoj Ni asertas ke la. Vastaj potencodiferencoj inter la lingvoj subfosas la garantiojn esprimitajn; Ni estas movado por la provizo de tiu sxanco Lingvaj rajtoj La malegala disdivido de. Estas senescepte du aux plurlingvaj Cxiu komunumano akceptis. Kaj evoluigo se gxi ne estas. 

% Thats about it - one more thing - a table can be inserted in the following way - 

\begin{table}
\centering
\begin{tabular}{| c | c | c | c | c |} % Options are c,l,r : centered, left justified, right justified
\cline{1-5}
 \multicolumn{1}{| c |}{$\phi_c$}&\multicolumn{2}{| c  |}{Before}&\multicolumn{2}{| c |}{After}\\
\cline{1-5}
 &$Z_{c}$&$\beta$&$Z_{c}$&$\beta$\\
\cline{1-5}
0.84058 &$2.390\pm 0.135$ &$0.5166 \pm 0.064$&$1.198 \pm 0.310$&$0.5024 \pm 0.093$\\
\cline{1-5}
0.84075&$2.512 \pm 0.138$&$0.5472 \pm 0.073$&$1.071 \pm 0.359$&$0.4601 \pm 0.090$\\
\cline{1-5}
0.84172&$2.632 \pm 0.151$&$0.4935 \pm 0.077$&$0.9747 \pm 0.458$&$0.3631 \pm 0.083$\\
\cline{1-5}
0.84204&$2.858 \pm 0.127$&$0.5637 \pm 0.086$&$1.183 \pm 0.413$ &$0.3665 \pm 0.079$ \\
\cline{1-5}
0.84236&$2.916 \pm 0.133$&$0.5555 \pm 0.093$&$1.744 \pm 0.298$&$0.445 \pm 0.088$\\
\cline{1-5}
0.84269&$3.003 \pm 0.124$&$0.5627 \pm 0.095$&$1.989 \pm 0.267$&$0.4691 \pm 0.092$\\
\cline{1-5}
0.84301&$3.075 \pm 0.12$&$0.5603 \pm 0.095$&$2.28 \pm 0.235$&$0.5245 \pm 0.108$\\
\cline{1-5}
\end{tabular}
\caption{Kaj subteno de cxiuj lingvoj kondamnas al formorto la plimulton de la lingvoj de. Ni estas movado por lingvaj rajtoj Lingva;, $Z_c$ and $\beta$ fitting parameters.}
\label{Table1} 
\end{table}

Ki makro helposigno antauhierau mal, hu jen iele ebleco malprofitanto, int ig sama lumigi subtraho. Op plena deziri hot, infano sensubjekta alternativo al sin. Kvin jesa povus ci dev, kor'o sekvanta kontraui ko cis. Nv pera simil sia, he propozicio antauelemento nia.

\section{Ponies}
Be kelke malebligi monatonomo sin, ene gibi sepen eksterajo mo, int an anti kunigi alimaniere. Suba frazparto vo cit. Mo horkvarono frakcistreko sen. Ies gv neniajo sensubjekta, eksterajo cirkumflekso ts unt. En nette singularo geinstruisto mil, ie samo grupo nen.

%figure here please


\subsection{Little Ponies}
Modo tiela us cii, ne ehe intere rilativo. Ferio multiplikite id ajn. Tiele nenio akuzativa co ian. Unu ilia longa leteri op, vola hola ge cit, altmontaro kromakcento mi des. Ont lo grupo sezononomo, um kaj elparolo sanskrito.

\subsection {Medium Ponies}
Bat'o gingivalo u ant. Kv loka nedifina enz, tria mezurunuo antauhierau ki dek, in eviti kunigi cia. Ac sat reen kiomas. Tiu uk istan dekono jugoslavo, mal minus iufoje oj. Volus hodiaua plue ol, hoj go lasi tempismo, as jaro rekta tra. As bis grupo infano esperantigo, nenio rilativa ligvokalo po iom.
\footnote{Dume horo centimetro uj jes 1999.}

\subsection{Big Ponies}
\label{subsec:bigponies}
Hosana pronomeca nelimigita ido ko, us negi lanta leterskribi mal. Re nia panjo alikvante nombrovorto, via tc bisi hekto koruso.\footnote{Dume horo centimetro uj jes 1997.} Cii go unun oble drumo. Ke ties okej laringalo mia, anti duona alial ing fi. Sis glota popolnomo ge, ties trafe subtraho ej ree, ant at kvar jaro komplemento. It sor tempa oktiliono antaupriskribo.
\footnote{Dodume horos centimetros uj jes 1997-8.}

So ebl poste posta nombrovorto, nul be fine jugoslavo kontraui. Sub ac deka sube, orda hiper u jam. Plu onin iometo ej, os peti irebla per. Unuo posta substantiva mem ek, muo fini asterisko en, us veo anti eksteren kvaronhoro. Ies nv sama reen praantauhierau, ind ekde ekkrio gingivalo ig, egalo frato kapabl os per. De por fora ofon altlernejo.

\[ \frac{\partial u}{\partial t}
   = h^2 \left( \frac{\partial^2 u}{\partial x^2}
      + \frac{\partial^2 u}{\partial y^2}
      + \frac{\partial^2 u}{\partial z^2} \right) \]

Ist land imaga alimaniere dz, ng plue kunigi interalie. Uta vt suli pona, jan nimi sina sinpin tu, anu pana akesi kulupu li. Musi pali mute a len, e mun telo poki. A anu unpa conj kiwen, suli sona n anu, waso mani akesi a oth. Wan pipi nena vt. Lete conj nasa ike mi. Awen mani utala n ken, ike o nena kulup
\end{doublespace}

\chapter{Second Chapter}
\pagenumbering{arabic}

\begin{doublespace}
% The \vspace{} command in this chapter is just for aesthetic reasons - I don't like something new to start at the last line 
%of the page

% ONE OF THE BEST ONLINE LATEX REFERENCES IS AT :
% http://www.eng.cam.ac.uk/help/tpl/textprocessing/latex_advanced/latex_advanced.html

%% ALL figures are in EPS format: It is the best possible format 

% A simple way to make sections

\section{Zebrafish as a Model Organism}

Laboratory model organisms have been a staple of research since the dawn of the scientific endeavor, but only in the past century has standardization of models allowed for improvements in reproducibility and reliability among experiments. One model in particular, the C57BL/6 Mus musculus mouse model, has become a ubiquitous feature of every major research institution all over the world due to their clonal nature\footnote{Genetic diversity between individual C57BL/6 mice is in the range of x single nucleotide polymorphisms per individual in a genome of x bases (ref).}, relative ease of use, and minimal expense\footnote{A single C57BL/6 mouse from Jackson Laboratories (jax.org) at the time of writing is \$USD.}. However, their genetic homogeneity fails to reproduce many phenotypes seen in human disease, making them an excellent model for some disorders and an insufficient one for others. This is nowhere more true than in developmental biology. Although mouse viviparous development is extremely well defined and stereotyped over the course of gestation, that is precisely the challenge. Gestation is an internal and ongoing process of physiological and anatomical development and while it is possible to catalog the process of development in snapshots in time through vivisection, it is impossible to understand the kinetics and processes of development using a model that does not allow for immediate visual accessibility. 

In the 1980s this led researchers in Oregon to seek a model that would allow for the full visual access only possible in oviparous organisms. Although Xenopus frogs had been in use for some time, their long time to sexual maturity (up to 2 years for Xenopus laevis, the dominant model at the time) and other challenges led researchers to a fish model, considered to be the root of the land-adapted branch of the tree of life. A happenstance purchase at a local pet store led to the establishment of the imminently powerful zebrafish model, which led to seminal and otherwise impossible findings in developmental biology. This model has since found applications in nearly every field of biology for many of the same reasons: optical transparency, extremely rapid development, high fecundity, and genetic tractability. These features make the zebrafish an extremely powerful and robust tool for the study of many different biological processes and, thanks to their intolerance for inbreeding, have remained a genetically diverse outbred\footnote{The scale of zebrafish outbreeding is difficult to define, even among “strains” that are used in research laboratories. For instance, the majority of the work in subsequent chapters is done in the *AB background, a classic “wild-type” reference strain used around the world. This strain, similar to other strains, has upwards of 6000 copy number variations between individuals (~15\% of the genome) in addition to approximately 1 single nucleotide polymorphism (SNP) for every 500 bases of genome sequence. Experimentalist anecdotes of the intolerance of the zebrafish for inbreeding are ubiquitous, as this widespread genome-level heterozygosis appears to confer some important advantages to individuals.} model for research that allows for more sophisticated modeling of complex processes with the caveat that it also fuels a need for high n-values due to inherent variation between individuals. Conversely, detectable effects from a high-noise environment are often more robust associations.

Only in the past twenty years has an earnest effort been put forth to develop the zebrafish as a model for immunological studies. Although it has long been known that zebrafish, like all vertebrates, possess the full repertoire of immune cells and responses, little was done with that knowledge until recently, given the perceived benefits of the C57BL/6 model, which more closely resembles some aspects of the human immune system and has a superabundance of useful genetic tools with which to study immune responses in cancer, inflammation, autoimmunity, and infection. 
The mouse has served as the model for immunology for the past 50 years. It has enables monumental discoveries that have resulted in new medications and therapies to treat nearly every conceivable human disease and is the foundation of every single chemotherapeutic medicine on the market today. The diminutive mouse is an outstanding model for a vast array of human diseases and continues to be the go-to model for many processes. However, classical inbred mouse models, including C57BL/6 and other popular lines, including BALB/c, A/J, and 129S1, fail to replicate defining characteristics of tuberculosis in ways that compromise our ability to apply findings from these models to the kinetics and pathology of human disease. 

2.1a Zebrafish and their History in Developmental Biology
2.1b Modern Applications of Laboratory Zebrafish
2.1c Zebrafish as a Model System for the Study of Immunity
2.1d Challenges in the Use of the Zebrafish Model
2.2 Mycobacterium marinum-Zebrafish Model of Tuberculosis

Other popular laboratory models of tuberculosis are able to form granulomas, including rabbits and guinea pigs; the former is highly resistant to tuberculosis while the latter is highly susceptible. However, these tend to require maintenance via outbreeding, are larger mammals with associated higher husbandry costs, and are devoid of most useful genetic tools. This left a clear gap in our ability to understand some of the aspects of this important human disease that required innovative new approaches and a whole new paradigm. 

A foundational study in 2002 set the tone for the next two decades of research into host-microbe interactions in the zebrafish. Davis & Ramakrishnan took advantage of the optical transparency and manipulative amenability of the zebrafish larvae to infect them with an aquatic pathogen in the Mycobacterium genus – Mycobacterium marinum. M. marinum is a globally dispersed pathogen of fish and amphibians that causes tuberculosis in fish, which tends of manifest in superficial lesions, spinal deformities, and wasting. The use of this heterologous host-pathogen system allowed for the first ever in vivo visualization of the early processes of granuloma formation through the interactions between the invading bacteria and the responding host macrophages, which serve as the first responding innate immune cells to mycobacterial infections. 

Further developments over the following years, most notably by Swain et al. in 2006, established the zebrafish as a sophisticated and multifaceted model that allows for both comprehensive live imaging of the early processes of infection and dissection of the later stages of infection using adult zebrafish that form granulomas morphologically similar to those formed by humans in response to both M. tuberculosis and during opportunistic infections by M. marinum. These findings set the stage for the continued development of the zebrafish-M. marinum model of tuberculosis and has enabled the study of processes of human disease that have been long described but previously unable to be evaluated.

2.2a Relevance and Natural History of Mycobacterium marinum
2.2b Deficits of Mouse Models of Tuberculosis


\end{doublespace}

\chapter{Third Chapter}
\pagenumbering{arabic}

% The \vspace{} command in this chapter is just for aesthetic reasons - I don't like something new to start at the last line 
%of the page

% ONE OF THE BEST ONLINE LATEX REFERENCES IS AT :
% http://www.eng.cam.ac.uk/help/tpl/textprocessing/latex_advanced/latex_advanced.html

%% ALL figures are in EPS format: It is the best possible format 

\section{Summary}
During mycobacterial infections, pathogenic mycobacteria manipulate both host immune and stromal cells to establish and maintain a productive infection. In humans, non-human primates, and zebrafish models of infection, pathogenic mycobacteria produce and modify the specialized lipid trehalose 6-6’-dimycolate (TDM) in the bacterial cell envelope to generate host angiogenesis at the site of forming granulomas, leading to enhanced bacterial growth. Here we use the zebrafish-Mycobacterium marinum infection model to define the signaling basis of the host angiogenic response. Through intravital imaging and targeted, cell-specific peptide-mediated inhibition, we identify macrophage-specific activation of NFAT signaling as essential to TDM-mediated angiogenesis in vivo.  Exposure of human cells to Mycobacterium tuberculosis results in robust induction of VEGF that is dependent on a signaling pathway downstream of host TDM detection and culminates in NFATC2 activation. As granuloma-associated angiogenesis is known to serve bacterial-beneficial roles, these findings identify potential host targets to improve tuberculosis disease outcomes.

\section{Introduction}

The host immune response to infection is driven by an intricately regulated, but occasionally discordant or maladaptive, immune response to pathogenic stimuli at the cell-intrinsic, innate, and adaptive levels (Iwasaki and Medzhitov, 2010). While the contributions of immune cells have been widely studied, there is growing appreciation that non-immune populations such as stromal cells and the endothelium (Honan and Chen, 2021; Worrell and MacLeod, 2021; Amersfoort et al., 2022) are also crucial in shaping the host response in both acute and chronic infections (Mueller and Germain, 2009; Randow et al., 2013; Krishnamurty and Turley, 2020). Just as pathogens have evolved sophisticated mechanisms to hijack signaling pathways in immune cells (Finlay and McFadden, 2006), they have also been shown to manipulate developmental and homeostatic processes to direct them toward pathogen-beneficial host responses (Menzies and Kourteva, 1998; Guichard et al., 2013).

Mycobacterium tuberculosis (Mtb) is among history’s most widespread and successful pathogens. It has evolved a range of sophisticated mechanisms to manipulate its human host in order to survive, replicate, and transmit. Upon infection, Mtb induces an intricate immune response wherein innate immune cells, consisting initially of macrophages, congregate at the bacterial focus and then undergo an epithelioid transformation and interdigitate to form an encased granuloma, the hallmark feature of tuberculosis (TB), which provides both the replicative niche and the major host-pathogen interface of TB disease (Cronan et al., 2016; Pagan and Ramakrishnan, 2018; Cronan et al., 2021). Granuloma-associated vasculature has long been noted in human and animal models of TB (Cudkowicz, 1952; Russell et al., 2010) but the mechanisms of induction and precise contributions to infection are not yet fully understood.

Many of the major pathological features of mycobacterial granulomas, including associated vascularization, are conserved from zebrafish to humans (Swaim et al., 2006; Bohrer et al., 2021). Zebrafish can be infected with a natural pathogen, Mycobacterium marinum, which induces a robust angiogenic response during granuloma formation. This process, much like that in humans, non-human primates, and rabbits, is associated with production of the pro-angiogenic chemokine, Vegfaa, at the site of infection (Oehlers et al., 2015). This chemokine is a critical regulator of angiogenesis in both developmental and pathological contexts. Similarly, human granulomas have been shown to express VEGFA and are physically associated with blood vessels that penetrate the outer granulomatous layers (Datta et al., 2015). Subsequent work has demonstrated a role for these vessels in supporting bacterial growth and in dissemination of the bacilli from their primary site of infection (Polena et al., 2016). Recent profiling of human and non-human primate granulomas have confirmed the presence of aberrant vasculature associated with Mtb granulomas (Gideon et al., 2022; McCaffrey et al., 2022) during a non-canonical type II immune response (Cronan et al., 2021).

Pathogenic mycobacteria have evolved specialized mechanisms to promote and accelerate angiogenesis. Notably, the extensively modified and essential outer cell envelope component trehalose 6-6’-dimycolate (TDM) is cis-cyclopropanated by the enzyme PcaA (Glickman et al., 2000). Mutation of pcaA results in a reduction in granuloma angiogenesis and reduction in bacterial burden; correspondingly, cyclopropanated TDM alone is sufficient to induce host angiogenesis (Saita et al., 2000; Sakaguchi et al., 2000; Walton et al., 2018). As pcaA-dependent vascularization supports bacterial growth, factors driving this represent potential sites of therapeutic intervention yet the signals that mediate this host process remain unclear.

TDM is an extraordinarily long-chain, hydrophobic (C60-C90) glycolipid that has been shown to be detected in cell culture and murine models by host C-type lectin receptors, most notably MCL (CLEC4D) and MINCLE (CLEC4E), as well as by Toll-like receptor 2 (TLR2) and MARCO (Bowdish et al., 2009; Matsunaga and Moody, 2009; Miyake et al., 2013). Canonically, C-type lectin signaling is transmitted through a CARD9-NF-κB signaling pathway that results in the transcription and production of TNFα, IL-1β, IL-6 and other cytokines (Yamasaki et al., 2008; Goodridge et al., 2009; Lobato-Pascual et al., 2013; Zhao et al., 2014; Deerhake et al., 2021). However, beyond CARD9, a number of other downstream signaling pathways are engaged by C-type lectin activation and likely control discrete aspects of lectin signaling (Goodridge et al., 2007; Deerhake et al., 2021).

Here, we synthesize findings from zebrafish and cell culture models to define the in vivo angiogenic response induced by pathogenic mycobacteria. Contrary to classical models of C-type lectin signaling, we find that cis-cyclopropanated TDM exerts its pro-angiogenic effects through an alternative NFAT-driven pathway rather than canonical CARD9-NF-κB signaling. We use peptide-mediated, cell-specific inhibition of NFAT to demonstrate that both early and mature granuloma angiogenesis are dependent upon macrophage-NFAT signaling. We identify Nfatc2a as the predominant isoform mediating vegfaa induction and angiogenesis. These findings define the basis of granuloma-associated angiogenesis during pathogenic mycobacterial infections and suggest new targets for host-directed therapeutic interventions during tuberculosis.

\section{Results}

\subsection{Macrophage Induction of \textit{vegfaa} and Angiogenesis during Mycobacterial Infection}

Injection of live Mycobacterium marinum into the dorsal trunk of the zebrafish larva is sufficient to induce a robust angiogenic response adjacent to nascent granulomas in a macrophage-dependent manner (Oehlers et al., 2015) (Fig. 1A). The stereotyped vasculature along this region of the larva allows facile quantitation of neovascularization during and after granuloma formation or other insult (Lawson and Weinstein, 2002; Jin et al., 2005; Gore et al., 2012; Matsuoka and Stainier, 2018). We demonstrated previously that cis-cyclopropanated trehalose 6-6’-dimycolate (TDM) is required for the induction of vegfaa and angiogenesis at the site of infection. Furthermore, we found that genetic blockade of Vegfaa signaling was sufficient to abolish angiogenesis during infection with wildtype mycobacteria (Walton et al., 2018). Taken together, these findings suggest that the failure to induce vegfaa is a major contributor to the loss of angiogenesis in pcaA-deficient granulomas.

To study this phenomenon further, we began by examining the kinetics of vegfaa induction to identify the cellular source of vegfaa during granuloma formation. To test whether macrophages were a significant source of vegfaa, we developed a macrophage-specific reporter using the previously described acod1 promoter (also known as irg1), Tg(irg1:tdTomatoxt40) (from here, irg1:tdTomato). irg1 has been found to be expressed specifically in zebrafish macrophages and is upregulated during infection (Sanderson et al., 2015; Kwon et al., 2022). We then crossed this line with the vegfaa reporter line TgBAC(vegfaa:eGFPpd260) (vegfaa:eGFP throughout) and infected double transgenic irg1:tdTomato; vegfaa:eGFP progeny with M. marinum expressing eBFP2 (Mm-eBFP2) to simultaneously visualize bacteria, macrophage localization, and vegfaa production in vivo (Takaki et al., 2013).

We began imaging at a time point that preceded robust induction of vegfaa:eGFP but would allow us to capture the maximum time span of these events. We observed an increase in vegfaa reporter signal over time that appeared largely localized to macrophages (Fig. 1B). We observed that bacteria initially grew primarily intracellularly within individual macrophages at 36 hours post infection but began to grow in characteristic extracellular cords by approximately 84 hours post infection with little to no intracellular containment at this site by 96 hours post infection (Fig. 1C). The increase in extracellular growth coincided with the induction of eGFP signal in macrophages at ~64 hours (Fig. 1B), suggesting that, at low overall burden, intracellular detection is unable to induce vegfaa expression while extracellular engagement correlates with vegfaa expression during early stages of granuloma formation (Fig. 1B; Supplemental Movie 1). 

We next visualized the production of angiogenic vessels throughout infection in parallel to our characterization of vegfaa induction. Due to an inability to separate discrete emission wavelengths using two GFP reporter lines, we were unable to examine all four components (bacteria, vegfaa induction, macrophages, and vasculature) simultaneously. To relate this process directly to the angiogenesis observed in mycobacterial granulomas, we crossed the irg1:tdTomato macrophage reporter to the Tg(kdrl:eGFPs843) (from here, kdrl:eGFP) line, which labels vasculature (irg1:tdTomato; kdrl:eGFP). Under the same conditions and burden at which we infected the vegfaa and macrophage dual reporter line, we observed robust vascularization at approximately 96 hours post-infection, subsequent to initial granuloma formation and vegfaa induction (Fig. 1C, 1D, 1E, Supplemental Movie 2). 

\subsection{Genetic \textit{card9} Deficiency Does Not Compromise Mycobacteria-Induced Angiogenesis}

Given these observations suggesting that macrophages engaging extracellular bacteria are an important source of vegfaa expression, we interrogated pattern recognition receptor (PRR) signaling pathways that had been implicated in host responses to TDM, a major external component of the mycobacterial cell envelope. We had previously found that myd88 was dispensable for the induction of angiogenesis in response to TDM in vivo (Bowdish et al., 2009; Walton et al., 2018). This suggested that the described TLR2-mediated responses that function downstream of TDM detection in some contexts were unlikely to be required for this process. Rather, we found that the FcRγ homologs in zebrafish, fcer1g and fcer1gl, are required for the full angiogenic response to TDM (Walton et al., 2018), implicating C-type lectin receptors signaling in mediating this response (Richardson and Williams, 2014; Zhao et al., 2014).

As many of the downstream activities of C-type lectin receptors have been ascribed to the activation of CARD9-NF-κB signaling (Goodridge et al., 2009; Lobato-Pascual et al., 2013; Zhao et al., 2014; Williams, 2017; Deerhake et al., 2021), we assessed what role this pathway might play in angiogenesis during mycobacterial infection. We developed a card9 knockout zebrafish line using CRISPR/Cas9 that carries a 28 bp insertion, resulting in an early stop after 59 amino acids (card9xt31) (Supp Fig. 1A). We then assayed these animals in the kdrl:eGFP transgenic background by incrossing kdrl:eGFP; card9xt31/+ animals and infecting the resulting offspring with tdTomato-fluorescent M. marinum (Mm-tdTomato) at 2 days post fertilization (dpf) (Jin et al., 2005; Oehlers et al., 2015) (Fig. 2A, 2B). We quantitated the resulting aberrant vasculature at 4 days post-infection (dpi) under genotypic blinding and post hoc matched these measurements to genotype. There were no significant differences between the three genotypes (Supp. Fig. 1B, 1C), suggesting either redundancy between multiple established pathways or the existence of an alternative pathway downstream of TDM detection that was fcer1g/fcer1gl-dependent, but independent of both myd88- and card9. 

\subsection{Pharmacological Inhibition of NFAT Induction Limits Mycobacteria-Induced Angiogenesis}

Although many of the physiological consequences of C-type lectin receptor induction are often ascribed to CARD9-NF-κB signaling, this PRR class is also known to activate a distinct transcription factor family with known roles in immunity – the nuclear factor of activated T cells, or NFAT (Goodridge et al., 2007; Deerhake et al., 2021). This calcium-responsive transcription factor pathway is best described in its role regulating T cell biology, but there are numerous reports describing various roles for the members of this pathway in other cell types, including macrophages (Symes et al., 1998; Jones et al., 2000; Crabtree and Olson, 2002; Horsley and Pavlath, 2002; Elloumi et al., 2012). Given that there are four mammalian members of this pathway and six zebrafish homologs with potentially overlapping functions, we began with a pharmacological approach to globally inhibit NFAT signaling through all six zebrafish isoforms. 

We first infected 2 dpf kdrl:eGFP larval zebrafish with Mm-tdTomato in the trunk and treated them with 125 nM FK506, a clinically utilized calcineurin inhibitor that blocks NFAT activation, for the duration of the experiment. This modest dose of FK506 was chosen due to developmental toxicities observed at higher doses. We imaged them at 4 dpi and quantitated the degree of vasculature induced in the presence and absence of inhibitor. Even with a low dose of FK506, we noted a small, but statistically significant reduction in the mean degree of neovascularization at this time point, consistent with a role for NFAT in controlling angiogenesis in response to M. marinum infection (Fig. 2C, 2D) (Kujawski et al., 2014). To ask whether this effect was specific to recognition of TDM, we injected purified TDM or vehicle (incomplete Freund’s adjuvant; IFA) alone into the trunks of 2 dpf larvae. Treatment with FK506 resulted in a statistically significant reduction in the degree of angiogenesis induced at 2 days post-injection (Fig. 2E, 2F), suggesting that this pathway was relevant specifically to TDM-dependent angiogenesis.

\subsection{The Isoform NFATC2 is Specifically Required for Mycobacteria-Induced Angiogenesis}

Combining our observations on the correspondence of granuloma formation and the induction of vegfaa with our data implicating the NFAT pathway, we sought to identify NFAT isoforms that were enriched in granuloma macrophage populations. Aside from investigations made into nfatc1, which is restricted to the endocardium, lymphatic vessels, and the notochord during much of zebrafish development (Pestel et al., 2016; Shin et al., 2019; Bagwell et al., 2020), little is known of the expression patterns of these genes in zebrafish, especially in the context of infection. We first made use of published scRNA-seq datasets from mycobacterial granulomas in zebrafish and non-human primates for nfat transcripts that were robustly expressed in granuloma macrophages and identified both zebrafish nfatc2a and nfatc3a as plausible candidates (Cronan et al., 2021; Gideon et al., 2022).

To examine potential roles for nfatc2a and nfatc3a in granuloma-associated angiogenesis in vivo, we first screened F0 CRISPR-injected mosaic knockouts (“crispants”) to rapidly evaluate these candidate genes. Using this approach, similar to that used previously by other groups, we assessed the relative roles of these two isoforms individually and in tandem, measuring the angiogenic response to mycobacterial infection in the kdrl:eGFP background (Jao et al., 2013; Wu et al., 2018; Hoshijima et al., 2019; Kroll et al., 2021). We found that nfatc2a inhibition resulted in a ~50-80% reduction in angiogenesis. In contrast, nfatc3a had no effect on the length of ectopic blood vessels present. The dual targeted double mosaics were statistically indistinguishable from the nfatc2a injected fish alone (Fig. 3A, Supp. Fig. 2A, 2B). This allowed us to prospectively identify nfatc2a as an NFAT isoform required for full angiogenic response to mycobacteria while nfatc3a, despite expression in overlapping cell populations, appeared to be entirely dispensable for this process at this timepoint (Fig. 3A; Supp. Fig. 2A, 2B). 

We then established stable, germline transmitting indel mutant alleles for both genes to validate our results from mosaic animals. Recapitulating our results in the F0 generation, the nfatc3axt59 mutation carrying a 22 bp deletion (leading to an early stop codon at amino acid 9 in exon 1) had no effect on angiogenesis at 4 dpi (Fig. 3B, Supp. Fig. 2C). We then developed a knockout line of nfatc2a bearing a net 4 bp insertion leading to an early stop codon in the second exon (at amino acid 273, frameshifted after amino acid 247), prior to the DNA-binding domain (nfatc2axt69) (Supp. Fig. 2D). We repeated our angiogenesis assay using larvae from incrosses of kdrl:eGFP; nfatc2axt69/+ animals that produced expected Mendelian ratios of wildtype, heterozygous, and homozygous mutant offspring. Consistent with the results from mosaic animals, homozygous knockout of nfatc2a was sufficient to reduce the degree of angiogenesis present in larval zebrafish at 4 dpi (Fig. 3C, 3D; Supp. Fig. 2E, 2F). Importantly, given the known role of NFAT isoforms in T cell function, these defects emerged prior to the developmental emergence of functional T cells (Trede et al., 2004). However, whole animal knockouts could not address potential roles for other cell types in mediating this process. 

\subsection{NFAT is Essential for Angiogenesis Induction in vivo in a Macrophage-Specific Manner}

Given our observations on vegfaa induction in macrophages at the granuloma, we tested whether NFAT signaling was required specifically in macrophages for granuloma-associated angiogenesis. For in vivo inhibition of macrophage NFAT signaling during infection, we applied an approach that takes advantage of the NFAT-inhibitory peptide, VIVIT, which competitively inhibits calcineurin-dependent activation of all the NFATc isoforms (Aramburu et al., 1999). This approach has been successfully used as an exogenous treatment in cell culture (Deerhake et al., 2021) and mice (Noguchi et al., 2004; Elloumi et al., 2012; Rojanathammanee et al., 2015), through ectopic overexpression in cell culture (McCullagh et al., 2004),  and, more recently, in mice (Poli et al., 2022). We developed a transgenic zebrafish line in which VIVIT is expressed specifically in macrophages, Tg(irg1:VIVIT-tdTomatoxt38) (from here, simply irg1:VIVIT) (Fig. 4A, 4B) (Sanderson et al., 2015). We assessed whether the macrophage-specific expression of VIVIT would be sufficient to reduce the degree of angiogenesis during infection in the trunk with wildtype M. marinum expressing mCerulean (Mm-mCerulean). We found that macrophage-specific VIVIT expression significantly reduced angiogenesis in response to infection (Fig. 4C, Supp. Fig. 2G, 2H). This suggested a macrophage-specific role for NFAT signaling downstream of mycobacterial detection that was necessary to induce angiogenesis, presumably through the nfatc2a isoform.

To ask more directly whether the decreased angiogenesis observed in the NFAT-deficient macrophages was via the TDM-mediated pathway, we used the TDM injection assay we had developed previously. We injected TDM or the IFA vehicle  into the trunk of 2 dpf larval zebrafish (Fig. 4D) and measured the resulting angiogenesis at 2 dpi (Walton et al., 2018). TDM was sufficient to induce angiogenesis in vivo and this effect was dependent upon functional NFAT signaling, with the degree of TDM-induced angiogenesis reduced to the level of the vehicle alone in irg1:VIVIT animals compared to irg1:tdTomato controls (Fig. 4E, Supp. Fig. 2I, 2J).

\subsection{NFAT Activation is Essential for Angiogenesis in Adult Granulomas}

Adult zebrafish are equipped with both innate and adaptive immunity and form mycobacterial granulomas that histologically mirror epithelioid human tuberculosis granulomas (Swaim et al., 2006), including induction of a surrounding vascular network. To assess whether our findings in the larvae translated to a longer-term context in the presence of adaptive immunity, we infected adult kdrl:eGFP; nfatc2axt69/xt69 zebrafish and kdrl:eGFP; nfatc2a+/+ siblings with Mm-tdTomato and examined their peritoneal organs at 18 dpi after CLARITY-based clearing (Chung and Deisseroth, 2013; Cronan et al., 2015). Cleared organs were then imaged by spinning disk confocal microscopy (Fig. 5A). We measured the total vascular network surrounding the granulomas in a programmatically blinded fashion (Salter, 2016) and found that nfatc2axt69/xt69 fish had a significant reduction (~50%) in the length of the vascular network compared to wildtype siblings, further validating this gene as important for the angiogenic response in vivo (Fig. 5B, 5C; Supp. Fig. 3A, 3B). These putatively neovascular vessels tend to be highly branched and to be comprised of a limited number of cells with small or non-existent luminal volume, indicating that they are still in the sprouting stage of angiogenesis and suggesting a potential failure to mature. We observed robust effects that are likely understated in our quantitation, as we could not formally make any distinction between thicker, existing vasculature present at baseline that happens to fall nearby the granuloma and the characteristic neovascularization more intimately associated with the granuloma and present in wildtype but reduced in nfatc2a mutants (Fig. 5A; Supp. Fig. 3C).

\subsection{Macrophage-specific NFAT Inhibition in Mature Granulomas Reduces Angiogenesis}

We next evaluated whether macrophage-specific NFAT inhibition had similar effects on vascularization in adult zebrafish. We infected adult irg1:VIVIT; kdrl:eGFP and irg1:tdTomato; kdrl:eGFP double transgenic zebrafish with Mm-mCerulean and examined visceral organs at 14 dpi. We used confocal imaging to visualize individual CLARITY-cleared organs and measured the total length of granuloma-proximal vasculature under blinding as above (Salter, 2016). We found that the degree of vascularization was significantly reduced around granulomas from irg1:VIVIT fish as compared to irg1:tdTomato fish (Fig. 5D, 5E, Supp. Fig. 3D, 3E). The extent of the vascular network in the irg1:VIVIT condition was notably restricted in cases or solely comprised of more mature, luminal vessels, suggesting a total failure to induce an angiogenic response (Fig. 5D). These findings, consistent with our previous data from both larval zebrafish infections in the irg1:VIVIT background and in the nfatc2a mutant adult fish, point to a critical role for macrophage-specific NFAT activation in inducing the angiogenic response at mycobacterial granulomas. Furthermore, this establishes that NFAT function is broadly conserved from early larval infection through to the mature necrotic granulomas that characterize adult infection.

\subsection{Inhibition of NFAT Signaling Results in Decreased Bacterial Burden}

We had previously shown that inhibition of granuloma-associated vascularization is associated with decreased bacterial burden. Mycobacterial mutants unable to induce vascularization (ΔpcaA), and either genetic or pharmacological inhibition of VEGF signaling all result in lower bacterial burden, presumably due to functions of the aberrant vasculature promoting bacterial growth and/or inhibiting bacterial killing (Glickman et al. 2000; Oehlers et al. 2015; Walton et al. 2018). To examine the effect on burden of inhibition of NFAT signaling, we performed colony forming unit (CFU) assays at timepoints after the induction of angiogenesis and granuloma maturation. We infected nfatc2a+/+ and nfatc2axt69/xt69 adult zebrafish with Mm-tdTomato and plated them for CFU at 24 dpi. We found that knockout of nfatc2a resulted in a ~50% decrease in median colony number compared to wildtype after extended infection (Fig. 5F). 

Finally, we evaluated the impact of macrophage-specific NFAT inhibition on whole organism bacterial burden. We infected adult zebrafish possessing either the irg1:VIVIT or irg1:tdTomato transgenes with Mm-tdTomato and then homogenized and plated these fish at 18 dpi. We found that macrophage expression of the VIVIT peptide resulted in a median reduction of ~60% of the bacterial burden in these fish at this time point relatively soon after the formation of necrotic granulomas and robust induction of angiogenesis (Fig. 5G).
 
\subsection{Pharmacological Inhibition of NFAT in Human THP-1 Macrophages Limits VEGFA Induction by Mycobacterium tuberculosis}

The zebrafish mycobacterial infection model shares important conserved features with Mtb infection of humans, host response and granuloma angiogenesis (Swaim et al., 2006; Datta et al., 2015; Oehlers et al., 2015; Cronan et al., 2021). In addition, important aspects of the response to cyclopropanated TDM appears to be largely maintained between zebrafish and humans (Walton et al., 2018). We next asked whether our findings discovered in vivo with the zebrafish-M. marinum model were conserved in human cells exposed to Mtb. We developed a cell culture model of macrophage-Mtb interactions using differentiated THP-1 monocytic cells exposed to γ-irradiated Mycobacterium tuberculosis H37Rv (γMtb), which produces the full spectrum of TDM species, presented to the cell in their native configuration (as compared to heat-killed Mtb, which disrupts cell envelope structure and organization) (Romero et al., 2014; Secanella-Fandos et al., 2014) (Fig. 6A). We found that exposure of differentiated THP-1 macrophages to γMtb was sufficient to induce VEGFA transcription as well as VEGFA secretion (Fig. 6B, 6C). To examine whether NFAT signaling is required for production and secretion of VEGFA we treated THP-1 macrophages with the small molecule inhibitor INCA-6, which specifically disrupts the interaction between the NFAT family members and their activating phosphatase, calcineurin (Roehrl et al., 2004). Strikingly, treatment of THP1 cells with INCA-6 during γMtb exposure significantly inhibited transcriptional induction of VEGFA (Fig. 6B, Supp. Fig. 4A, 4B), as well as VEGFA secretion (Fig. 6C, Supp. Fig. 4C, 4D). Immunofluorescence revealed robust translocation of NFAT (using an NFATC2 antibody) that was broadly correlated to VEGFA signal (Fig. 6D). Taken together these experiments suggest that human NFAT signaling is required for VEGF production in response to Mtb exposure.

\subsection{Requirement of human NFATC2 for VEGFA induction}

To identify functionally important NFAT human isoforms, we exposed THP-1 macrophages to γMtb and subsequently used the secretion inhibitor brefeldin A to lock VEGF within secreting cells. Simultaneous staining for each of the four human NFATc proteins along with VEGFA allowed us to identify NFAT isoforms that underwent changes in expression and localization and correlate this with VEGFA production (Fig. 7A). While THP-1 macrophages express all of the isoforms to varying degrees, the most intense co-staining with VEGF was found with NFATC2 (Fig. 7B). Additionally, while each of the isoforms showed alterations after γMtb exposure, only NFATC2 showed robust nuclear localization that appeared to correspond to VEGFA induction in individual cells (Supp. Fig. 4E). While some NFAT isoform translocation was observable with at least NFATC1 and NFATC3, this generally had no correspondence to the degree or presence of VEGFA production. Given the strong correlation for NFATC2 with nuclear localization and VEGFA production after γMtb exposure, expression data from zebrafish and non-human primate granulomas, as well as the in vivo zebrafish results implicating macrophage Nfatc2a in Vegfaa production and angiogenesis, we focused on human NFATC2 as a key isoform.

To test a functional role for human NFATC2 in macrophage induction of VEGFA during γMtb exposure, we used a lentivirus-mediated CRISPR/Cas9 approach to introduce high-efficiency disruption of NFATC2. We compared these cells to those transduced with lentiviruses expressing safe-targeting control sgRNAs. (Supp. Fig. 4F-4H) (Kabadi et al., 2014; Sanjana et al., 2014; Morgens et al., 2017; Kitamura and Kaminuma, 2021). We simultaneously expressed four distinct guide RNAs targeting NFATC2 or safe-targeting controls, to maximize the percentage of puromycin-resistant cells possessing complete null mutations (Wu et al., 2018). Due to technical challenges associated with long-term culture of THP-1 cells and to address heterogeneity among cellular responses, we focused these assays on VEGFA induction in these cells by immunofluorescence after γMtb exposure. Because the N-terminal epitope recognized by our NFATC2 antibody was upstream of the targeted sites, we were unable to examine functional protein levels directly and simultaneously in the immunofluorescence images (Supp. Fig. 4I). However, we found that transduced cells targeted by NFATC2 lentivirus generally failed to induce VEGFA while safe-targeting control lentivirus-transduced cells responded normally (Fig. 7D, 7E). Thus, macrophage NFATC2-mediated induction of VEGFA downstream of mycobacterial TDM exposure is conserved from zebrafish to human cells exposed to M. tuberculosis.
 
\section{Discussion}

This work uncovers an unexpected role for macrophage NFAT activation in immune responses to pathogenic mycobacteria and the maladaptive angiogenic responses that occur during infection. This activation of NFAT is driven through recognition of bacterial  cyclopropanated trehalose 6-6’-dimycolate, a major constituent of the cell envelope in pathogenic mycobacteria, that we have previously found is necessary and sufficient to drive pathological angiogenesis (Walton et al., 2018). Identifying this unexpected role for NFAT in angiogenesis expands our understanding of the mechanisms governing mycobacterial pathogenesis and offers targets for potential host directed therapeutics. Traditionally, work on TDM-mediated C-type lectin activation has focused on CARD9 and NF-κB signaling. Here, in contrast, we describe a specific role for alternative C-type lectin signaling responses through the NFAT pathway to drive VEGFA production and granuloma-associated angiogenesis. 

VEGFA induction is a prominent feature of tuberculosis in human disease as well as in a number of animal models, including non-human primates, rabbits, mice, and zebrafish (Datta et al., 2015; Oehlers et al., 2015; Harding et al., 2019; Cronan et al., 2021; Gideon et al., 2022). We found that VEGF was produced specifically within newly arrived macrophages at nascent granulomas. Macrophage populations are critical to VEGF induction, as macrophage-specific inhibition of NFAT signaling as well as deletion of nfatc2a result in reductions in granuloma-associated angiogenesis. Using a human cell culture model, we found that NFATC2 was similarly engaged in human cells as amongst all NFAT isoforms, only NFATC2 underwent robust nuclear translocation in response to M. tuberculosis stimulation. Correspondingly, pharmacological inhibition of NFAT signaling in human cell culture as well as genetic inhibition of NFATC2 resulted in reduced VEGFA production.

Although animal models of tuberculosis generally report high VEGF levels, there are few studies that center on VEGFA induction in cell culture infection models. Through high-resolution time-lapses and reporter lines, we found that vegfaa induction generally does not occur until the formation of initial granulomas and generally correlated with the appearance of extracellular bacteria that could be recognized by incoming, likely uninfected macrophages. This concentration-dependent effect on signaling may reflect key aspects of the disease itself, wherein large masses of extracellular bacteria accumulate in the necrotic core of the granuloma, potentially triggering relatively insensitive and/or chronic C-type lectin signaling in this context.

Consistent with the recognition of extracellular bacteria, exposure of human macrophage-like cells to γ-irradiated M. tuberculosis rapidly induced VEGF signaling in a NFATC2-dependent manner in a dose dependent manner. Standard cell culture infection models generally eliminate extracellular bacteria using gentamicin treatment and media changes, and so it is possible that engagement of this pathway by extracellular bacteria or TDM stimulation is a key component of this response. A survey of the literature and a variety (Lee et al., 2019; Pisu et al., 2020; Hall et al., 2021; Looney et al., 2021; Pu et al., 2021) of RNA-seq datasets from macrophage-Mtb infection experiments reveal modest or nonexistent induction of VEGFA, further supporting the notion that extracellular exposure to Mtb may be an important element of the angiogenic response and may reflect some aspects of the macrophage-Mtb interface within granulomas.

As its name suggests, the NFAT pathway plays an indispensable role in normal T cell biology. Accordingly, whole animal knockouts of NFAT in standard mouse models of M. tuberculosis infection – where granuloma formation itself may be limited – may have obscured a role for myeloid-specific effects of NFAT signaling (Via et al., 2012). The zebrafish model, by looking at early timepoints, uncovered a role both in angiogenesis and, presumably as a consequence, bacterial control. Wholesale, longer-term inactivation of NFAT, which also plays important roles in T cells, would compromise important aspects of a productive adaptive immune response during mycobacterial infection. While genetically manipulable animal models allow for cell-specific separation, any host-based therapeutic approaches might require cell-specific macrophage delivery methods (Hu et al., 2019; Mukhtar et al., 2020), NFATC2-specific targeting (Kitamura and Kaminuma, 2021), and/or contend with the adaptive immune response, an important aspect of host resistance during mycobacterial infection.

It remains unclear why NFATC2, but not any of the other isoforms, is specifically required in macrophages for the induction of VEGFA, given evidence that the others are present in resting macrophages (Fig. 7A). The functional distinctions between the isoforms have long been of basic interest, but relatively few specific differences between them have been identified beyond basal regulation to provide tissue-specificity and more recent findings describing layers of kinetic regulation with isoform-specific stimulation thresholds, nuclear retention, and more (Lyakh et al., 1997; Rao et al., 1997; Kar et al., 2014; Kar and Parekh, 2015; Kar et al., 2016). These novel levels of regulation offer opportunities for uncovering new features of the cell biology of NFAT.

Here, we identify the unique requirement for this single isoform in macrophages to induce angiogenesis in response to mycobacterial infection. One hypothesis is that NFATC2 has binding partner(s) unique among NFAT isoforms required for its effect on the VEGFA promoter. Whether this is HIF1α (the canonical regulator of VEGFA) or one of the many previously described interacting partners is, as yet, unknown, but could be tested either in vitro or in vivo with genetic or chemical approaches. However, higher order regulatory mechanisms that result in the production of VEGF in the absence of overt hypoxia have been understudied and this work proposes at least one potentially generalizable mechanism whereby NFATC2 activation results in VEGFA transcriptional upregulation, a process that can be inhibited with chemical and genetic intervention. Despite the widespread presence of putative NFAT binding motifs (5’-GGAAA-3’) (Supp. Fig. 4J) in the proximal VEGFA promoter (Gearing et al., 2019), their influence on VEGFA transcription has been relatively unexplored as this specific effect is generally not seen in T cells or other cell types (Chang et al., 2004). NFAT involvement in the induction of a variety of cytokines is well-documented, but which, if any, are at play in the macrophage-Mtb interaction is a promising subject for future research. 

A more comprehensive characterization of NFAT-dependent innate immune responses has begun in recent years (Deerhake et al., 2021; Peuker et al., 2022; Poli et al., 2022), but this pathway has remained unstudied in the context of macrophage signaling during mycobacterial infection. Furthermore, this work draws a connection between the induction of calcium fluctuations – which can occur in response to many different developmental, homeostatic, and pathological stimuli, including to mycobacterial infection (Kusner and Barton, 2001; Jayachandran et al., 2007; Matty et al., 2019) – to the angiogenic response to that stimulation. Our identification of NFAT regulation of VEGFA offers a novel approach to both pro- and anti-angiogenic intervention in various pathological contexts.

\subsection{Limitations of the Study}

While we have identified interesting macrophage biology mediating an important host immune response during mycobacterial infection, there is no data as to whether this might translate to other disease contexts, especially those with a prominent role for C-type lectin signaling. Whether or not this mechanism is broadly generalizable is important to understanding key aspects of pro-angiogenic macrophage behavior. Additionally, we have validated important aspects of our observations in the zebrafish with a mammalian cell culture model, but subsequent studies may warrant further integration of mammalian models of tuberculosis infection where angiogenesis is present or human patient samples to better understand certain aspects of the underlying biology.


% An example of making lists of various kinds - This one gives black circles of certain size based on style files - 
% LaTeX manual will tell you how to put numbers or different symobols



\chapter{Fourth Chapter}
\pagenumbering{arabic}

% The \vspace{} command in this chapter is just for aesthetic reasons - I don't like something new to start at the last line 
%of the page

% ONE OF THE BEST ONLINE LATEX REFERENCES IS AT :
% http://www.eng.cam.ac.uk/help/tpl/textprocessing/latex_advanced/latex_advanced.html

%% ALL figures are in EPS format: It is the best possible format 

High-resolution, high-throughput microscopy has opened possibilities for biological analysis that were inconceivable only a few short years ago, but the methods by which to analyze these data remain largely lacking. While heroic efforts have been made to use both standard thresholding methods as well as newer machine learning-powered methods to simplify and automate image processing, these approaches are often somewhat limited the range of purposes for which they were designed. Additionally, many of the tasks that experimenters seek to do are both relatively simple and highly repetitive but they are often unaware of the means by which to cut down on manual processing time in order to economize their time and energies. These scripts are a combination of automation scripts that will process particular image types into smaller and more informative images (compression of Z-stacks, time-series, etc.) and those meant for analysis of image data (pixel intensity distribution across an image, bacterial burden within larval zebrafish). While none of these are of the caliber to open entirely new methods of analysis, I hope they are a catalyst for others in the zebrafish community at large to explore the potential for computational automation to save time and frustration in the process of analyzing often thousands of very large, data-rich images.

All of the scripts in their latest versions will be able to be found in perpetuity at \href{http://github.com/jaredbrewer/image-analysis}. A static version of these has been created at Zenodo (). Scripts at the end for RNA sequencing analysis are available at Zenodo (). 

\section{FIJI/ImageJ}\label{fiji}



\section{Maximum-Intensity Projection and Composite Image Generation}

The fundamental premise of the ImageJ macro system is to simplify repetitive tasks to free up user time for higher forms of analysis, but the default language (the ImageJ1 macro language) make it easy to write procedural operations to be done on single images, but is difficult to scale to whole directories of images or accept various types of user input. I have thus developed a set of scripts that allow the user to rapidly generate maximum-intensity projections from sets of images and then generate composite images. This often condenses hundreds of megabytes of data into $<$30 MB, allows more flexible image viewing and understanding, and allows the images to be opened in essentially any number (many hundreds to thousands) on any personal computer. These procedures are also internally memory managed, allowing them to run on most personal computers indefinitely to process the sometimes thousands or tens of thousands of images that can be generated over the course of an experiment with multiple wavelengths, Z stacks, XY positions, and times. In our lab, we primarily use output files from epifluorescent Zeiss microscopes, which generate .czi files and from Metamorph connected to a custom spinning disk confocal system, which generates .tiff files.

The maximum intensity projection is an extremely common way of condensing multidimensional images into a single two-dimension representation by finding the brightest pixel at every XY position and accepting it as the most ``in focus'' pixel. The success of this depends on images being properly exposed, but in most instances will generate a reasonably sharp image ready to be quantitated or presented.

\begin{code}
\caption{This script allows the user to open as many files as their memory allotment will allow and then to Z project them one at a time with custom start and end positions. This ability often generates cleaner, sharper images by individually selecting the lowest and highest in-focus frames, but necessarily takes more time than a more automated approach.}
\label{slowmanmip}

\inputminted[breaklines,frame=single]{python}{source/manMIPper.py}

\end{code}

The use of \autoref{slowmanmip} is to process a set of already opened images and generate maximum intensity projections from these and, optionally, save them back into the directory that they came from. This relatively simple set of GUI-guided processes allows the user to, in two clicks, accomplish a task that previously would have required a great deal of menuing to accomplish. The goal is narrow, but this execution is extremely useful when working through large numbers of images.

\begin{code}
\caption{A low overhead version of the manual maximum intensity projection script described above. Instead of opening all of the images first and then running the script, the script will processively open unanalyzed images one at a time and periodically garbage collect, allowing for entire directories to be processed at once on most reasonably modern computers.}
\label{fastmanmip}

\inputminted[breaklines,frame=single]{python}{source/fast_manMIPper.py}

\end{code}

\autoref{fastmanmip}, while not always the correct choice depending on user preferences and system capabilities, is much less memory hungry than the original version above, but typically results in slightly slower overall operations due to the delay in opening images. If the images are on fast internal storage, then this is certain to be faster than \autoref{slowmanmip}, but when reading from external storage, the I/O limitations will likely make it faster to open all of the images prior to processing unless system memory is severely limited. Nevertheless, the nature of this makes it very generally useful on older systems.

\begin{code}
\caption{This script can be used in instances where the first and last stacks of a desired Z projection span the entire set of stacks provided. It will process an entire directory of images together and output the result into a subdirectory of the original.}
\label{bulkmip}

\inputminted[breaklines,frame=single]{python}{source/bulkMIPper.py}

\end{code}

While the previous scripts have expected user input for each image, the skilled microscopist can select top and bottom slices that will suffice for generating Z projections during imaging itself. This means that the first and last slices of the projection are typically the first and last slices of the images \textit{in toto}. This script takes a directory of images as an input and will perform total maximum intensity projections on all of them and save in a new subdirectory. After observation, any that seem incorrect can then be processed with one of the preceding scripts, especially \autoref{slowmanmip}. 

\begin{code}
\caption{An interface to functions allowing slices in a Z-stack to be kept or removed as desired through function calls. This can integrate into other workflows and be connected to the previous scripts through higher-order wrappers.}
\label{reslicer}

\inputminted[breaklines,frame=single]{python}{source/reSlicer.py}

\end{code}

FIJI/ImageJ comes with a built-in option to keep and remove particular slices, but the native plugin does not readily fit into object-oriented programming pipelines like those used by Python and Java. Thus, I have adapted the underlying logic of these plugins to be wrapped in various other scripts. For instance, this allows calls to the maximum intensity projection plugins to be funneled into this plugin to simultaneously generate the kept slices as well as the maximum intensity projection of those kept slices for some useful improvements to data integrity. 

\section[Surface Plot Analysis for Cellular Distribution of Labeled Proteins]{Surface Plot Analysis for Cellular Distribution of Labeled Proteins\footnote{Taken from personal contributions to \citet{Saelens2022}.}}

\begin{code}
\caption{A script to conduct computational filename blinding from the command line written in Python.}
\label{blinder}

\inputminted[breaklines,frame=single]{python}{source/renamer.py}

\end{code}

\begin{code}
\caption{A script to conduct computational filename blinding from the command line written in Python.}
\label{blinder}

\inputminted[breaklines,frame=single]{python}{source/renamer.py}

\end{code}

\begin{code}
\caption{A script to conduct computational filename blinding from the command line written in Python.}
\label{blinder}

\inputminted[breaklines,frame=single]{python}{source/renamer.py}

\end{code}


\section{py-LaRoMe}\label{larome}

\begin{code}
\caption{A Python translation of the FIJI function ``Label image to ROIs'' from LaRoMe. This function allows the user to take images generated from CellProfiler and convert them into a set of regions of interest in the ROI Manager.}
\label{l2r}

\inputminted[breaklines,frame=single]{python}{source/labelsToROIs.py}

\end{code}

\begin{code}
\caption{A Python translation of the FIJI function ``ROIs to label image'' from LaRoMe. This allows the user to use a set of ROIs to regenerate a label image, useful for creating masks on existing images and comparing areas between different channels.}
\label{r2l}

\inputminted[breaklines,frame=single]{python}{source/ROIsTolabels.py}

\end{code}

\begin{code}
\caption{A Python translation of the FIJI function ``ROIs to Measurement Image''. This combines the a defined set of ROIs (probably from labelsToROIs.py) and a raw image and generates an image that graphically represents measurements such as area or circularity.}
\label{r2m}

\inputminted[breaklines,frame=single]{python}{source/ROIsToMap}

\end{code}


\section[Experimental Blinding via a Single-Click Command Line Interface]{Experimental Blinding via a Single-Click Command Line Interface\footnote{Implementation from \citet{Brewer2022}, original conception from \citet{Salter2016}.}}\label{blinders}

\begin{code}
\caption{A script to conduct computational filename blinding from the command line written in Python.}
\label{blinder}

\inputminted[breaklines,frame=single]{python}{source/renamer.py}

\end{code}

\section[User-Friendly Analysis of RNA Sequencing Data using Kallisto/Sleuth in a Python Environment]{User-Friendly Analysis of RNA Sequencing Data using Kallisto/Sleuth in a Python Environment\footnote{Taken and expanded from personal contributions to \citet{Saelens2022}.}}\label{rnaseq}

While an old technology today, the analysis of RNA sequencing data still unfairly remains a challenge for the technologically na\"{i}ve researcher. To ameliorate part of this problem, in the course of the work in \citet{Saelens2022}, I developed a set of pipelines for the analysis of RNAseq data using a combination of Kallisto and Sleuth, a pair of analysis and visualization applications that utilize pseudoalignment to calculate read counts and then display them in a Shiny application via R \citep{Pimentel2017}.

Conducting this portion of the work required acquainting myself with a number of commonly used computational tools, including cmake and a deeper knowledge of Python and how that can translate into generating a broadly useful cross-platform tool to analyze complex sequencing data. Doing so also required interfacing with FTP and other networking functions and navigating server directories to fetch reference cDNA from Ensembl.

While these were originally implemented as two parallel scripts, one for bacteria and the other for eukaryotes, they have since been consolidated into a single all-purpose script that requests different input based on what is available. The issue with bacterial analysis is that Ensembl has discontinued generating bioMarts for bacterial genomes and has adopted an unpredictable folder structure for fetching these files via FTP. Thus, the user will have to provide the reference transcriptome of their bacterial strain of choice, which can generally be acquired from species-specific databases (Mycobrowser being the notable one here) or from the set of available strains on NCBI or Ensembl. 

\begin{code}
\caption{A script to conduct computational filename blinding from the command line written in Python.}
\label{blinder}

\inputminted[breaklines,frame=single]{python}{source/renamer.py}

\end{code}

\begin{code}
\caption{A script to conduct computational filename blinding from the command line written in Python.}
\label{blinder}

\inputminted[breaklines,frame=single]{python}{source/renamer.py}

\end{code}

\begin{code}
\caption{A script to conduct computational filename blinding from the command line written in Python.}
\label{blinder}

\inputminted[breaklines,frame=single]{python}{source/renamer.py}

\end{code}

\section{Bacterial Burden Analysis by Fluorescence Intensity in a Semi-Automated Manner with a User-Friendly Graphical Interface}

One of the major routine tasks in the field of zebrafish-\textit{M. marinum} host-pathogen interactions is the quantitation of the total bacterial burden per larva. While it has been well established that the integrated fluorescence intensity of the image corresponds well to the colony forming units of bacteria present, the larval zebrafish has particular challenges. Many of these are attributable to autofluorescence from the yolk and pigment cells or the physical background of the imaging surface, but it is important to avoid catching these in the quantitation as these can vary greatly from fish to fish and are difficult to subtract \textit{post hoc}. However, through clever approaches to image pre-processing it is possible to eliminate these sources of misquantitation and streamline analysis to minimize user intervention.

\begin{code}
\caption{This graphical user interface allows for automatic background subtraction from images of \textit{M. marinum}-infected larval zebrafish and then quantitation of the remaining signal above a manually set threshold that captures as much of the true signal as possible.}
\label{burden}

\inputminted[breaklines,frame=single]{python}{source/burdenMeasurer.py}

\end{code}

This graphical user interface facilitates automatic processing of arbitrary numbers of images at once by allowing users to select various parameters to test for appropriateness in their particular experiment. The underlying logic will automatically create Z projections if applicable and then use those for subsequent analysis. Users are encouraged to select a subset of images to start and the computational thresholds are used to capture any objects over a certain size for background removal, which will typically only capture the yolk and any background fluorescence. This approach also allows for more generous manual thresholds to be selected for quantitation, an issue that often arises in manual approaches to fluorescence quantitation due to the need to accommodate this autofluorescence. 

It is my hope that this application, after further beta testing and refinement can supplant these manual approaches and replace them with something that free researcher time to conduct more experiments rather than spend many hours drawing circles around zebrafish in order to measure the bacterial burden of these fish. Additionally, this could in principle to adapted to measuring other aspects of zebrafish biology, from macrophage clustering to transcriptional reporter signals. The elegance of this approach is that it utilizes open-source and well-defined mechanisms for measuring signal over noise through the implementation of the automatic thresholds and wraps a set of utility functions within an interface that avoids the need for the end user to write burdensome macros to accomplish the same goal. This approach, after a minutes-long period of optimizing is able to measure the burden of thousands of larvae in mere minutes. The output can then be spot-checked for accuracy, as in a small subset of the larvae parts of the background may not be perfectly removed and these can then be reprocessed either manually or with altered parameters. 




\chapter{Fifth Chapter}
\pagenumbering{arabic}

% The \vspace{} command in this chapter is just for aesthetic reasons - I don't like something new to start at the last line 
%of the page

% ONE OF THE BEST ONLINE LATEX REFERENCES IS AT :
% http://www.eng.cam.ac.uk/help/tpl/textprocessing/latex_advanced/latex_advanced.html

%% ALL figures are in EPS format: It is the best possible format 

%Some ideas:

%\begin{itemize}
%
%\item The Zebrafish Mincle
%\item Sufficiency
%\item Cotranscriptional interplay
%\item Integration of HIF-1$\upalpha$ signaling
%\item Lymphangiogenesis
%\item Aspects that differentiate the isoforms in macrophages
%\item Effect of NFAT mutation on macrophage biology
%\item Mycobacterial interactions with other vasculature-relevant features -- plasmin, TIE2, fibronectin, etc.
%\item Role of NFAT in neutrophils
%\item New tools to study this pathway \textit{in vivo}
%\item Generalizability
%\item Novel Methods to Automate Measurement of Angiogenesis
%\item Novel Methods to Automate Cell Feature Quantitation
%\item Other Contributions of NFAT to Host-Mycobacterial Interactions
%\item Promise as a Host-Directed Therapy
%
%\end{itemize}

At the conclusion of the present work, many new questions have been generated while others remain unanswered. This work has accomplished two primary goals: addressing the intracellular signaling pathway within macrophages that is responsible for inducing angiogenesis during mycobacterial infection (the NFAT pathway) and setting the stage for future work to simplify and automate common procedures commonly used in the analysis of imaging data relevant to both zebrafish and tissue culture research. Some of these lingering questions will be addressed in the coming weeks and months while others will stretch over the course of many years or decades as we delve into deeper and deeper understandings of the fundamental processes governing the nature of the angiogenic response to tuberculosis infection and how and when this can be a fruitful target for therapeutic intervention. 

This leaves a set of important questions, pertinent to model development, deeper understanding of the biology of NFAT within (granuloma) macrophages, the intersections between this pathway and other, known pathways involved in angiogenic responses, and the future of imaging analysis in the context of ever-growing computational power. 

\begin{itemize}
\item What is the TDM receptor in zebrafish and do they have an as-yet unannotated MINCLE homolog? 
\item How or why is NFATC2 special and is it sufficient to induce VEGF? 
\item How does the NFAT pathway alter other aspects of macrophage behavior potentially relevant to tuberculosis biology and does this pathway intersect with HIF-1$\upalpha$ signaling? 
\item Aside from TDM, do mycobacteria have other mechanisms for manipulating the host angiogenic response and, if so, what are they and how does that enhance our overall understanding of this process? 
\item Are our findings on the nature of NFATC2 in inducing tuberculous angiogenesis relevant to other disease contexts where VEGF signaling plays an important role? 
\item Could NFAT offer a meaningful mechanism for inhibiting angiogenesis in the context of disease as a host-directed therapy?
\end{itemize}

These questions, among many others, are the subject of this concluding chapter; it is hoped that a comprehensive presentation of these questions will stimulate future generations to pursue answers and that these will further inform our understanding of the pathogenesis of tuberculosis toward the goal of eradicating this disease.

\section{The Zebrafish MINCLE}

As discussed in \autoref{tdmreceptor}, data from human cell culture and mice has implicated MCL and MINCLE as the primary C-type lectin receptors for TDM, which induces a variety of downstream responses, seemingly including the upregulation of VEGF and downstream angiogenesis (see \autoref{chap3}). However, the precise identity of the homolog of MCL or MINCLE in the zebrafish remains unknown. These two proteins arose from a tandem duplication and inversion at an unknown point in evolutionary history, although the two are ubiquitous across reptiles, birds, and mammals and present in at least some non-teleost fish, including the spotted gar \citep{Miyake2013, Richardson2014}. Given the strong, bidirectional selective pressure on both host and pathogen to modulate host PRR activity, these divergences are expected even between closely related species \citep{Rambaruth2015}. This diversification is especially notable among CLRs: mice have no fewer than eight putative DC-SIGN homologs and a great deal of work was done to narrow down the functional ones in order to model human disease \citep{GarciaVallejo2013}; on the other hand, the bovine homolog of MINCLE was readily identifiable but had diverged in non-critical domains from the human MINCLE \citep{Feinberg2013, Furukawa2013}. These aspects of structural diversity add unique complexities to the identification of any putative functional homolog in the fish, which may have substantially diverged from the ancestral protein as well as the mammalian versions. Despite these challenges, such identification would both substantially advance the zebrafish\textit{M. marinum} model and deepen our understanding of shared mechanisms of detection and response to C-type lectin receptor ligands.

Despite these challenges, there is an abundance of evidence that zebrafish possess an as-yet unidentified MINCLE (and/or MCL) homolog including, but not limited to: a long evolutionary history alongside pathogenic mycobacteria, the clear, \textit{myd88}-independent inflammatory response to purified TDM, and the \textit{in vivo} attenuation of mutants lacking fully mature TDM. Zebrafish have long been speculated to have functional homologs of other C-type lectin receptors despite difficulty in their identification \citep{Petit2019}. MINCLE has not been previously linked to angiogenesis, but the identification of this receptor in zebrafish would allow us to better understand the relevant pathway in humans; indeed, the role of MINCLE during infection is unclear given that some groups have demonstrated that it contributes to bacterial control while others have seen no effect \citep{Behler2012, Behler2015, Heitmann2013, Lee2012}. This has translational implications for modulating the activity of the human MINCLE to enrich for host-beneficial responses and also basic science implications in revealing the diversification of a receptor that maintains the ability to detect a common ligand. 

Unlike MINCLE, the basal receptor MCL has clear roles in mediating protection against mycobacteria \citep{Wilson2015}.

Although large amino acid segments of CLRs are able to undergo radical changes in primary sequence with few deleterious effects, there are several domains that have been identified as absolutely essential for binding to TDM. A specific set of criteria for this selection are listed in \autoref{minctab}(Alenton et al., 2017; Bird et al., 2018; Feinberg et al., 2013; Furukawa et al., 2013; Zelensky and Gready, 2005). Based on these criteria, we have identified three putative homologs with $>$50\% amino acid similarity to the human CLEC4E in the carbohydrate binding domain (\autoref{minctab}) and have identified transmitting nonsense mutations in each of them (\autoref{zfmincs}). As further evidence, two of these homologs (77975 and 79903) are organized in tandem, mirroring the genomic organization of MINCLE and MCL in mammals. 

\singlespacing
\begin{center}
\begin{longtable}{|>{\raggedright\arraybackslash}m{1.5in}|>{\raggedright\arraybackslash}m{4in}|}
\caption{Criteria used to select putative zebrafish homologs of the human MINCLE.}\label{minctab} \tabularnewline

\hline
\thead{Criteria} & \thead{Rationale} \tabularnewline
\hline
Possesses a gEPNn motif & Of the two major carbohydrate recognition domain motifs, the EPN motif is known to bind glucose-derived sugars while QPD motifs are known to bind galactose-derived sugars. As trehalose is a di-glucose and MINCLE and MCL both possess this EPN motif, this is an important first-pass selection criterion. \tabularnewline
\hline
Lacks an intracellular ITAM motif & In humans, both MINCLE and MCL use Fc$\upgamma$R to signal as they lack their own ITAM motif. While not an essential quality to detect and respond to TDM, this would strengthen the similarities between the two; we have also published data implicating Fc$\upgamma$R in the zebrafish, which argues in favor of this shared layer of similarity as well. \tabularnewline
\hline
Induced by infection & Using existing RNA-seq datasets, expression of these genes under inflammatory stimulus is an important indicator that they may be acting similarly to MINCLE, which is an inducible gene responsive to various inflammatory stimuli. \tabularnewline
\hline
Transmembrane helix & These surface receptors use a single-pass transmembrane helix to remain bound to the plasma membrane and transduce signals. \tabularnewline
\hline
Hydrophobic amino acids in the CRD & One of the defining biochemical features of MINCLE is a small hydrophobic pocket that appears to be useful for binding to the mycolate tails of TDM; the presence of such a pocket would be evocative of further similarity to MINCLE. \tabularnewline
\hline

\end{longtable}
\end{center}

\doublespacing

Multiple approaches can be taken to identify the capacity of these proteins to bind TDM and generate meaningful biological responses. Going forward, I propose to take a biochemistry-first approach to this question as it enables greater flexibility in responding to new data and starts with a foundation of known interactions. Thus, I will utilize established methods of TDM blotting \citep{Jegouzo2014} and wash recombinant carbohydrate recognition domain-streptavidin fusion protein lysate across them and then detect these interactions using standard biotin-horseradish peroxidase detection. This will provide a quantifiable readout for both presence/absence of an interaction but also the strength of the interaction. Should none of these proteins efficiently bind TDM, it is relatively trivial to generate new chimeric proteins and test a range of others present in the zebrafish genome. This data can then be used to go back into the zebrafish to assess the \textit{in vivo} consequences of this interaction and also allow for more flexibility in approach -- rather that seeking the receptor, we can explore phenotypes that may be altered in this context across both angiogenic responses and more general immune responses.

Additionally, it may be of some use to study the specific human MINCLE-mycobacteria interactions in the context of a whole immune system. Thus, going forward, it would be logical for future researchers to develop transgenic zebrafish that express human versions of MINCLE and MCL in macrophages and, perhaps, neutrophils, to assess the contributions of the human protein to conserved responses. This may also help to clarify some of the conflicting data in the literature around the role of MINCLE by using an overexpression model to capture the effect of excess MINCLE signaling. In the long term, gene replacement of the native MINCLE-like homolog with the human MINCLE would allow for a more authentically humanized model of macrophage biology within the zebrafish.

\singlespacing

\begin{center}
\begin{table}	
\caption{Putative zebrafish MINCLE homologs with key details about their native structure and mutants that have been generated thus far.}
\label{zfmincs} \tabularnewline
\vspace{0.5cm}
\begin{tabular}{|l|l|l|l|l|l|}
\hline
\thead{Gene ID} & \thead{Length (a.a.)} & \thead{CRD} & \thead{Similarity} & \thead{Mutation} & \thead{Site (a.a.)} \tabularnewline
\hline
56379 & 263 & kEPNn & 50.7\% & ::13 & 43 \tabularnewline
\hline
79903 & 263 & gEPNn & 53.3\% & ::13 & 207 \tabularnewline
\hline
77975 & 170 & gEPNn & 60.3\% & $\upDelta$8 & 108 \tabularnewline
\hline
\end{tabular}
\end{table}
\end{center}

\doublespacing

\section{Integration of Hypoxia Signaling}

The literature is replete with descriptions of HIF-1$\upalpha$ regulation of VEGFA production and signaling; the logical means by which to alleviate local hypoxia is through the recruitment of vasculature carrying oxygenated blood. This allows angiogenesis to occur when necessary and for the vasculature to remain quiescent under homeostatic conditions, where intravital oxygen concentrations are maintained at a high level. However, in areas of pathogen invasion, tumor growth, or tissue damage, the local oxygen concentration can fall, triggering the activation of the HIF-1$\upalpha$ signaling pathway. HIF-1$\upalpha$ is an oxygen-sensing protein that is expressed and rapidly degraded under normoxic conditions but stabilized under hypoxia. Under standard oxygen concentrations, two classes of regulatory proteins mediate prolylhydroxylation and proteosomal degradation of HIF-1$\upalpha$ in a process dependent on molecular oxygen.

Alternatively, HIF-1$\upalpha$ can be induced through alterations in the homeostatic stoichiometry of HIF-1$\upalpha$ itself and the two families of regulatory proteins, PHD and FIH. This allow HIF-1$\upalpha$ to be induced under normoxic conditions, the presumed whole-body state of the zebrafish larva. This normoxia activation has been found to be important for myeloid immune responses, including those downstream of MINCLE activation, and may play an important role especially in the early signaling events of mycobacterial infection \citep{Nishi2008, Schatz2016, Schoenen2014, Thompson2017}. 

In the environment of the granuloma, both the host and pathogen must adapt to reduced oxygen tension. Mycobacteria can temporarily revert into a non-replicating state known as persistence, but this is not a viable strategy for long-term evolutionary success \citep{Ehrt2018, Stewart2003, Manabe2000, Pandey2008, zuBentrup2001}. However, during persistence, the bacteria are extremely difficult to kill as most antitubercular drugs are only effective on replicating bacteria \citep{Veatch2018}. Even during active growth, the bacteria and associated host cells must alter their metabolism to accommodate for reduced oxygen availability, with important consequences for host immunity \citep{Harper2012, Tsai2006, Prosser2017, Rustad2009, Galagan2013}. Despite our superficial knowledge about the importance and contributions of hypoxia in the lifestyle of \textit{M. tuberculosis}, we do not have a clear mechanism to genetically manipulate these responses or to differentiate the roles of hypoxia \textit{per se} from the activity of HIF-1$\upalpha$ signaling. Thus, going forward, new tools are going to be required to study not only the contributions of HIF-1$\upalpha$ in mycobacterial infections, but even more importantly, how those contributions intersect with the role of NFAT in inducing VEGFA production and angiogenesis in this environment.

\subsection{HIF for HIF's Sake}

Several groups, the most notable of which being Philip Elks's lab, have studied the contributions of HIF-1$\upalpha$ signaling to the immune response to mycobacterial infections, but there remain several needs as yet unaddressed. The Elks lab has utilized both dominant-negative and dominant-active versions of the zebrafish \textit{hif1ab} to modulate the activity of this pathway, particularly in neutrophils, and have found that activation of HIF-1$\upalpha$ prolongs inflammation and improve mycobacterial clearance, suggesting that at early time points, HIF-$\upalpha$ plays a protective role in infection \citep{Elks2011, Elks2013}. It was also found that HIF-1$\upalpha$ is important for the induction of TNF-$\upalpha$, which is a critical protective factor during infection \citep{Lewis2019, Flynn1995}. This work nicely complements work from Didier Stainier's lab, which used a combination of \textit{hif1aa} and \textit{hif1ab} mutant zebrafish and new macrophage-specific transgenic tools to manipulate the HIF-1$\upalpha$ signaling pathway and found that this pathway, specifically in macrophages, was critical for mediating developmental angiogenesis. Unlike the behaviors seen in neutrophils, specific expression of even a wild-type \textit{hif1ab} in macrophages was toxic to the cells and rendered them impotent \citep{Gerri2017}. This toxicity is likely due to sequestering of important binding partners, but may be due to metabolic changes that leave the cells unable to produce sufficient ATP, albeit by two divergent mechanisms -- wild-type \textit{hif1ab} may produce a strict reliance on glycolysis while dominant-negative \textit{hif1ab} may force oxidative phosphorylation at rates exceeding the ability to produce pyruvate. This evokes an important function of this pathway in these cells that current tools remain unable to address. 

To further the study of the HIF pathway in the context of mycobacterial infection, a set of new tools should be made: one is a set of reporter constructs to better identify both hypoxia and transcriptional induction of \textit{hif1ab} in macrophages and the other is a conditional approach to the expression of dominant negative and dominant active versions of HIF-1$\upalpha$ in macrophages, potentially enabling new cell-autonomous understanding of the role of this pathway in mycobacterial pathogenesis and angiogenesis.

Previous work in our lab using in situ hybridization for phd3 mRNA revealed the upregulation of this gene surrounding the mycobacteria, indicating hypoxia \citep{Oehlers2015}. The Elks lab generated a transgenic line using a bacterial artificial chromosome containing the promoter for \textit{phd3} that expresses GFP \citep{Santhakuma2012}, which serves as a useful spatial and temporal readout for HIF-1$\upalpha$ transcription factor activity across different tissues. However, this tool is unable to distinguish between normoxic and hypoxic activation or different cell types, so new tools are required to better address these questions.

During conditions of hypoxia, HIFs are degraded through hydroxylation in the oxygen dependent degradation domain (ODD) that contains two proline residues that are hydroxylated by PHD proteins, leading to proteasomal degradation. This ODD has been shown to be both necessary and sufficient to direct oxygen-dependent degradation, so it seems reasonable to use a macrophage-specific promoter to drive expression of ODD linked to a fluorescent protein as a reporter for granuloma hypoxia. This would be stabilized at low oxygen concentration while being constitutively degraded under normoxia. This would allow for a clearer report of the degree of present hypoxia and complement existing tools, including hypoxyprobe \citep{Cousins2016, Huang1998}.

In parallel, a reporter is needed to provide a readout of normoxic activation of HIF-1$\upalpha$, which is predominantly thought to be regulated at the transcriptional level. Therefore, either ectopic expression or direct protein fusion strategies would be appropriate to the study of this pathway. If some promoter could be identified that responded comparably to the native \textit{hif1ab} promoter or a CRISPR-mediated knockin could be generated, this would be a useful reporter of transcription-level induction of HIF-1$\upalpha$, a process likely relevant to inflammatory responses during infection.  This, in tandem with the previously mentioned ODD transgenes would allow for a more thorough dissection of the relative contributions of hypoxia \textit{per se} and the activity of HIF-1$\upalpha$.

As previous transgenic attempts have failed, the expression of dn-hif1ab and da-hif1ab in macrophages will require new approaches. It seems that misregulation of HIF-1$\upalpha$ results in some sort of developmental toxicity in the macrophage, so it is critical that HIF is only modulated in the time and place where it is most relevant. Thus, using established estradiol-responsive constructs, a set if fish should be made expressing dn-hif1ab and da-hif1ab covalently linked to ER50 degradation domains, which sequester proteins in the cytosol and target them for degradation except when in the presence of tamoxifen. This would allow for the creation of HIF-modulating transgenes within macrophages that would be expected to have reduced toxicity and allow for time-targeted modulation of this pathway. Like the previous tools, this seems especially relevant in the context of mycobacterial granulomas from adult zebrafish, which are known to be hypoxic and more closely resemble human granulomas. These are likely to reveal not only new aspects of HIF modulation of angiogenesis, but broader impacts on the overall response to infection, especially given the central role of HIF-1$\upalpha$ in altering macrophage metabolism.

\subsection{Macrophage Metabolism in Immunity}

Upon activation, macrophages undergo a metabolic switch that corresponds, in part, to their longevity and function. While immediate-responding macrophages begin to utilize aerobic glycolysis for metabolism as a way to rapidly (albeit inefficiently) generate energy, longer-term macrophages utilize oxidative phosphorylation to optimize energy consumption over a long course of engagement. Thus, while macrophages may initially use bursts of glycolysis to fuel rapid, oxidizing responses, this is unsustainable over the long term and these macrophages must either transition to the use of oxidative phosphorylation, disperse, or perish. These initial responses are regulated in part by the activation of HIF-1$\upalpha$. Wound sites and other sites of insult tend to have a reduced oxygen concentration, which drives this metabolic shift as well as the previously described induction of VEGFA to alleviate this hypoxia. 

The earliest events of the glycolytic switch in macrophages is triggered by the inducible nitric oxide synthase, which generates NO to inhibit mitochondrial respiration. The net effect is a reduction in oxidative phosphorylation capacity by the cell and a need for the utilization of an alternative mode of metabolism, which is supplied by glycolysis. Subsequently, HIF-1$\upalpha$ activation drives the transcription of glucose transporters and lactate dehydrogenase to block pyruvate utilization by the mitochondria. HIF-1$\upalpha$, as we have seen, also induces the expression of cytokines and chemokines, making it an integrated part of the overall response to detection of PAMPs, the model of which is LPS (see \autoref{lps}). As a major mediator of inflammatory responses, this glycolytic phenotype allows for rapid, energetically expensive response to immediate threats but is unsustainable over the long-term, leading to a shift back to oxidative phosphorylation.

In order to switch back to oxidative phosphorylation, HIF-1$\upalpha$ must be downregulated and alternate pathways induced; rather than nitric oxide synthase consuming cellular arginine to make nitric oxide and citrulline, arginase needs to be induced to produce urea and ornithine. Additional amino acid biosynthesis pathways also seem to contribute to the process of granuloma formation, but the precise roles of these are somewhat unclear. One of the issues with the study of mutants in discrete genes in these pathways is that much of this has been done \textit{in vitro} under the standard binary LPS or IL-4/IL-13 polarization conditions or in the mouse, where more comprehensive profiling is possible but the knockout of \textit{Arg1} or \textit{Nos2} might not fundamentally change the cellular identity of these populations although it may shift them on the M1-M2 dichotomy by the formal definitions. Nevertheless, the relative contributions of HIF-1$\upalpha$ to the metabolic state of the macrophages depends on the interplay of formal hypoxia as well. 

Under sustained immunological stimulation in hypoxia, such as that found in a tuberculous granuloma, metabolic reprogramming toward oxidative phosphorylation still occurs in macrophages. The conflict between the production of VEGF, the M2-like identity of many granuloma macrophages that express VEGF, and the downregulation of HIF-1$\upalpha$ in M2 macrophages raises important questions about the factors that were presumed to be regulating VEGF production prior to the present work, which presents the alternative model of a C-type lectin receptor-NFAT transduction pathway able to induce VEGF. It may be that, under this model, that the metabolic status of the macrophages is unimportant for VEGF production, although this would be, to some degree, in contradiction with other parts of the literature. A further mystery is why VEGFA is so frequently considered an M2 cytokine when its transcription was thought to be dependent on HIF-1$\upalpha$, a gene induced under M1 conditions. A more thorough integration with the findings of this work with the general metabolic status of granuloma macrophages that produce VEGF normally will be important for resolving some of these quandaries on the contribution of HIF-1$\upalpha$ to granuloma macrophage metabolism, antibacterial responses, and angiogenesis.

\subsection{HIF-NFAT Interactions}

A tempting hypothesis is that HIF-1$\upalpha$ and NFATC2 are regulating one another to induce the expression of VEGFA. The model most consistent with the existing literature is that NFATC2 is upregulating the transcriptional expression of HIF-1$\upalpha$ under circumstances where HIF-1$\upalpha$ is already being stabilized. This would feed-forward the induction of VEGFA and may be essential for regulating these responses in normoxic environments or very early in infection. 

A piece of standalone support for this model is a study conducted in mast cells that demonstrated a calcineurin-NFAT dependent upregulation of HIF-1$\upalpha$ at the transcriptional and protein levels. Treatment of these mast cells with ionomycin, which increases cytosolic calcium concentrations, caused these effects in a way that was able to be inhibited by tacrolimus treatment. Additionally, this effect was exacerbated by culture under 1\% O\textsubscript{2}, a description that corresponds well to the behavior of NFATC3 in the myocardium, where hypoxia increased expression of endothelin-1 in a way that is sensitive to superoxide and the activity of NFATC4, which can also be activated by hypoxia \citep{deFrutos2011, RamiroDiaz2014, Moreno2015}. Additional studies found that hypoxia activated NFATC3 to promote pulmonary smooth muscle proliferation in a way that might promote pulmonary hypertension \citep{Hou2013}. Hypoxia can also increase proliferation of human fibroblasts through NFATC2 independently of HIF-1$\upalpha$ but dependent on HIF-2$\upalpha$ \citep{Senavirathani2018}. None of these studies but the last investigated a mechanism to integrate this activation of NFAT under hypoxia with the roles of HIF-1$\upalpha$ despite these phenotypes having been previously described to correspond with HIF-1$\upalpha$ activity \citep{Cui2021, Qi2017, Li2014, Thackaberry2002, SonanezOrganis2016}. Even the \citeauthor{Senavirathani2018} only investigated this effect within a relatively limited scope (proliferation) and not in respect to the diversity of phenotypes that are stimulated by hypoxia.

To study these interactions, it seems that combinatorial chemical inhibition strategies could be utilized, although HIF-1$\upalpha$ inhibitors are relatively sparse in number and understudied \citep{Viziteu2016}. However, this would only scratch the surface in pursuit of epistatic interactions between these proteins and would neglect the potential for protein-protein interactions in mediating part of the effect. A structure based approach targeting known interacting domains on HIF-1$\upalpha$ and its DNA binding domain may reveal important transactivation role for HIF-1$\upalpha$ as well as resolve some questions about the direct versus indirect nature of NFAT regulation of hypoxia responses. Much of this work could be done in a heterologous model (HEK-293T cells, for instance) to more efficiently assess these types of interactions. Larger scale screening approaches by two-hybrid would be another potential avenue for such a project to go as the interacting partners of various domains of NFATC2 remain only partially resolved. 

\section{Cotranscriptional Interplay}

To further dissect the contributions of other transcription factors to the NFAT-dependent angiogenesis response, we can make use of our THP-1 macrophage platform to study the role of NFAT interacting partners in the overall response to \textit{M. tuberculosis} exposure. While wild-type NFATC2 is able to interact with a panoply of other transcription factors through both C- and N-terminal domains, the domains important for particular binary interactions have been teased out over time, largely by Anjana Rao's lab. Thus, to simultaneously test the sufficiency of NFATC2 in inducing transcriptional responses, the importance of AP-1 transcription factor binding, and the ability for NFATC2 itself to bind DNA, I have developed a set of expression plasmids that drive expression of the following:

\singlespacing

\begin{center}
\begin{table}[h]
\caption{Lentiviral expression constructs to assess the role of NFATC2 domains for induction of VEGFA signaling.}
\label{table:canfat} \tabularnewline
\vspace{0.5cm}
\begin{tabular}{|p{2in}|p{3in}|}
\hline
\thead{Plasmid} & \thead{Utility} \tabularnewline
\hline
pLEX:mPapaya & Empty expression vector driving expression of only the conjugated fluorescent protein, for background comparison. \tabularnewline
\hline
pLEX:CA-NFAT1 & Expression of a constitutively active NFATC2 that drives transcriptional responses independent of calcineurin. \tabularnewline
\hline
pLEX:CA-NFAT1-$\upDelta$DBD & Expression of a constitutively nuclear NFATC2 that is unable to bind DNA, to assess the contributions of NFAT binding on the induction of VEGFA. \tabularnewline
\hline
pLEX:CA-NFAT1-$\upDelta$RIT & Expression of a constitutively active NFATC2 unable to interact with AP-1 transcription factors, to determine the contribution of this family to the VEGFA response. \tabularnewline
\hline
pLEX:CA-NFAT1-$\upDelta$DBD-$\upDelta$RIT & Expression of a constitutively nuclear DNA binding domain mutant also unable to interact with AP-1 transcription factors. \tabularnewline
\hline
\end{tabular}
\end{table}
\end{center}

\doublespacing

These plasmids will enable a genetics-first dissection of the activity of NFATC2 and its potential sufficiency to induce transcription of VEGFA. While, as noted in \autoref{pap:disc} and seen in \autoref{fig:nfatpro}, the VEGFA promoter has a number of putative NFAT binding sites, the activity of these remains unknown. Using these new genetic tools in the defined environment of THP-1 macrophages it should be possible to dissect the function of these sites. Distant future approaches may utilize deeper genetic probing to identify particular sites of especial importance, as has been done in \citet{Chang2004}. While that publication found one site that was bound by NFAT in the myocardium, the particular binding site of importance may vary by cell type and circumstance. 

It is also possible that non-AP-1 transcription factors are important for NFAT activity and this is a trickier proposition to address, but could be done through either immunoprecipitation-mass spectrometry or biotin tagging approaches via BioID. The advantage of the former is the identification of more stable interactions and could be done in tandem with DNase-seq, ChIP-seq, or ATAC-seq to identify promoter occupancy of this isoform. The latter may be a technically less challenging approach, however, and may provide a useful readout of the total set of interactions experienced by NFATC2 over the course of $\upgamma$Mtb exposure. The former can also be done in genetically unmodified cells while the latter would require some (minimal) additional cloning and generation of yet another lentivirus for the purpose of tagging NFATC2 with one of the several biotin ligases now available \citep{Cho2020}. Both seem useful and complementary and should be considered going forward to more comprehensively understand the NFAT interactome during mycobacterial infections. I hypothesize that there are as-yet unobserved protein-protein interactions between NFATC2 and HIF-1$\upalpha$ that would potentially shed light on the transcriptional regulation of VEGFA in diverse contexts including in cancer and autoimmunity. 

\subsection{NFAT:AP-1 Interactions}

\citep{Macian2001}

\subsection{NFAT and Other Transcription Factors}

\section{New Genetics Tools for the Study of NFAT Signaling}

One of the dominant tools in the field for the study of intracellular calcium flux is the use of GCaMP, a modified green fluorescent protein that fluoresces in response to calcium binding \citep{Nakai2001}. Since its initial development, many interactions have developed allowing for ever-finer detection of various aspects of cellular calcium concentrations and localization. These tools have been used in the fish to detect and manipulate cellular behavior in both fixed and freely moving fish \citep{Beerman2015, Kim2017}. These tools have clear promise in better understanding the biology of NFAT activation, but likely need tethering to either the channels or proteins themselves or a specific cellular compartment to increase spatial resolution; a whole-cell approach is no longer sufficient for the proper understanding of NFAT activity in this context and finer resolution would greatly aid in the identification of future mechanisms. Given findings from \citet{Kar2015} on the importance of nuclear calcium in regulating the activity of at least NFATC3, it would be beneficial to have new tools to monitor these alterations in real time. 

An initial approach would take advantage of some of the new technologies available in the zebrafish (reviewed in \autoref{newtech}) by generating a GCaMP-tagged \textit{nfatc2a} in the zebrafish to monitor the timing of calcium flashes and how those correspond to the activation state of the protein, able to be visualized by intercompartmental trafficking. By using high temporal and spatial resolution imaging offered by LightSheet \citep{Reynaud2008}, it would be possible to monitor on the seconds resolution the activation status of Nfatc2a within macrophages in response to mycobacterial infection. This finely tuned reporting behavior would reveal key details about the timing and kinetics of NFAT activation during infection and how that activation corresponds to the induction of \textit{vegfaa} and angiogenesis.

\section{Dissection of the NFAT Isoforms in Macrophages}

\citep{Shiau2015} % Talk about how to dissect macrophage-dependent vs. macrophage-independent responses.

\section{Mycobacterial Interactions with Other Aspects of Vascular Biology}

\citep{Correa2014}
\citep{ClaessonWelsh2015}
\citep{Eklund2017}
\citep{Oehlers2017}
\citep{Hato2008}
\citep{Sakamoto2010}
\citep{Keskin2015}
\citep{Shin2016}

\section{Lymphangiogenesis}

This project has focused strictly on the proliferation and growth of vascular endothelial cells during mycobacterial infections, a process known as angiogenesis. However, a parallel process exists concerning the less popularly known lymphatic vascular system and this is known as lymphangiogenesis. While this process is less extensively studied, it is a critical component of the vascular development of vertebrate organisms and is essential for proper fluid homeostasis and immune system function, as lymphatic vessels are the route along which antigen presenting cells "drain" into the lymph nodes to prime the adaptive immune system. Failures of lymphangiogenesis result in lymph\oe dema and general problems with fluid balance. In the context of disease, the lymphatic system is essential for proper immune behavior, but during chronic conditions like cancer or tuberculosis, lymphangiogenesis can be subverted by the insult for their own benefit. In cancer, lymphangiogenesis is utilized to provide additional routes of metastasis away from the primary tumor; the role in tuberculosis is comparatively less well studied.

\citep{Alitalo2005}
\citep{Bower2017a}
\citep{Bower2017b}
\citep{Bussmann2010}
\citep{Campuzano2017}
\citep{Dietrich2007}
\citep{Duong2012}
\citep{Hogan2009}
\citep{Wong2017b}
\citep{Makinen2001}
\citep{LeGuen2014}
\citep{Kuchler2006}
\citep{Haiko2008}
\citep{Stacker2014}
\citep{Nicenboim2015}
\citep{Onder2017}
\citep{Okuda2012}
\citep{Han2017}
\citep{Jung2017}
\citep{Harding2015}
\citep{vanLessen2017}
\citep{Shin2017}

\section{Generalizability}

Cryptococcus

\citep{Bojarczuk2016, Lin2006b}

\section{A Pressing Need for More Objective Approaches to Image Analysis}

\citep{Heath2017}

%\item Novel Methods to Automate Measurement of Angiogenesis
%\item Novel Methods to Automate Cell Feature Quantitation

\section{Closing Remarks}


Ist jes ene nenii frikativo , hej op kuzo respondvorto. Ts frazparto komentofrazo iam, giga aliio ci hop. Ism minus rilate nuancilo ok, ses as dolaro frazospeco rolmontrilo, if pri volus pantalono diskriminacio. Mi mem plej rolvortajo, dume horo centimetro uj jes. \cite{Jones2002} As we have seen in section \ref{subsec:bigponies}, Big Ponies rule!

\fig
\begin{center}
\epsfig{figure=images/hertz.eps}
\caption{Venn Diagram}
% Provide a label so we can cross-reference it from the tex
\label{venn.figure}
\end{center}
\efig

%\begin{figure}
%\begin{center}
%\includegraphics[height=.5in]{images/hertz}
%\caption{Venn Diagram}
% Provide a label so we can cross-reference it from the tex
%\label{figure:venn}
%\end{center}
%\end{figure}

Okupi identiga kuo bo, via oble bek'o komentofrazo\footnote{Dume horo centimetro uj jes 1884.} ot, trema ilion negativaj cis nk. Co ebl malsupera kvadriliono, iz duono malantaue tiu, milo franjo ato al. Des solinfano parentezo hu. Peti responde tc ioj, ej tempismo pronomeca praantaulasta igi. Per nedifina popolnomo nk, ki ekoo kune sat. Hav frota akuzativo ar.


So ebl poste posta nombrovorto, nul be fine jugoslavo kontraui. Sub ac deka sube, orda hiper u jam. Plu onin iometo ej, os peti irebla per. Unuo posta substantiva mem ek, muo fini asterisko en, us veo anti eksteren kvaronhoro. Ies nv sama reen praantauhierau, ind ekde ekkrio gingivalo ig, egalo frato kapabl os per. De por fora ofon altlernejo.

%\begin{thebibliography}{99}
%\bibitem{a} Author. Title. 1857
%\bibitem{b} Author 2. Title. 1916
%\bibitem{c} Author 3. Title. 1961
%\end{thebibliography}
\nocite{*}
\bibliographystyle{cell}
\bibliography{lit}



\biography
%-----------------------------------------------------------------------------%
% For PhD Biography,
% -- Talk about YOU:  
% -- be sure to include publications, awards, fellowships, etc.
%-----------------------------------------------------------------------------%

Jared was born in the mountains of Appalachia in Barbourville, KY in 1994. After graduating valedictorian from Barbourville High School, he enrolled at Transylvania University in Lexington, KY where he received a Bachelor of Arts degree in Biology and Political Science, \textit{summa cum laude}. In the fall of 2016, he began his Ph.D. at Duke University in the Molecular Genetics and Microbiology department, having been awarded a James B. Duke Fellowship. He then joined the lab of David Tobin in the summer of 2017. In his time at Duke, he was awarded a Ruth L. Kirschstein National Research Service Award F31 fellowship from the National Heart, Lung, and Blood Institute.

\end{document}